%*****************************************************************************************
%*********************************** First Chapter ***************************************
%*****************************************************************************************

\chapter{Introduction}  %Title of the First Chapter

\ifpdf
    \graphicspath{{Chapter1/Figs/Raster/}{Chapter1/Figs/PDF/}{Chapter1/Figs/}}
\else
    \graphicspath{{Chapter1/Figs/Vector/}{Chapter1/Figs/}}
\fi


%********************************** %First Section  **************************************
\section{The Sun} %Section - 1.1 
The Sun has been the subject of study by humanity since the dawn of civilisation.
The processes in the Sun and in its atmosphere have a direct impact on life on Earth.
The radiation that reaches the Earth provides the energy life needs to flourish and the interaction of the solar wind with the upper atmosphere generates the aurora.
Modern study of the Sun focuses on understanding the processes that drive changes in the Sun such as total radiation output and dramatic events such as solar flares or coronal mass ejections.

The properties and behaviour of the Sun change dramatically with distance from the centre of the Sun.
The core of the Sun is where the nuclear fusion reaction occurs, this region is the source of energy for the Sun and the rest of the plasma above this layer transports this energy to the photosphere and beyond.
Above the solar core the first large region of plasma is the region where energy transport by radiation dominates, this region is very dense and it takes a single high energy photon of light a long time to move outwards.
This radiative zone extends out to XXX, at that point a narrow region called the tachocline exists, where the plasma stops rotating as a solid body, and starts rotating with different velocities at different latitudes.
Above this tachocline radial energy transport is dominated by convective motions, this convective zone extends out to the visible surface or photosphere.
The convective plasma motions that move hot plasma up to the photosphere are responsible for a lot of the interesting features and properties of the photosphere and higher layers of the solar atmosphere.
The photosphere is the point where the Sun becomes mostly transparent to light, it is defined as the region where $\frac{2}{3}$ of the light can escape the solar atmosphere entirely.
However, it is the photosphere and the layers above it that have the most direct influence on the Earth.

