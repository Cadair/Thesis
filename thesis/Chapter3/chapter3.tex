% !TeX root = ../smumford_thesis.tex
%*****************************************************************************************
%*********************************** Third Chapter ***************************************
%*****************************************************************************************
\chapter{Methodology}\label{ch:methodology}  %Title of the Second Chapter

\begin{pycode}[chapter3]
from __future__ import print_function

ch3 = texfigure.Manager(pytex, number=3, base_path='./Chapter3/')
\end{pycode}

\section{Magnetohydrostatic Background Conditions}\label{sec:mhsbackground}

The numerical simulations performed for the MHD wave mode analysis in this thesis, all make use of the ability of SAC (\cref{sec:SAC}, \cite{shelyag2008}) to solve for perturbations on top of a static MHD equilibria.
To perform these experiments, a background atmosphere needs to be constructed.
Quiet Sun regions were chosen for study, therefore this section details the construction of an atmosphere representative of the quiet Sun.
This choice reduces the complexity of the modelled magnetic field and should allow for more accurate separation of the MHD wave effect from any other energy transfer processes, such as bulk flows.

The first step in constructing a static model for the solar atmosphere is to understand the hydrodynamic properties of the background plasma.
Models of properties such as the density and pressure have been created for various heights in the solar atmosphere, based on observations and theory.
The profile used in this thesis is the \cite{vernazza1981} model C, which covers the range of height in the solar atmosphere from the photosphere ($0$ km) to $2.5$ Mm above the photosphere.
A plot of the density, pressure and temperature from this model is shown in \cref{fig:val3c}.

\begin{pycode}[chapter3]
import pysac.mhs_atmosphere as atm

#Read in the VAL3C model
empirical_data = atm.hs_atmosphere.read_VAL3c_MTW(MTW_file=False)


# Create a Z array at the interpolated resolution and interpolate.
ZZ = u.Quantity(np.linspace(empirical_data['Z'][0], empirical_data['Z'][-1], 128), unit=empirical_data['Z'].unit)
table = atm.hs_atmosphere.interpolate_atmosphere(empirical_data, ZZ, s=0)


# Create a figure and make space for the axes.
fig, ax = plt.subplots(gridspec_kw={'right':0.78, 'left':0.16, 'bottom':0.13})

# Shortcut all the Mm conversion.
Z = empirical_data['Z'].to(u.Mm)

lrho, = ax.plot(Z, empirical_data['rho'].quantity.si, 'x', color='blue')
lrho_i, = ax.plot(ZZ.to(u.Mm), table['rho'].quantity.si, color='blue')

ax2 = ax.twinx()
lp, = ax2.plot(Z, empirical_data['p'].to(u.Pa), 'x', color='green')
lp_i, = ax2.plot(ZZ.to(u.Mm), table['p'].to(u.Pa), color='green')


ax3 = ax.twinx()
ax3.spines["right"].set_position(("axes", 1.2))
lt, = ax3.plot(Z, empirical_data['T'].to(u.K), 'x', color='red')
lt_i, = ax3.plot(ZZ.to(u.Mm), table['T'].to(u.K), color='red')


# Set primary axes properties and labels
ax.semilogy()
ax.set_ylabel(r"Density [{}]".format(lrho._yorig.unit._repr_latex_()))
ax.set_xlabel(r"Height [{}]".format(lrho._xorig.unit._repr_latex_()))
ax.set_xlim(Z[0].value-0.1, Z[-1].value+0.1)


# Pressure Axis
ax2.semilogy()
ax2.set_ylabel(r"Pressure [{}]".format(lp._yorig.unit._repr_latex_()))


# Temp axis
ax3.set_ylabel(r"Temperature [{}]".format(lt._yorig.unit._repr_latex_()))

ax.set_xlim([-0.02,1.62])
ax3.set_ylim([3500,7500])
# Set the colours for the ticks and the labels.
ax.tick_params(axis='y', colors=lrho.get_color())
ax2.tick_params(axis='y', colors=lp.get_color())
ax3.tick_params(axis='y', colors=lt.get_color())

ax.yaxis.label.set_color(lrho.get_color())
ax2.yaxis.label.set_color(lp.get_color())
ax3.yaxis.label.set_color(lt.get_color())

val3c = ch3.save_figure('val3c', fig, fext='.pgf')
val3c.caption = r"Graph of the \cite{{vernazza1981}} model C showing the density ({}), pressure ({}), and temperature ({}). The solid lines show the interpolated values used to construct the numerical model, at the resolution of the numerical domain.".format(lrho.get_color(), lp.get_color(), lt.get_color())
\end{pycode}

\py[chapter3]|val3c|

On top of this hydrostatic background, a magnetic field is required to study MHD waves.
The type of magnetic phenomena to be investigated are small scale photospheric structures, that occur frequently over the disk of the Sun.
Magnetic Bright Points (MBPs) were chosen as an observational feature to use as a guide for the background model, MBPs are well studied and there are good estimates of their physical properties.
\cite{feng2013} performed a study of Photospheric Bright Points (PBPs), which are assumed to be analogous to MBPs, using the Dutch Open Telescope, and found that the peak of the log-normal distribution gave a diameter of $232\pm40$ km for the quiet Sun.
They also show that $50$\% of the PBPs they analysed had the major axis of their ellipse less than $1.5$ times longer than the minor axis, showing that they are to a first approximation circular, however, with some PBPs having quite a large deviation from this approximation.
\cite{utz2013} performed a similar study investigating the average magnetic field strength of MBPs using the Hinode SOT instrument.
They find that in a quiet Sun region the average magnetic field strength of a MBP is $135.0 \pm 1.0$ mT.
\cite{sanchezalmeida2004} also performed a statistical study of MBPs, finding that in their sample that most MBPs have lifetimes of less than 10 minutes.
They also note, and present some evidence, that this is probably an upper bound estimate, and true lifetimes may be shorter.


\begin{pycode}[chapter3]
from pysac.mhs_atmosphere.parameters.model_pars import mfe_setup as model_pars
import pysac.mhs_atmosphere as atm

# Cheeky Reset to Photosphere
model_pars['xyz'][4] = 0*u.Mm
#==============================================================================
# Build the MFE flux tube model using pysac
#==============================================================================
# model setup
scales, physical_constants = atm.units_const.get_parameters()
option_pars = atm.options.set_options(model_pars, False, l_gdf=True)
coords = atm.model_pars.get_coords(model_pars['Nxyz'], u.Quantity(model_pars['xyz']))

#interpolate the hs 1D profiles from empirical data source[s]
empirical_data = atm.hs_atmosphere.read_VAL3c_MTW(mu=physical_constants['mu'])
table = atm.hs_atmosphere.interpolate_atmosphere(empirical_data, coords['Zext'])
table['rho'] += table['rho'].min()*3.6

# calculate 1d hydrostatic balance from empirical density profile
# the hs pressure balance is enhanced by pressure equivalent to the
# residual mean coronal magnetic pressure contribution once the magnetic
# field has been applied
magp_meanz = np.ones(len(coords['Z'])) * u.one
magp_meanz *= model_pars['pBplus']**2/(2*physical_constants['mu0'])

# Make the vertical profile 3D
pressure_z, rho_z, Rgas_z = atm.hs_atmosphere.vertical_profile(coords['Z'], table, magp_meanz, physical_constants, coords['dz'])

# Generate 3D coordinate arrays
x, y, z = u.Quantity(np.mgrid[coords['xmin']:coords['xmax']:1j*model_pars['Nxyz'][0],
                              coords['ymin']:coords['ymax']:1j*model_pars['Nxyz'][1],
                              coords['zmin']:coords['zmax']:1j*model_pars['Nxyz'][2]], unit=coords['xmin'].unit)

# Get default MFE flux tube parameters out of pysac
xi, yi, Si = atm.flux_tubes.get_flux_tubes(model_pars, coords, option_pars)

# Generate the 3D magnetic field
allmag = atm.flux_tubes.construct_magnetic_field(x, y, z, xi[0], yi[0], Si[0], model_pars, option_pars, physical_constants, scales)
pressure_m, rho_m, Bx, By ,Bz, Btensx, Btensy = allmag

# local proc 3D mhs arrays
pressure, rho = atm.mhs_3D.mhs_3D_profile(z, pressure_z, rho_z, pressure_m, rho_m)
magp = (Bx**2 + By**2 + Bz**2) / (2.*physical_constants['mu0'])
energy = atm.mhs_3D.get_internal_energy(pressure, magp, physical_constants)
\end{pycode}

\begin{pycode}[chapter3]
#### YT STUFF ####

import yt

# Add derived Fields
def magnetic_field_strength(field, data):
    return np.sqrt(data["magnetic_field_x"]**2 + data["magnetic_field_y"]**2 + data["magnetic_field_z"]**2)
yt.add_field(("gas","magnetic_field_strength"), function=magnetic_field_strength, units=yt.units.T.units)

def alfven_speed(field, data):
    return np.sqrt(2.*data['magnetic_pressure']/data['density'])
yt.add_field(("gas","alfven_speed"), function=alfven_speed, units=(yt.units.m/yt.units.s).units)

bbox = u.Quantity([u.Quantity([coords['xmin'], coords['xmax']]),
                   u.Quantity([coords['ymin'], coords['ymax']]),
                   u.Quantity([coords['zmin'], coords['zmax']])]).to(u.m).value

# Now build a yt DataSet with the generated data:
data = {'magnetic_field_x':yt.YTQuantity.from_astropy(Bx.decompose()),
        'magnetic_field_y':yt.YTQuantity.from_astropy(By.decompose()),
        'magnetic_field_z':yt.YTQuantity.from_astropy(Bz.decompose()),
        'pressure': yt.YTQuantity.from_astropy(pressure.decompose()),
        'magnetic_pressure': yt.YTQuantity.from_astropy(magp.decompose()),
        'density': yt.YTQuantity.from_astropy(rho.decompose())}

ds = yt.load_uniform_grid(data, x.shape, length_unit='m', magnetic_unit='T', mass_unit='kg', periodicity=[False]*3, bbox=bbox)
\end{pycode}

\begin{pycode}[chapter3]
from astropy.modeling import models, fitting

x_lin = np.linspace(coords['xmin'], coords['xmax'], model_pars['Nxyz'][0])
bmag = np.sqrt((Bx**2 + By**2 + Bz**2))[:,64,0].to(u.mT)

gaussian = models.Gaussian1D()
bmag_fit = fitting.LevMarLSQFitter()(gaussian, x_lin, bmag)

fwhm = 2.*np.sqrt(2*np.log(2))*bmag_fit.stddev.value
fwhm = np.abs(fwhm) * u.Mm

fwtm = 2.*np.sqrt(2*np.log(10))*bmag_fit.stddev.value
fwtm = np.abs(fwtm) * u.Mm

fwhm = fwhm.to(u.km)
fwtm = fwtm.to(u.km)
\end{pycode}

% Fred's shortcuts
\newcommand{\BO}{{B_{0z}}}
\newcommand{\GO}{{G}}
\newcommand{\bc}{{b_{00}}}
\newcommand{\bF}{{b_{01}}}
\newcommand{\za}{{z_{1}}}
\newcommand{\bb}{{b_{02}}}
\newcommand{\zb}{{z_{2}}}

From this information, and taking into account the properties of the hydrostatic background chosen, a magnetic field can be constructed with properties similar to the MBPs.
The 3D magnetic field will be generated using work of \cite{gent2013, gent2014}, and the implementation of that paper in the pysac library\footnote{https://github.com/SWAT-Sheffield/pysac}.
The magnetic field will be constructed as a self-similar field, in the same manner as \cite{schluter1958}, it is constructed for a axis-symmetric flux tube, with a Gaussian radial profile given in \cref{eq:G}.
The vertical profile of the magnetic field, along the axis of the flux tube, is given in \cref{eq:BO}, as a summation of two exponentials, one for controlling the profile in the low atmosphere and one for controlling the profile in the upper atmosphere.
The magnetic field is therefore specified by \cref{eq:Bxyz,eq:f,eq:G,eq:r,eq:BO}, which are modified versions of the equations presented in \cite{gent2014}, simplified for a single flux tube model.

\begin{equation}\label{eq:Bxyz}
\begin{aligned}
B_{bx} &= -S(x-x_0) {\, \BO\GO \:} \frac{\partial \BO}{\partial z},
\\
B_{by} &= -S(y-y_0) {\, \BO\GO \:} \frac{\partial \BO}{\partial z},
\\        
B_{bz} &= \phantom{-}S{\BO^2\GO  \,} + \bc,
\end{aligned}
\end{equation}


\begin{align}
f &= \frac{r^2 B_0z^2}{2},  \label{eq:f}
\\
G &= \exp\left(-\frac{f}{f_0^2}\right),  \label{eq:G}
\\
r \,   &= \sqrt{(x-x )^2+(y-y )^2},\\  \label{eq:r}
\end{align}

\begin{equation}\label{eq:BO}
\BO = 
\bF\exp\left(-\frac{z}{\za}\right)\,
+
\bb\exp\left(-\frac{z}{\zb}\right),
\end{equation}
where $B_{bi}$ are the components of the background magnetic field; $f_0$ is the horizontal scaling length; $x_0,y_0,z_0$ are the coordinates of the footpoint of the field where $z_0=0$ which is considered to be the photosphere as shown in \cref{fig:val3c}; $b_{01}$ and $b_{02}$ are two real constants that sum to $1$ and specify the relative strength of the two exponentials controlling the vertical field strength profile, while $z_1$ and $z_2$ control the field strength profile with height; finally $S$ is the footpoint field strength.
Similarly to \cite{gent2014} a uniform background field ($\bc$) is also added to the model.
This weak ambient vertical field provides for a more realistic atmosphere, instead of an atmosphere where there is only one magnetic flux tube present.
The parameters used in constructing the background atmosphere in this thesis are given in \cref{tab:bgparams}.

\begin{table}
    \centering
    \begin{tabular}{|c|c|}
        \hline Parameter & Value \\ 
        \hline $\bc$ & \py[chapter3]|model_pars['B_corona'].to(u.mT)| \\ 
        \hline $\bF$ & \py[chapter3]|model_pars['coratio']| \\ 
        \hline $\bb$ & \py[chapter3]|model_pars['chratio']| \\ 
        \hline $f_0$ & \py[chapter3]|model_pars['radial_scale']| \\ 
        \hline $\za$ & \py[chapter3]|model_pars['chrom_scale']| \\ 
        \hline $\zb$ & \py[chapter3]|model_pars['corona_scale']| \\
        \hline $S$   & \py[chapter3]|Si[0][0].to(u.mT)| \\
        \hline
    \end{tabular}
    
    \caption{Table showing all the parameters for \cref{eq:Bxyz,eq:f,eq:G,eq:r,eq:BO} used in constructing the background atmosphere for the simulations in this thesis.}
    \label{tab:bgparams}
\end{table}

\begin{pycode}[chapter3]

gaussian = texfigure.MultiFigure(2, 1, "guassian")
# Make a 1D plot with Gaussian fit.

fig, ax = plt.subplots(figsize=texfigure.figsize(pytex))#, gridspec_kw={'left':0.15})
lb, = plt.plot(x_lin, bmag.to(u.mT), 'x')
X = np.linspace(coords['xmin'], coords['xmax'], 1000)
lb_f, = plt.plot(X, bmag_fit(X.value), color='green')

plt.xlabel("X [{}]".format(lb._xorig.unit._repr_latex_()))
plt.ylabel("Magnetic Field Strength [{}]".format(lb._yorig.unit._repr_latex_()))

lhm = plt.axhline(y=bmag_fit.amplitude.value/2, color='cyan', linestyle=':')
ltm = plt.axhline(y=bmag_fit.amplitude.value/10, color='magenta', linestyle=':')

plt.axvline(x=-1*fwhm.to(u.Mm).value/2., color='cyan', linestyle=':')
plt.axvline(x=fwhm.to(u.Mm).value/2., color='cyan', linestyle=':')

plt.xlim([-0.5, 0.5])

bmagf = ch3.save_figure('bmag-footpoint', fig, fext='.pgf')
bmagf.subfig_width = r'\columnwidth'
bmagf.caption = "1D slice through the centre of the domain. The {} line shows a fitted gaussian to the numerical domain. The {} lines show the FWHM and the {} line shows the FWTM. \citep{{theastropycollaboration2013}}".format(lb_f.get_color(), lhm.get_color(), ltm.get_color())
gaussian.append(bmagf)
\end{pycode}

\begin{pycode}[chapter3]
slc = yt.SlicePlot(ds, 'z', 'magnetic_field_strength', origin='center-domain',
axes_unit='km', center=("max", "magnetic_field_strength"), width=(1.0, "Mm"))
slc.set_figure_size(texfigure.figsize(pytex, scale=0.8)[0])

slc.set_cmap('all', 'Oranges')
bmax = ds.all_data().quantities.extrema("magnetic_field_strength")[1]
slc.annotate_contour('magnetic_field_strength', take_log=False,
plot_args={'levels':[bmax/2., bmax/10.],
'linewidths':3,
'colors':['black', 'green']})

bmag2d = ch3.save_figure('bmag-cut', slc)
bmag2d.subfig_width = r'0.8\columnwidth'
bmag2d.caption = "The image shows the the magnetic field magntitude from a 2D slice through domain. The black line shows the FWHM and the green line shows the FWTM of the gaussian profile."

gaussian.append(bmag2d)
gaussian.caption = "Magnetic field strength plots for the photospheric layer in the numerical domain."

\end{pycode}


\py[chapter3]|gaussian|

\begin{pycode}[chapter3]
slc = yt.SlicePlot(ds, 'x', 'alfven_speed', origin='lower-center-domain', axes_unit='Mm')
slc.set_figure_size(texfigure.figsize(pytex)[0])
slc.set_cmap('all', 'viridis')

seed_points = np.zeros([11,2]) + 1.52
seed_points[:,0] = np.linspace(-0.99, 0.95, seed_points.shape[0], endpoint=True)
slc.annotate_streamlines('magnetic_field_y', 'magnetic_field_z', field_color='magnetic_field_strength',
plot_args={'start_points':seed_points, 'density':15, 'cmap':'Blues', 'linewidth':2,
'norm': mpl.colors.LogNorm(*ds.all_data().quantities.extrema("magnetic_field_strength"))})

alf2d = ch3.save_figure('alf2d', slc)
alf2d.caption = r"Vertical slice through the background atmosphere, with the Alfv\'en Speed shown in the background with magnetic field lines overplotted in blue."
\end{pycode}

\py[chapter3]|alf2d|

The result of the flux tube construction in \cite{gent2013, gent2014} was a 2D Gaussian profile for the cross-sectional magnetic field, with a full width at half maximum (FWHM) of $\approx$ \py[chapter3]|np.round(fwhm, -1)|, and a full width at a 10th of maximum (FWTM) of \py[chapter3]|np.round(fwtm, -1)| at the photosphere, which corresponds to the lower bound of the observations in \cite{utz2013}.
The primary reason for the smaller footpoint, in comparison with the observations in \cite{utz2013}, is the hydrodynamical model used for the photosphere, which has an upper limit to the amount of magnetic pressure it can support.
Plots of the magnetic field strength in the photosphere are shown in \cref{fig:guassian}, for both a 1D and 2D profile.
A vertical slice through the background atmosphere is then shown in \cref{fig:alf2d}, with the structure of the magnetic field shown over-plotted on the Alfv\'en speed.
As can be seen in \cref{fig:alf2d,fig:domain-3d}, the magnetic field expands non-linearly as the kinetic pressure decreases with height.
It should also be noted that the background atmosphere has been constructed up to $1.6$ Mm above the photosphere defined in \cite{vernazza1981}.
This is a deliberate choice to exclude the region where the plasma $\beta = 1$.
The reasons for this choice lie in the studies to be performed using this atmosphere, this thesis is explicitly studying the generation of MHD waves in the photosphere and chromosphere and not their propagation into the higher layers of the solar atmosphere.
It is therefore easier to exclude the very interesting, but also complex physics that occurs around the $\beta = 1$ region.

Once the magnetic field has been constructed, the final step in constructing the background atmosphere is to `add' the magnetic field to the hydrostatic background.
To achieve this in a way that creates a physical and magnetohydrostatic background atmosphere, the physical effects of the presence of the magnetic field on the plasma must be taken into account.
This is done using the principal of magnetohydrostatic equilibrium as described in \cref{eq:mhs-condition},

\begin{equation}
\nabla P = 
\nabla p + \nabla \frac{|\vec{B}|^2}{2\mu_0} - (\vec{B}\cdot\nabla)
{\frac{\vec{B}}{\mu_0} }
= \rho \vec{g}, 
\label{eq:mhs-condition}
\end{equation}
%\nabla P = (\mathbf{B}\cdot \nabla)\mathbf{B} + \nabla\left(\frac{\mathbf{B}^2}{2}\right) + \nabla p = \rho\mathbf{g},
which describes the pressure balance between the kinetic and magnetic forces present in a MHD plasma;
where $P$ is the total pressure, $p$ is the kinetic pressure $\rho$ is the density, $g$ is the acceleration due to gravity (assumed to be constant), and $\mu_0$ is the vacuum magnetic permeability coefficient.
The solution to this equation with the magnetic field specified above is derived in \cite{gent2013,gent2014}.


The numerical domain chosen for the work presented in this thesis is $2.0 \times 2.0 \times 1.6$ Mm in size in the $x$, $y$ and $z$ directions, with a resolution of $128^3$ grid points, giving a physical size of $15.6\ \times\ 15.6\ \times\ 12.5$ km$^3$ for each grid cell.
A 3D rendering of the domain is shown in \cref{fig:domain-3d}, which shows the extent of the domain, the magnetic field geometry and the background density and thermal pressure.



\begin{pycode}[chapter3a]
ch3 = texfigure.Manager(pytex, number=3, base_path='./Chapter3/')
ch3.fig_count += 100

from mayavi import mlab
from mayavi.tools.sources import vector_field, scalar_field

mlab.options.offscreen = True

import pysac.yt

from pysac.analysis.tube3D import process_utils as mpu
from pysac.plot import mayavi_plotting_functions as mpf
from pysac.plot.mayavi_seed_streamlines import SeedStreamline

ds = pysac.yt.SACGDFDataset((ch3.data_file('Slog_p240-0_A10_B005_00001.gdf')))

xmax, ymax, zmax = ds.domain_dimensions - 1
#==============================================================================
# Create Mayavi Scene
#==============================================================================
fig = mlab.figure(1, bgcolor=(1, 1, 1), fgcolor=(0.5, 0.5, 0.5),
size=[1200,900])


#==============================================================================
# Create Mayavi datasets from yt dataset
#==============================================================================
cg = ds.index.grids[0]

bfield = vector_field(cg['mag_field_x']*1e3,
                      cg['mag_field_y']*1e3,
                      cg['mag_field_z']*1e3,
                      figure=None)
                      
bmag = mlab.pipeline.extract_vector_norm(bfield, name="Field line Normals")

density = scalar_field(cg['density']*1e6, name="Density", figure=None)

pk = mpu.yt_to_mlab_scalar(ds, 'thermal_pressure')

#==============================================================================
# Add a timestamp
#==============================================================================
#text = mlab.text(0.75,0.01,"t=%3.2f s"%ds.current_time)
#mpf.set_text(text.property)
#text.actor.text_scale_mode = 'none'
#text.property.font_size = 24


#==============================================================================
# Add Axes (Has to be done after first object added)
#==============================================================================
axes, outline = mpf.add_axes(np.array(zip(ds.domain_left_edge,
ds.domain_right_edge)).flatten()/1e8)


#==============================================================================
# Generate seeds for fieldlines and plot them in the image
#==============================================================================
xc = xmax/2
yc = ymax/2
seeds = [[xc,yc,zmax]]
for ti,r in enumerate(np.linspace(15, yc-5, 3)):
    dth = (2 * np.pi) / 5
    for theta in np.linspace(0, 2 * np.pi, 6, endpoint=False):
        theta += dth*ti
        seeds.append([r * np.cos(theta + 0.5 * ti) + xc,
                      r * np.sin(theta + 0.5 * ti) + yc, zmax])
seeds = np.array(seeds)
seed_points = mlab.points3d(seeds[:,0], seeds[:,1], seeds[:,2],
color=(0.231, 0.298, 0.752), scale_mode='none',
scale_factor=1.8)


field_lines = SeedStreamline(seed_points=np.array(seeds))
bmag.add_child(field_lines)

field_lines.module_manager.scalar_lut_manager.data_range = np.array([0.01,110])
field_lines.parent.scalar_lut_manager.lut.scale = 'log10'
#field_lines.module_manager.scalar_lut_manager.reverse_lut = True
field_lines.module_manager.scalar_lut_manager.lut_mode = 'BuGn'

flines_cbar = mpf.add_colourbar(field_lines, [0.84,0.68], [0.11,0.31], '')
vv_text = mpf.add_cbar_label(flines_cbar,'Magnetic Field Strength [mT]')
vv_text.property.font_size = 10
tw = 0.015
vv_text.width = tw
vv_text.property.bold = True

#==============================================================================
# Density Plane
#==============================================================================

image_plane_widget1 = mlab.pipeline.image_plane_widget(density)
image_plane_widget1.ipw.slice_position = 128.0
image_plane_widget1.parent.scalar_lut_manager.lut.scale = 'log10'
image_plane_widget1.module_manager.scalar_lut_manager.lut.table = plt.get_cmap('inferno')(range(255))*255
ipw1_cbar = mpf.add_colourbar(image_plane_widget1, [0.84,0.34], [0.11,0.31], '')
vv_text = mpf.add_cbar_label(ipw1_cbar,'     Density [mg / m3]')
vv_text.property.font_size = 10
vv_text.width = tw
vv_text.property.bold = True

image_plane_widget2 = mlab.pipeline.image_plane_widget(pk)
image_plane_widget2.ipw.plane_orientation = 'y_axes'
image_plane_widget2.parent.scalar_lut_manager.lut.scale = 'log10'
image_plane_widget2.module_manager.scalar_lut_manager.lut.table = plt.get_cmap('viridis')(range(255))*255
ipw2_cbar = mpf.add_colourbar(image_plane_widget2, [0.84,0.03], [0.11,0.31], '')
vv_text = mpf.add_cbar_label(ipw2_cbar, '   Thermal Pressure [Pa]')
vv_text.property.font_size = 10
vv_text.width = tw
vv_text.property.bold = True

#==============================================================================
# Magnetic field strength contours
#==============================================================================

cntr = mlab.pipeline.contour_grid_plane(bmag, name="B Contours")
cntr.grid_plane.axis = 'z'
cntr.grid_plane.position = 0
cntr.contour.auto_contours = False
Bmax = 119
cntr.contour.contours = [Bmax, Bmax/2, Bmax/10, Bmax/100]

cntr.actor.mapper.scalar_visibility = False
cntr.actor.property.color = (0.7843137254901961, 0.0, 0.0)

#==============================================================================
# Set view and Show
#==============================================================================

field_lines.module_manager.scalar_lut_manager.lut_mode = 'BuGn'

m3d = ch3.save_figure('domain-3d', fig, fext='.png', size=(1200,900),
                      azimuth=153, elevation=63, distance=420,
                      focalpoint=[ 75.,  16.,  30.])
m3d.caption=r"A 3D view of the background atmosphere. Magnetic field lines are shown in green (log scale) with their seed points plotted as blue dots, density (log scale) in yellow-black and thermal pressure (log scale) in yellow-blue. The full width at half, tenth and 100th of maximum are shown in red contours at the base of domain."
\end{pycode}

\py[chapter3a]|m3d|

\section{Flux Surfaces}\label{sec:fluxsurfaces}

Once the background atmosphere has been constructed and simulations performed, it is necessary to be able to identify and quantify the MHD waves excited by the various drivers.
This section describes the analysis method developed for the research presented in this thesis, which the proceeding chapters will utilize.

As shown in \cref{sec:Vpert}, MHD waves propagating through a plasma cause perturbations in velocity.
As the perturbation velocity is one of the physical variables calculated by the SAC code, this fact provides a mechanism by which we can identify the waves in the simulation domain.
However, the challenge is decomposing the velocity vector, calculated by SAC in the reference frame of the simulation domain ($V_x$, $V_y$, $V_z$) into the frame of the magnetic field \textit{i.e.} a velocity parallel to the magnetic field, $V_\parallel$, perpendicular to the magnetic field, $V_\perp$, and a component perpendicular to the magnetic field in the other plane $V_\phi$, in which the waves are mathematically described.

Considering this problem more closely, the $V_\parallel$ component is trivial to calculate, as it is the projection of $\vec{V}$ onto $\vec{B}$.
The $V_\phi$, the azimuthal component, can be defined as $V_\phi = V_\parallel \times V_\perp$.
Therefore, the challenging component to calculate is $V_\perp$, the component perpendicular to the magnetic field.
When only a two dimensional system is considered, the vector perpendicular to the magnetic field is well defined, as it is perpendicular to the parallel vector in the one degree of freedom available to it.
In three dimensions however, there is a whole plane perpendicular to the parallel vector.
It is therefore obvious that some other construct is needed.

The chosen solution to finding the $V_\perp$ vector is to construct a magnetic flux tube.
A flux tube is a mathematical construct in a magnetic field that encloses a constant amount of magnetic flux over its whole length.
Flux tubes, as a consequence of the fact they contain a fixed amount of magnetic flux, trace the field and move with the magnetic field lines.
A flux tube constructed in the numerical domain would have a boundary, a `flux surface' which would allow the computation of a normal vector from this surface.
The magnetic field described in \cref{sec:mhsbackground}, has a Gaussian profile, and is therefore continuous at every point in the domain.
This means that there is no defined `edge' upon which to draw a flux surface, meaning that it is possible to define any number of arbitrary flux surfaces at any point in the domain.



The flux surface would be constructed numerically in the domain, and therefore would be a series of small planes connected together to form the surface.
The equation of a plane is defined by three arbitrary points in a Cartesian geometry, $x_{1,2,3}$, $y_{1,2,3}$, and $y_{1,2,3}$, where the numerical subscript denotes the index of the point; the normal vector $\vec{n}=(a,b,c)$ and a constant $d$:
\begin{equation}
	ax_{1,2,3}+by_{1,2,3}+cz_{1,2,3}+d=0.
    \label{eq:plane}
\end{equation}
\Cref{eq:plane} can therefore be used to calculate the normal vector to the numerically constructed flux surface.
The three points which define the plane and setting $d=0$ leads to a set of simultaneous equations which can be solved for $\vec{n}$.

\subsection{Constructing Flux Surfaces Numerically}
\begin{pycode}[chapter3a]

# This section is all about describing the construction of flux tubes, many imports are needed:
import sys
import os

import yt
import numpy as np
import matplotlib.pyplot as plt
from mayavi import mlab
from tvtk.api import tvtk

#pysac imports
import pysac.yt
import pysac.analysis.tube3D.tvtk_tube_functions as ttf
import pysac.plot.mayavi_plotting_functions as mpf
from pysac.plot.mayavi_seed_streamlines import SeedStreamline
from pysac.plot.divergingcolourmaps import get_mayavi_colourmap
from pysac.analysis.tube3D.process_utils import get_yt_mlab

### Load in and Config ###

# loaded above
#ds = pysac.yt.SACGDFDataset((ch3.data_file('Slog_p240-0_A10_B005_00001.gdf')))
tube_r = 60

#if running this creates a persistant window just get it out of the way!
mlab.options.offscreen = True
fig = mlab.figure(bgcolor=(1, 1, 1))

cg = ds.index.grids[0]

#Slices
cube_slice = np.s_[:,:,:-5]
x_slice = np.s_[:,:,:,:-5]

#Define the size of the domain
xmax, ymax, zmax = np.array(cg['density_bg'].to_ndarray()[cube_slice].shape) - 1
domain = {'xmax':xmax, 'ymax':ymax, 'zmax':zmax}

bfield, vfield = get_yt_mlab(ds, cube_slice, flux=False)

#Create a scalar field of the magntiude of the vector field
bmag = mlab.pipeline.extract_vector_norm(bfield, name="Field line Normals")

vtk_width = r'0.58\columnwidth'

################################################################################
#### Figure 1 ##################################################################
################################################################################

xc = domain['xmax']/2
yc = domain['ymax']/2
ti = 0
n = 100

surf_seeds = []
for theta in np.linspace(0, 2 * np.pi, n, endpoint=False):
    surf_seeds.append([tube_r * np.cos(theta + 0.5 * ti) + xc,
    tube_r * np.sin(theta + 0.5 * ti) + yc, domain['zmax']])

seeds = np.array(surf_seeds)
#Add axes:
axes, outline = mpf.add_axes(np.array(zip(ds.domain_left_edge,ds.domain_right_edge)).flatten()/1e8)

#Add seed points to plot:
seed_points = mlab.points3d(seeds[:,0], seeds[:,1], seeds[:,2],
color=(0.231, 0.298, 0.752), scale_mode='none',
scale_factor=1.5)
circ_seeds = ch3.save_figure('circ-seeds', fig, fext='.png')
circ_seeds.figure_width = vtk_width
circ_seeds.caption=r"Due to the axisymmetric nature of the background conditions, an axisymmetric ring of seed points is chosen. This example uses a ring at a radius of $936$ km from the centre of the domain."
circ_seeds.placement = 'H'

################################################################################
#### Figure 2 ##################################################################
################################################################################

field_lines = SeedStreamline(seed_points = np.array(seeds))
bmag.add_child(field_lines)
field_lines.actor.mapper.scalar_visibility = False
field_lines.actor.property.color = (0,0,0)
field_lines.actor.property.line_width = 1.5

fig_fieldlines = ch3.save_figure('fieldlines', fig, fext='.png')
fig_fieldlines.figure_width = vtk_width
fig_fieldlines.caption=r"The seed points are then used to trace field lines. The field lines naturally form a flux surface by definition."
fig_fieldlines.placement = 'H'

################################################################################
#### Figure 3 ##################################################################
################################################################################

pd_seeds = ttf.make_circle_seeds(100, 60, **domain)
fieldlines, surface = ttf.create_flux_surface(bfield.outputs[0], pd_seeds)
surface.output.lines = None
flux_surface = mlab.pipeline.surface(surface.output)
flux_surface.actor.mapper.scalar_visibility = False
flux_surface.actor.property.color = (0.8,0.8,0.8)
#flux_surface.actor.property.line_width = 0

fig_flux_surface = ch3.save_figure('flux-surface', fig, fext='.png')
fig_flux_surface.figure_width = vtk_width
fig_flux_surface.caption=r"Once the field lines have been traced a surface is constructed from small polygons (triangles) using the vtkRuledSurfaceFilter algorithm."
fig_flux_surface.placement = 'H'

################################################################################
#### Figure 4 ##################################################################
################################################################################

axes.visible = False
outline.visible = False
flux_surface.actor.property.edge_visibility = True
zoomed_surface = ch3.save_figure('zoomed-surface', fig, fext='.png', azimuth = 90, elevation = 75, distance=80, focalpoint=[63, 120, 110], aa=20)
zoomed_surface.figure_width = vtk_width
zoomed_surface.caption=r"This figure shows the outlines of the triangles of which the surface comprises."
zoomed_surface.placement = 'H'

################################################################################
#### Figure 5 ##################################################################
################################################################################

poly_norms = ttf.make_poly_norms(surface.output)
normvec = mlab.pipeline.glyph(poly_norms.output)
normvec.glyph.glyph_source.glyph_source = normvec.glyph.glyph_source.glyph_dict['arrow_source']
normvec.glyph.glyph.scale_mode = 'data_scaling_off'
normvec.glyph.glyph.color_mode = 'color_by_scale'
normvec.glyph.glyph.scale_factor = 5
normvec.glyph.glyph_source.glyph_position = 'tail'

zoomed_glyphs = ch3.save_figure('zoomed-glyphs', fig, fext='.png', azimuth = 85, elevation = 80, distance=50, focalpoint=[63, 120, 110], aa=20)
zoomed_glyphs.figure_width = vtk_width
zoomed_glyphs.caption=r"By using the surface triangles, a vector normal to the surface can be calculated at each vertex. This is done automatically by the vtkRuledSurfaceFilter algorithm."
zoomed_glyphs.placement = 'H'
\end{pycode}

This section describes the numerical construction of the flux surfaces, and the set of normal vectors defined upon them.
This algorithm employs the Visualisation Tool Kit (VTK\footnote{VTK 6.1 (\url{www.vtk.org})}) and the MayaVi package \citep{ramachandran2011} to provide a high-level Python interface to VTK.
A flux surface is defined as the set of field lines which form a boundary of a flux tube.
Due to the fact that the magnetic field constructed in \cref{sec:mhsbackground} is continuous the choice of `boundary' is arbitrary, as any axisymmetric boundary could be chosen.
The process of computing these flux surfaces is illustrated in \cref{fig:circ-seeds,fig:fieldlines,fig:flux-surface,fig:zoomed-surface,fig:zoomed-glyphs}.

The first step, shown in \cref{fig:circ-seeds} as blue dots, is the selection of the seed points for the field lines.
This effectively defines the location of the flux surface.
For the background conditions described in \cref{sec:mhsbackground} the magnetic field is axisymmetric, therefore a circle of seed points is chosen at the top of the domain.
The radius of this circle defines which arbitrary flux surface is constructed.

\py[chapter3a]|circ_seeds|

\py[chapter3a]|fig_fieldlines|

The second step (\cref{fig:fieldlines}) is the tracing of the field lines from the defined seed points through the domain.
This calculation is performed using the \verb|vtkStreamTracer| algorithm, using the Runge Kutta 4 integrator, integrating backwards along the field lines.


The third step is the construction of a surface from the set of field lines traced in step 2.
The result of this is shown in \cref{fig:flux-surface}.
The surface is generated using the \\ \verb|vtkRuledSurfaceFilter|, which, as configured, connects each point in field line $f_i$ with the two adjacent points in field line $f_{i+1}$.
This is then repeated for each field line calculated, and between the last and first field lines.
A zoomed in view of the constructed surface is shown in \cref{fig:zoomed-surface} with the edges of the component polygons shown.
The result of the \verb|vtkRuledSurfaceFilter| is a series of triangles connecting each point in adjoining field lines.


\py[chapter3a]|fig_flux_surface|
\py[chapter3a]|zoomed_surface|

Finally, now that a set of triangular planes have been defined between the field lines, normal vectors can be extracted.
This is done automatically by \verb|vtkRuledSurfaceFilter|, however, the orientation of the vector can vary.
Therefore a check is implemented in the pipeline to ensure the normal vector is always orientated away from the original axis of the background magnetic field.
The resulting normal vectors are shown in \cref{fig:zoomed-glyphs}

The number of normal vectors calculated is dependant on the number of triangles which is determined by the number of seed points and the step size of the field line integrator.
The number of seed points used $100$ for all the analysis in this thesis.
The resolution of the integrator is left at the default as calculated by VTK.

\py[chapter3a]|zoomed_glyphs|

Once the normal vectors have been constructed, it is simple to compute the azimuthal vector from the magnetic field unit vector, and the normal vector $\vec{n_\phi}= \vec{n_\perp} \times \vec{n_\parallel}$ where $\vec{n_\perp},\ \vec{n_\parallel} $ are also both unit vectors. 
Using  $\vec{n_\perp},\ \vec{n_\parallel}$ and $\vec{n_\phi}$ it  is possible to project any vector quantity calculated in the simulations into this reference frame, and use it for the analysis of MHD waves.


