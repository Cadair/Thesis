%*****************************************************************************************
%*********************************** Third Chapter ***************************************
%*****************************************************************************************

\chapter{Driving MHD Waves in the Photosphere}  %Title of the Third Chapter

\begin{pycode}[chapter3]
ch3 = Chapter(pytex, 3, './Chapter3/')
\end{pycode}

\graphicspath{\py|ch3.fig_path|}

{\Large Knowledge at start:}
\begin{itemize}
	\item The ideal MHD equations
	\item Wave solutions for a uniform media
	\item velocity perturbation calculations
	\item Wave flux calcs
	\item SAC, and numerical solutions to the ideal MHD equations
	\item Static background conditions
	\item Construction of flux surfaces
	\item The specific numerical domain used for all the simulations.
\end{itemize}

\section{Driving Waves from the Photosphere}

As discussed in \cref{ch:Background} the plasma conditions in the photosphere are conducive to the generation of MHD waves.
The photosphere is seeded with small scale magnetic features and dynamic plasma motions as a result of the granulation.
In this chapter the motions that are common in the inter-granular lanes will be studied for their potential ability to drive MHD waves in small scale magnetic flux tubes.
Considering the physical conditions of the inter-granular lanes, where parcels of hot plasma rise, expand and then sink down these lanes.
It is easy to imagine the motions present: vertical motion caused by rising and sinking plasma; horizontal motion caused by the expansion and contraction of the plasma as it cools; spiralling motion in which plasma sinks down magnetic field lines, analogous to that of water down a plug hole.

The horizontal and vertical motions are commonly observed using high resolution observations, the spiralling motions were observed in an inter-granular lane by \cite{Bonet2008} and various types of spirals have been observed in the higher layers of the atmosphere by \cite{Wedemeyer-Bohm2009,Wedemeyer-Bohm2012,Wedemeyer2013}.
The different types of spiral observed in the atmosphere and the difficulty of performing high quality observations of plasma motions in the inter-granular lanes, the spiralling motions will be modelled by a variety of drivers.
A circular driver, a `spiral' with no radial (expansion) component; an Archemedian spiral, a spiral which expands by a fixed amount for each rotation; and a logarithmic spiral where the spiral expands by an exponentially increasing amount with every rotation.

To drive waves in the numerical domain described in \cref{ch:Background} we have to `move' the plasma by numerically adding a velocity field to the domain.
This is done by adding the desired velocity field to a 3D layer close to the bottom of the domain, within the modelled photosphere.
This velocity field is attenuated with a Gaussian profile in three dimensions and is located at the centre of the domain aligned with the foot point of the magnetic field.
This velocity field is then multiplied by a sine function to make it periodic. The generic form of the driver is given in \cref{eq:generic_driver}.
\begin{equation}
	V(x,y,z) = F(x,y,z) \ e^{-\left(\frac{z^2}{\Delta z^2} + \frac{x^2}{\Delta x^2} + \frac{y^2}{\Delta y^2}\right)} \sin \left(2\pi \frac{t}{P}\right),
	\label{eq:generic_driver}
\end{equation}
where, $V(x,y,z)$ is the output velocity field, $F(x,y,x)$ is an arbitrary function which defines the form of the driver, $\Delta x$, $\Delta y$, $\Delta z$ are the widths of the Gaussian function in the three spatial dimensions, and $P$ is the period of the driver.

The five driving motions, horizontal, vertical, circular, Archemedian spiral and logarithmic spiral are then defined by the form of $F(x,y,z)$. For the horizontal and vertical drivers $F(x,y,z)$ is a constant in the direction of the motion, $x$ for horizontal and $z$ for vertical, for the spiral drivers the forms of $F(x,y,z)$ are given in \cref{eq:Suni,eq:Sarch,eq:Slog}.

\begin{subequations}
	\begin{align}
		F(x) &= A \frac{y}{\sqrt{x^2 + y^2}}\\
		F(y) &= - A \frac{x}{\sqrt{x^2 + y^2}},
	\end{align}
	\label{eq:Suni}
\end{subequations}
 
\begin{subequations}
	\begin{align}
		F(x) &= A \frac{\cos(\theta + \phi)}{\sqrt{x^2 + y^2}},\\
		F(y) &= - A \frac{\sin(\theta + \phi)}{\sqrt{x^2 + y^2}},\\
			&\text{where}\notag\\
			&\theta = \tan^{-1}\left(\frac{y}{x}\right),\ \phi = \tan^{-1}\left(\frac{1}{B_L}\right).\notag	
	\end{align}
	\label{eq:Slog}
\end{subequations}
and $B_L = 0.05$ and is a dimensionless expansion parameter for the logarithmic spiral.
 
\begin{subequations}
	\begin{align}
		F(x) &= A \frac{B_Ax}{x^2 + y^2} \frac{y}{\sqrt{x^2 + y^2}},\\
		F(y) &= - A \frac{B_Ay}{x^2 + y^2} \frac{x}{\sqrt{x^2 + y^2}},
	\end{align}
	\label{eq:Sarch}
\end{subequations}
$B_A = 0.005$ is similar in nature to $B_L$, \textit{i.e.} a dimensionless expansion parameter.
The amplitude $A$ of all the drivers is set to $10$ ms$^{-1}$ for all the simulations performed in this chapter and the period is fixed at $240$ s.


