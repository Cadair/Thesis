%*****************************************************************************************
%*********************************** Third Chapter ***************************************
%*****************************************************************************************

\chapter{MHD Waves excited by Different Photospheric Driver Profiles}  %Title of the Third Chapter

\begin{pycode}[chapter3]
ch3 = Chapter(pytex, 3, './Chapter3/')
\end{pycode}

\graphicspath{\py|ch3.fig_path|}

{\Large Knowledge at start:}
\begin{itemize}
	\item The ideal MHD equations
	\item Wave solutions for a uniform media
	\item velocity perturbation calculations
	\item Wave flux calcs
	\item SAC, and numerical solutions to the ideal MHD equations
	\item Static background conditions
	\item Construction of flux surfaces
	\item The specific numerical domain used for all the simulations.
\end{itemize}

\section{Driving Waves from the Photosphere}\label{sec:5drivers}

As discussed in \cref{ch:Background} the plasma conditions in the photosphere are conducive to the generation of MHD waves.
The photosphere is seeded with small scale magnetic features and dynamic plasma motions as a result of the granulation.
In this chapter the motions that are common in the inter-granular lanes will be studied for their potential ability to drive MHD waves in small scale magnetic flux tubes.
Considering the physical conditions of the inter-granular lanes, where parcels of hot plasma rise, expand and then sink down these lanes.
It is easy to imagine the motions present: vertical motion caused by rising and sinking plasma; horizontal motion caused by the expansion and contraction of the plasma as it cools; spiralling motion in which plasma sinks down magnetic field lines, analogous to that of water down a plug hole.

The horizontal and vertical motions are commonly observed using high resolution observations, the spiralling motions were observed in an inter-granular lane by \cite{Bonet2008} and various types of spirals have been observed in the higher layers of the atmosphere by \cite{Wedemeyer-Bohm2009,Wedemeyer-Bohm2012,Wedemeyer2013}.
The different types of spiral observed in the atmosphere and the difficulty of performing high quality observations of plasma motions in the inter-granular lanes, the spiralling motions will be modelled by a variety of drivers.
A circular driver, a `spiral' with no radial (expansion) component; an Archemedian spiral, a spiral which expands by a fixed amount for each rotation; and a logarithmic spiral where the spiral expands by an exponentially increasing amount with every rotation.

To drive waves in the numerical domain described in \cref{ch:Background} the plasma has to be `moved' by numerically adding a velocity field to the domain.
This is done by adding the desired velocity field to a 3D layer close to the bottom of the domain, within the modelled photosphere.
This velocity field is attenuated with a Gaussian profile in three dimensions and is located at the centre of the domain aligned with the foot point of the magnetic field.
This velocity field is then multiplied by a sine function to make it periodic. The generic form of the driver is given in \cref{eq:generic_driver}.
\begin{equation}
	V(x,y,z) = F(x,y,z) \ e^{-\left(\frac{z^2}{\Delta z^2} + \frac{x^2}{\Delta x^2} + \frac{y^2}{\Delta y^2}\right)} \sin \left(2\pi \frac{t}{P}\right),
	\label{eq:generic_driver}
\end{equation}
where, $V(x,y,z)$ is the output velocity field, $F(x,y,x)$ is an arbitrary function which defines the form of the driver, $\Delta x$, $\Delta y$, $\Delta z$ are the widths of the Gaussian function in the three spatial dimensions, and $P$ is the period of the driver.
The values used for the width of the Gaussians are fixed through out this thesis and are: $\Delta x = \Delta y = 0.1$ Mm and $\Delta z = 0.05$, the origin of the driver is at $z = 100$ km above the photosphere.

The five driving motions, horizontal, vertical, circular, Archemedian spiral and logarithmic spiral are then defined by the form of $F(x,y,z)$. For the horizontal and vertical drivers $F(x,y,z)$ is a constant in the direction of the motion, $x$ for horizontal and $z$ for vertical, for the spiral drivers the forms of $F(x,y,z)$ are given in \cref{eq:Suni,eq:Sarch,eq:Slog}.

\begin{subequations}
	\begin{align}
		F(x) &= A \frac{y}{\sqrt{x^2 + y^2}}\\
		F(y) &= - A \frac{x}{\sqrt{x^2 + y^2}},
	\end{align}
	\label{eq:Suni}
\end{subequations}
 
\begin{subequations}
	\begin{align}
		F(x) &= A \frac{\cos(\theta + \phi)}{\sqrt{x^2 + y^2}},\\
		F(y) &= - A \frac{\sin(\theta + \phi)}{\sqrt{x^2 + y^2}},\\
			&\text{where}\notag\\
			&\theta = \tan^{-1}\left(\frac{y}{x}\right),\ \phi = \tan^{-1}\left(\frac{1}{B_L}\right).\notag	
	\end{align}
	\label{eq:Slog}
\end{subequations}
and $B_L = 0.05$ and is a dimensionless expansion parameter for the logarithmic spiral.
 
\begin{subequations}
	\begin{align}
		F(x) &= A \frac{B_Ax}{x^2 + y^2} \frac{y}{\sqrt{x^2 + y^2}},\\
		F(y) &= - A \frac{B_Ay}{x^2 + y^2} \frac{x}{\sqrt{x^2 + y^2}},
	\end{align}
	\label{eq:Sarch}
\end{subequations}
$B_A = 0.005$ is similar in nature to $B_L$, \textit{i.e.} a dimensionless expansion parameter.
The amplitude $A$ of all the drivers is set to $10$ ms$^{-1}$ for all the simulations performed in this chapter and the period is fixed at $240$ s.
Visualisations of these velocity fields can be seen in \cref{fig:spiral_driver_cut}.


\begin{pycode}[chapter3]
from streamlines import Streamlines
#Use Equation 1 to calculate the vector field in a 2D plane to plot it.
time = np.linspace(0,60,480)
dt = time[1:] - time [:-1]
period = 240.

x = np.linspace(7812.5,1992187.5,128)
y = np.linspace(7812.5,1992187.5,128)

x_max = x.max()
y_max = y.max()

xc = 1.0e6
yc = 1.0e6

xn = x - xc
yn = y - yc

delta_x=0.1e6
delta_y=0.1e6

xx, yy = np.meshgrid(xn,yn)
exp_y = np.exp(-(yn**2.0/delta_y**2.0))
exp_x = np.exp(-(xn**2.0/delta_x**2.0))

exp_x2, exp_y2= np.meshgrid(exp_x,exp_y)
exp_xyz = exp_x2 * exp_y2


#==============================================================================
# Define Driver Equations and Parameters
#==============================================================================
#A is the amplitude, B is the spiral expansion factor
A = 10

#Tdamp defines the damping of the driver with time, Tdep is the ocillator
tdamp = lambda time1: 1.0 #*np.exp(-(time1/(period)))
tdep = lambda time1: np.sin((time1*2.0*np.pi)/period) * tdamp(time1)

#Define a peak index to use for scaling in the inital frame
max_ind = np.argmax(tdep(time) > 0.9998)

def log():
	B = 0.05
	phi = np.arctan2(1,B)
	theta = np.arctan2(yy,xx)
	
	uy = np.sin(theta + phi)
	ux =  np.cos(theta + phi)
	
	vx = lambda time1: (ux / np.sqrt(ux**2 + uy**2)) * exp_xyz * tdep(time1) * A
	vy = lambda time1: (uy / np.sqrt(ux**2 + uy**2)) * exp_xyz * tdep(time1) * A
	
	vv = np.sqrt(vx(time[max_ind])**2 + vy(time[max_ind])**2)
	
	return vx, vy, vv

def arch():
	B = 0.005
	r = np.sqrt(xx**2 + yy**2)
	
	vx = lambda time1: ( (B*1e6 * xx) / (xx**2 + yy**2) + yy/r ) * exp_xyz * tdep(time1) * A
	vy = lambda time1: ( (B*1e6 * yy) / (xx**2 + yy**2) - xx/r ) * exp_xyz * tdep(time1) * A
	
	vv = np.sqrt(vx(time[max_ind])**2 + vy(time[max_ind])**2)
	
	return vx, vy, vv

def uniform():
    #Uniform
    vx = lambda time1: A * (yy / np.sqrt(xx**2 + yy**2)) * exp_xyz * tdep(time1)
    vy = lambda time1: A * (-xx / np.sqrt(xx**2 + yy**2)) * exp_xyz * tdep(time1)
    vv = np.sqrt(vx(time[max_ind])**2 + vy(time[max_ind])**2)
    
    return vx, vy, vv

drivers = [log, arch, uniform]

blfigs = []
for driver_func in drivers:
    fig, ax = plt.subplots(figsize=(5,4))
    #============================================================================
    # Do the Plotting
    #============================================================================
    vx, vy, vv = driver_func()
    # Calculate Streamline
    slines = Streamlines(x,y,vx(time[max_ind]),vy(time[max_ind]),maxLen=7000,
                         x0=xc, y0=yc, direction='forwards')

    im = ax.imshow(vv, cmap='Blues', extent=[7812.5,x_max,7812.5,y_max])
    im.set_norm(matplotlib.colors.Normalize(vmin=0,vmax=A))
    #ax.hold()
    
    if driver_func != uniform:
        Sline, = ax.plot(slines.streamlines[0][0],slines.streamlines[0][1],color='red',linewidth=2, zorder=40)
    else:
	    Sline = matplotlib.patches.Circle([1e6, 1e6], radius=.15e6, fill=False, color='red', linewidth=2, zorder=40)
	    ax.add_artist(Sline)

    #Add colourbar
    divider = make_axes_locatable(ax)
    cax = divider.append_axes("right", size="5%", pad=0.2)
    cbar = plt.colorbar(im,cax)
    cbar.set_label(r"$|V|$ [ms$^{-1}$]")
    scalar = matplotlib.ticker.ScalarFormatter(useMathText=False,useOffset=False)
    scalar.set_powerlimits((-3,3))
    cbar.formatter = scalar
    cbar.ax.yaxis.get_offset_text().set_visible(True)
    cbar.update_ticks()
    #cbar.solids.set_rasterized(True)
    cbar.solids.set_edgecolor("face")

    #Add quiver plot overlay
    qu = ax.quiver(x,y,vx(time[max_ind]),vy(time[max_ind]),scale=25*A,color='k',zorder=20, linewidth=1)
    ax.axis([8.0e5,12.0e5,8.0e5,12.0e5])

    ax.xaxis.set_major_formatter(scalar)
    ax.yaxis.set_major_formatter(scalar)
    ax.xaxis.set_major_locator(matplotlib.ticker.MaxNLocator(5))
    ax.yaxis.set_major_locator(matplotlib.ticker.MaxNLocator(5))
    ax.xaxis.get_offset_text().set_visible(False)
    ax.yaxis.get_offset_text().set_visible(False)
    ax.set_xlabel("X [Mm]")
    ax.set_ylabel("Y [Mm]")

    plt.tight_layout()
    ch3.save_figure(driver_func.__name__, fig)

\end{pycode}

\begin{figure}[h]
	\py[chapter3]|ch3.build_subfigure('arch', width=r'0.49\columnwidth', caption="
	Archemdian spiral velocity field with expansion factor $B_A=0.005$")|
	\py[chapter3]|ch3.build_subfigure('log', width=r'0.49\columnwidth', caption="
	Logarithmic spiral velocity field with expansion factor $B_L=0.05$")|

	\centering
	\py[chapter3]|ch3.build_subfigure('uniform', width=r'0.49\columnwidth', caption="
	Uniform spiral velocity field")|
	
	\caption{Horizontal cuts through the spiral driver at the peak amplitude height $z = 0.01$ Mm for the two spiral drivers. Red lines are a streamline of the velocity vector field, sampled as black arrows, overplotted on the velocity magnitude $|V|$.}
	\label{fig:spiral_driver_cut}
\end{figure}

\section{Simulations}

Five different simulations, one for each driver profile shown in \cref{fig:spiral_driver_cut}, were performed using the SAC code as described in \cref{sec:SAC}.
These simulations were performed using the background conditions described in \cref{sec:mhsbackground} on a $128^3$ grid with physical dimensions of $2.0 \times\ 2.0\ \times\ 1.6$ Mm$^3$ in the $x$, $y$ and $z$ directions respectively, and with an origin in the $z$ direction of $0.061$ Mm above the photosphere.
The plasma was driven using the different drivers described in \cref{sec:5drivers} continuously for the length of the simulations. %TODO: What length are we showing here?

\section{Analysing Waves Using Flux Surfaces}

As described in \cref{sec:fluxsurfaces} flux surfaces can be used to analyse the relative properties of the major MHD modes in a system.
In this section we will apply this analysis method to the simulations of various drivers.

