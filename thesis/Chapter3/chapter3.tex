%*****************************************************************************************
%*********************************** Third Chapter ***************************************
%*****************************************************************************************
\label{ch:Methodology}
\chapter{Methodology}  %Title of the Second Chapter

\begin{pycode}[chapter3]
ch3 = texfigure.Manager(pytex, number=3, base_path='./Chapter3/')
\end{pycode}

%
%{\Large Knowledge at end:}
%\begin{itemize}
%	\item Static background conditions
%	\item Construction of flux surfaces
%	\item our domain setup, size, etc.
%\end{itemize}

%\section*{Outline}
%
%\begin{enumerate}
%
%	\item MHD
%	\begin{enumerate}
%		\item Ideal MHD
%		\item MHD Waves
%		
%	\end{enumerate}
%	\item Computational Methods
%	\begin{enumerate}
%		\item Numerical Solutions to Partial Differential Equations
%		\item Numerical MHD
%		\item The Sheffield Advanced Code (SAC)
%	\end{enumerate}
%	\item Magneto-static Background Conditions
%	
%\end{enumerate}
	
\section{Magnetohydrostatic Background Conditions}\label{sec:mhsbackground}

The numerical simulations performed for the analysis in this thesis, all make use of the ability of SAC (\cref{sec:SAC}) to solve for perturbations on top of a static MHD equilibria.
To perform these experiments, it is needed that a background atmosphere be constructed, for all the work in this thesis, quiet Sun regions were chosen for study.
This choice reduces the complexity of the modelled magnetic field and should allow for more accurate separation of the MHD wave effect from any other processes.
Also, the quiet Sun regions are the largest by total area during most of the solar cycle.

%HS Background
The first step in constructing a static model for the solar atmosphere is to understand the hydrodynamic properties of the background plasma.
Models of properties such as the density, and pressure have been created for various heights in the solar atmosphere, based on observations and theory.
The profiles used in this thesis are from the \cite{vernazza1981} model, which covers the range of height in the solar atmosphere from the photosphere ($0$ km) to $2.5$ Mm above the photosphere.
A plot of the density, pressure and temperature from this model is shown in \cref{fig:val3c}.

\begin{pycode}[chapter3]
import pysac.mhs_atmosphere as atm

#Read in the VAL3C model
empirical_data = atm.hs_atmosphere.read_VAL3c_MTW(MTW_file=False)

# Create a Z array at the interpolated resolution and interpolate.
ZZ = u.Quantity(np.linspace(empirical_data['Z'][0], empirical_data['Z'][-1], 128), unit=empirical_data['Z'].unit)
table = atm.hs_atmosphere.interpolate_atmosphere(empirical_data, ZZ)

# Create a figure and make space for the axes.
fig, ax = plt.subplots(gridspec_kw={'right':0.775, 'left':0.13})

# Shortcut all the Mm conversion.
Z = empirical_data['Z'].to(u.Mm)

lrho, = ax.plot(Z, empirical_data['rho'].quantity.si, 'x', color='blue')
lrho_i, = ax.plot(ZZ.to(u.Mm), table['rho'].quantity.si, color='blue')

ax2 = ax.twinx()
lp, = ax2.plot(Z, empirical_data['p'].to(u.Pa), 'x', color='green')
lp_i, = ax2.plot(ZZ.to(u.Mm), table['p'].to(u.Pa), color='green')


ax3 = ax.twinx()
ax3.spines["right"].set_position(("axes", 1.18))
lt, = ax3.plot(Z, empirical_data['T'], 'x', color='red')
lt_i, = ax3.plot(ZZ.to(u.Mm), table['T'], color='red')


# Set primary axes properties and labels
ax.semilogy()
ax.set_ylabel(r"Density [{}]".format(lrho._yorig.unit._repr_latex_()))
ax.set_xlabel(r"Height [{}]".format(lrho._xorig.unit._repr_latex_()))
ax.set_xlim(Z[0].value-0.1, Z[-1].value+0.1)


# Pressure Axis
ax2.semilogy()
ax2.set_ylabel(r"Pressure [{}]".format(lp._yorig.unit._repr_latex_()))


# Temp axis
ax3.set_ylabel(r"Temperature [{}]".format(lt._yorig.unit._repr_latex_()))

# Set the colours for the ticks and the labels.
ax.tick_params(axis='y', colors=lrho.get_color())
ax2.tick_params(axis='y', colors=lp.get_color())
ax3.tick_params(axis='y', colors=lt.get_color())

ax.yaxis.label.set_color(lrho.get_color())
ax2.yaxis.label.set_color(lp.get_color())
ax3.yaxis.label.set_color(lt.get_color())

val3c = ch3.save_figure('val3c', fig, fext='.pgf')
val3c.caption = r"Graph of the \cite{{vernazza1981}} model C showing the density ({}), pressure ({}), and temperature ({}). The solid lines show the interpolated values used to construct the numerical model, at the resolution of the numerical domain.".format(lrho.get_color(), lp.get_color(), lt.get_color())
\end{pycode}

\py[chapter3]|val3c|

On top of this hydrostatic background, a magnetic field is required to study MHD waves.
%MBPs and magnetic structures
The type of magnetic phenomena to be investigated are small scale photospheric structures, that occur frequently over the disk of the Sun.
Magnetic Bright Points (MBPs) were chosen as an observational feature to use as a guide for the background model, MBPs are well studied and there are good estimates of their physical properties.
\cite{feng2013} performed a study of Photospheric Bright Points (PBPs), which are assumed to be analogous to MBPs, using the Dutch Open Telescope, and find the peak of the log-normal distribution give a diameter of $232\pm40$ km for the quiet Sun.
They also show that $50$\% of the PBPs they analysed had the major axis of their ellipse less than $1.5$ times longer than the minor axis.
Showing that they are to a first approximation circular, however, with some PBPs having quite a large deviation from this approximation.
\cite{utz2013} perform a similar study looking at the average magnetic field strength of MBPs using the Hinode SOT instrument, they find that in a quiet Sun region the average magnetic field strength is $135.0 \pm 1.0$ mT.
\cite{sanchezalmeida2004} also perform a statistical study of MBPs, they find that in their sample most MBPs have lifetimes of less than 10 minutes.
They also note, and present some evidence, that this is probably an upper bound estimate, and true lifetimes may be shorter.


%MHS background construction
\begin{pycode}[chapter3]
from pysac.mhs_atmosphere.parameters.model_pars import mfe_setup as model_pars
import pysac.mhs_atmosphere as atm

# Cheeky Reset to Photosphere
model_pars['xyz'][4] = 0*u.Mm
#==============================================================================
# Build the MFE flux tube model using pysac
#==============================================================================
# model setup
scales, physical_constants = atm.units_const.get_parameters()
option_pars = atm.options.set_options(model_pars, False, l_gdf=True)
coords = atm.model_pars.get_coords(model_pars['Nxyz'], u.Quantity(model_pars['xyz']))

#interpolate the hs 1D profiles from empirical data source[s]
empirical_data = atm.hs_atmosphere.read_VAL3c_MTW(mu=physical_constants['mu'])
table = atm.hs_atmosphere.interpolate_atmosphere(empirical_data, coords['Zext'])

# calculate 1d hydrostatic balance from empirical density profile
# the hs pressure balance is enhanced by pressure equivalent to the
# residual mean coronal magnetic pressure contribution once the magnetic
# field has been applied
magp_meanz = np.ones(len(coords['Z'])) * u.one
magp_meanz *= model_pars['pBplus']**2/(2*physical_constants['mu0'])

# Make the vertical profile 3D
pressure_z, rho_z, Rgas_z = atm.hs_atmosphere.vertical_profile(coords['Z'], table, magp_meanz, physical_constants, coords['dz'])

# Generate 3D coordinate arrays
x, y, z = u.Quantity(np.mgrid[coords['xmin']:coords['xmax']:1j*model_pars['Nxyz'][0],
                              coords['ymin']:coords['ymax']:1j*model_pars['Nxyz'][1],
                              coords['zmin']:coords['zmax']:1j*model_pars['Nxyz'][2]], unit=coords['xmin'].unit)

# Get default MFE flux tube parameters out of pysac
xi, yi, Si = atm.flux_tubes.get_flux_tubes(model_pars, coords, option_pars)

# Generate the 3D magnetic field
allmag = atm.flux_tubes.construct_magnetic_field(x, y, z, xi[0], yi[0], Si[0], model_pars, option_pars, physical_constants, scales)
pressure_m, rho_m, Bx, By ,Bz, Btensx, Btensy = allmag

# local proc 3D mhs arrays
pressure, rho = atm.mhs_3D.mhs_3D_profile(z, pressure_z, rho_z, pressure_m, rho_m)
magp = (Bx**2 + By**2 + Bz**2) / (2.*physical_constants['mu0'])
energy = atm.mhs_3D.get_internal_energy(pressure, magp, physical_constants)
\end{pycode}

\begin{pycode}[chapter3]
#### YT STUFF ####

import yt

# Add derived Fields
def magnetic_field_strength(field, data):
    return np.sqrt(data["magnetic_field_x"]**2 + data["magnetic_field_y"]**2 + data["magnetic_field_z"]**2)
yt.add_field(("gas","magnetic_field_strength"), function=magnetic_field_strength, units = "T")

def alfven_speed(field, data):
    return np.sqrt(2.*data['magnetic_pressure']/data['density'])
yt.add_field(("gas","alfven_speed"), function=alfven_speed, units = "m/s")

bbox = u.Quantity([u.Quantity([coords['xmin'], coords['xmax']]),
                   u.Quantity([coords['ymin'], coords['ymax']]),
                   u.Quantity([coords['zmin'], coords['zmax']])]).to(u.m).value

# Now build a yt DataSet with the generated data:
data = {'magnetic_field_x':yt.YTQuantity.from_astropy(Bx.decompose()),
        'magnetic_field_y':yt.YTQuantity.from_astropy(By.decompose()),
        'magnetic_field_z':yt.YTQuantity.from_astropy(Bz.decompose()),
        'pressure': yt.YTQuantity.from_astropy(pressure.decompose()),
        'magnetic_pressure': yt.YTQuantity.from_astropy(magp.decompose()),
        'density': yt.YTQuantity.from_astropy(rho.decompose())}

ds = yt.load_uniform_grid(data, x.shape, length_unit='m', magnetic_unit='T', mass_unit='kg', periodicity=[False]*3, bbox=bbox)
\end{pycode}


\begin{pycode}[chapter3]
from astropy.modeling import models, fitting

x_lin = np.linspace(coords['xmin'], coords['xmax'], model_pars['Nxyz'][0])
bmag = np.sqrt((Bx**2 + By**2 + Bz**2))[:,64,0].to(u.mT)

gaussian = models.Gaussian1D()
bmag_fit = fitting.LevMarLSQFitter()(gaussian, x_lin, bmag)

fwhm = 2.*np.sqrt(2*np.log(2))*bmag_fit.stddev.value
fwhm = np.abs(fwhm) * u.Mm

fwtm = 2.*np.sqrt(2*np.log(10))*bmag_fit.stddev.value
fwtm = np.abs(fwtm) * u.Mm

fwhm = fwhm.to(u.km)
fwtm = fwtm.to(u.km)
\end{pycode}
From this information, and taking into account the properties of the hydrostatic background chosen, a magnetic field can be constructed with properties similar to the MBPs.
The 3D magnetic field was generated using work of \cite{gent2013, gent2014}, and the implementation of that paper in the pysac library\footnote{https://github.com/SWAT-Sheffield/pysac}.

%TODO: talk parameters here.
* footpoint bfield
* 

The result of the flux tube construction in \cite{gent2013, gent2014} is a 2D Gaussian profile for the cross-sectional magnetic field, with a full width at half maximum of $\approx$ \py[chapter3]|np.round(fwhm, -1)|, and a full width at a 10th of maximum (\py[chapter3]|np.round(fwtm, -1)|) at the photosphere, corresponding to the lower bound of the observations in \cite{utz2013}.
Plots of the magnetic field strength in the photosphere are shown in \cref{fig:gaussian}, for both a 1D and 2D profile.
A vertical slice through the background atmosphere is then shown in \cref{fig:alf2d}, with the structure of the magnetic field shown over-plotted on the Alfv\'en speed.
As can be seen in \cref{fig:alf2d} the magnetic field expands non-linearly as the kinetic pressure decreases with height.
It should also be noted that the background atmosphere has been constructed up to $1.6$ Mm above the photosphere defined in \cite{vernazza1981}, this is a deliberate choice to exclude the region where the plasma $\beta = 1$.
The reasons for this choice lie in the studies to be performed using this atmosphere, this thesis is explicitly studying the generation of MHD waves in the photosphere and not their propagation into the higher layers of the solar atmosphere.
It is therefore easier to exclude the very interesting, but also complex physics that occurs around the $\beta = 1$ region.

The numerical domain chosen for the work presented in this thesis is $2.0 \times 2.0 \times 1.6$ Mm in size, with a resolution of $128^3$ grid points, giving a physical size of $15.6\ \times\ 15.6\ \times\ 12.5$ km$^3$ for each grid cell.


\begin{pycode}[chapter3]

gaussian = texfigure.MultiFigure(1, 2, "guassian")
# Make a 1D plot with Gaussian fit.

fig, ax = plt.subplots(figsize=texfigure.figsize(pytex, scale=0.49), gridspec_kw={'left':0.15})
lb, = plt.plot(x_lin, bmag.to(u.mT), 'x')
X = np.linspace(coords['xmin'], coords['xmax'], 1000)
lb_f, = plt.plot(X, bmag_fit(X.value), color='green')

plt.xlabel("X [{}]".format(lb._xorig.unit._repr_latex_()))
plt.ylabel("Magnetic Field Strength [{}]".format(lb._yorig.unit._repr_latex_()))

lhm = plt.axhline(y=bmag_fit.amplitude.value/2, color='cyan', linestyle=':')
ltm = plt.axhline(y=bmag_fit.amplitude.value/10, color='magenta', linestyle=':')

plt.axvline(x=-1*fwhm.to(u.Mm).value/2., color='cyan', linestyle=':')
plt.axvline(x=fwhm.to(u.Mm).value/2., color='cyan', linestyle=':')

plt.xlim([-0.2, 0.2])

bmagf = ch3.save_figure('bmag-footpoint', fig, fext='.pgf')
bmagf.caption = "1D slice through the centre of the domain. The {} line shows a fitted gaussian to the numerical domain. The {} lines show the FWHM and the {} line shows the FWTM.".format(lb_f.get_color(), lhm.get_color(), ltm.get_color())
gaussian.append(bmagf)
\end{pycode}

\begin{pycode}[chapter3]
slc = yt.SlicePlot(ds, 'z', 'magnetic_field_strength', origin='center-domain',
                   axes_unit='km', center=("max", "magnetic_field_strength"), width=(0.5, "Mm"))
slc.set_figure_size(texfigure.figsize(pytex, scale=0.49)[0])

slc.set_cmap('all', 'Oranges')
bmax = ds.all_data().quantities.extrema("magnetic_field_strength")[1]
slc.annotate_contour('magnetic_field_strength', take_log=False,
                     plot_args={'levels':[bmax/2., bmax/10.],
                                'linewidths':3,
                                'colors':['black', 'green']})

bmag2d = ch3.save_figure('bmag-cut', slc)
bmag2d.caption = "2D slice through domain. The black line shows the FWHM and the green line shows the FWTM of the gaussian profile. The image shows the the magnetic field magntitude."

gaussian.append(bmag2d)
gaussian.caption = "Magnetic field strength plots for the photospheric layer in the numerical domain."

\end{pycode}


\py[chapter3]|gaussian|

\begin{pycode}[chapter3]
slc = yt.SlicePlot(ds, 'x', 'alfven_speed', origin='lower-center-domain', axes_unit='Mm')
slc.set_figure_size(texfigure.figsize(pytex)[0])
slc.set_cmap('all', 'BuPu')

seed_points = np.zeros([11,2]) + 1.52
seed_points[:,0] = np.linspace(-0.99, 0.95, seed_points.shape[0], endpoint=True)
slc.annotate_streamlines('magnetic_field_y', 'magnetic_field_z', field_color='magnetic_field_strength',
plot_args={'start_points':seed_points, 'density':15, 'cmap':'YlGn',
'norm': mpl.colors.LogNorm(*ds.all_data().quantities.extrema("magnetic_field_strength"))})

alf2d = ch3.save_figure('alf2d', slc)
alf2d.caption = r"Vertical slice through the background atmosphere, with the Alfv\'en Speed sown in the background with magnetic field lines overplotted in blue."
\end{pycode}

\py[chapter3]|alf2d|

%TODO: Domain size choices

\section{Flux Surfaces}\label{sec:fluxsurfaces}

As shown in \cref{sec:Vpert}, MHD waves propagating through a plasma cause perturbations in the velocity of the plasma.
As the perturbation velocity is one of the physical variables calculated by the SAC code this fact provides a mechanism by which we can identify the waves in the simulation domain.
However, the challenge is decomposing the velocity vector calculated by SAC in the reference frame of the simulation domain ($Vx$, $Vy$, $Vz$) into the frame of the magnetic field ($V_\parallel$, $V_\perp$, $V_\phi$).
Considering this problem more closely, the $V_\parallel$ component is trivial to calculate, as it is the projection of $\vec{V}$ onto $\vec{B}$.
The $V_\phi$, the azimuthal component, is also simple, it can be defined as $V_\phi = V_\parallel \times V_\perp$.
Clearly therefore, the challenging component to calculate is the $V_\perp$ component, the component perpendicular to the magnetic field.
When only a two dimensional system is considered the perpendicular vector is well defined, it is perpendicular to the parallel vector in the one degree of freedom available to it.
In three dimensions however, there is a whole plane perpendicular to the parallel vector.
It is therefore obvious that some other construct is needed.

The solution to finding a perpendicular vector in three dimensions is to consider the concept of flux tubes.
A flux tube is a mathematical construct of a surface within a magnetic field that encloses a constant amount of magnetic flux over its whole length.
Flux tubes as a consequence of the fact they contain a fixed amount of magnetic flux trace the field and move with the magnetic field lines.
A flux tube constructed in the numerical domain would have a boundary, a `flux surface' which would allow the computation of a normal vector from this surface.
This flux surface would be constructed numerically in the domain, and therefore would be a series of small planes connected together to form the surface.
These planes are mathematically described by the equation of a plane.
The equation of a plane is minimally defined by: three arbitrary points in a Cartesian geometry, $x_{1,2,3}$, $y_{1,2,3}$, and $y_{1,2,3}$, where the numerical subscript denotes the index of the point; the normal vector $\vec{n}=(a,b,c)$ and a constant $d$.
\begin{equation}
	ax_{1,2,3}+by_{1,2,3}+cz_{1,2,3}+d=0.
\end{equation}
We can therefore use this to calculate a vector normal to our surface for each set of three points numerically defining the surface.

\subsection{Constructing Flux Surfaces Numerically}
\begin{pycode}[chapter3a]
ch3 = texfigure.Manager(pytex, number=3, base_path='./Chapter3/')
ch3.fig_count += 100
# This section is all about describing the construction of flux tubes, many imports are needed:
import sys
import os

import yt
import numpy as np
import matplotlib.pyplot as plt
from mayavi import mlab
from tvtk.api import tvtk

#pysac imports
import pysac.yt
import pysac.analysis.tube3D.tvtk_tube_functions as ttf
import pysac.plot.mayavi_plotting_functions as mpf
from pysac.plot.mayavi_seed_streamlines import SeedStreamline
from pysac.plot.divergingcolourmaps import get_mayavi_colourmap
from pysac.analysis.tube3D.process_utils import get_yt_mlab

### Load in and Config ###

ds = pysac.yt.SACGDFDataset((ch3.data_file('Slog_p240-0_A10_B005_00001.gdf')))
tube_r = 60

#if running this creates a persistant window just get it out of the way!
mlab.options.offscreen = True
fig = mlab.figure(bgcolor=(1, 1, 1))

cg = ds.index.grids[0]

#Slices
cube_slice = np.s_[:,:,:-5]
x_slice = np.s_[:,:,:,:-5]

#Define the size of the domain
xmax, ymax, zmax = np.array(cg['density_bg'].to_ndarray()[cube_slice].shape) - 1
domain = {'xmax':xmax, 'ymax':ymax, 'zmax':zmax}

bfield, vfield = get_yt_mlab(ds, cube_slice, flux=False)

#Create a scalar field of the magntiude of the vector field
bmag = mlab.pipeline.extract_vector_norm(bfield, name="Field line Normals")

vtk_width = r'0.59\columnwidth'
\end{pycode}

The construction one of these flux surfaces, and the set of normal vectors defined upon it, is described below.
This process makes use of the Visualisation Tool Kit (VTK\footnote{VTK 6.1 (\url{www.vtk.org})}) and the MayaVi package \cite{Ramachandran2011} to provide a high level Python interface to VTK.
The flux surface is constructed based on the definition of flux tubes and field lines, a flux tube will enclose a fixed number of field lines dependant upon the magnetic flux density.
Therefore the edge of the flux tube can be thought of as a set of field lines.
We can therefore calculate the surface of the flux tube by computing the field lines that form its outer boundary.
The process of computing these flux surfaces is illustrated in \cref{fig:circ_seeds,fig:fieldlines,fig:flux_surface,fig:zoomed_surface,fig:zoomed_glyphs} and described in the respective captions.

\begin{pycode}[chapter3a]
xc = domain['xmax']/2
yc = domain['ymax']/2
ti = 0
n = 100

surf_seeds = []
for theta in np.linspace(0, 2 * np.pi, n, endpoint=False):
    surf_seeds.append([tube_r * np.cos(theta + 0.5 * ti) + xc,
                  tube_r * np.sin(theta + 0.5 * ti) + yc, domain['zmax']])
                  
seeds = np.array(surf_seeds)
#Add axes:
axes, outline = mpf.add_axes(np.array(zip(ds.domain_left_edge,ds.domain_right_edge)).flatten()/1e8)

#Add seed points to plot:
seed_points = mlab.points3d(seeds[:,0], seeds[:,1], seeds[:,2],
                            color=(0.231, 0.298, 0.752), scale_mode='none',
                            scale_factor=1.5)
circ_seeds = ch3.save_figure('circ_seeds', fig, fext='.png')
circ_seeds.figure_width = vtk_width
circ_seeds.caption=r"Due to the axisymmetric nature of the background conditions, an axisymmetric ring of seed points is chosen. This example uses a ring at a radius of $936$ km from the centre of the domain."

\end{pycode}
\py[chapter3a]|circ_seeds|

\begin{pycode}[chapter3a]
field_lines = SeedStreamline(seed_points = np.array(seeds))
bmag.add_child(field_lines)
field_lines.actor.mapper.scalar_visibility = False
field_lines.actor.property.color = (0,0,0)
field_lines.actor.property.line_width = 1.5

fieldlines = ch3.save_figure('fieldlines', fig, fext='.png')
fieldlines.figure_width = vtk_width
fieldlines.caption=r"The seed points are then used to trace field lines. The field lines naturally form a flux surface by definition."
\end{pycode}
\py[chapter3a]|fieldlines|

\begin{pycode}[chapter3a]
pd_seeds = ttf.make_circle_seeds(100, 60, **domain)
fieldlines, surface = ttf.create_flux_surface(bfield.outputs[0], pd_seeds)
surface.output.lines = None
flux_surface = mlab.pipeline.surface(surface.output)
flux_surface.actor.mapper.scalar_visibility = False
flux_surface.actor.property.color = (0.8,0.8,0.8)
#flux_surface.actor.property.line_width = 0

fig_flux_surface = ch3.save_figure('flux_surface', fig, fext='.png')
fig_flux_surface.figure_width = vtk_width
fig_flux_surface.caption=r"Once the field lines have been traced a surface is constructed from small polygons (triangles) using the vtkRuledSurfaceFilter algorithm."
\end{pycode}
\py[chapter3a]|fig_flux_surface|

\begin{pycode}[chapter3a]
axes.visible = False
outline.visible = False
flux_surface.actor.property.edge_visibility = True
zoomed_surface = ch3.save_figure('zoomed_surface', fig, fext='.png', azimuth = 90, elevation = 75, distance=80, focalpoint=[63, 120, 110], aa=20)
zoomed_surface.figure_width = vtk_width
zoomed_surface.caption=r"This figure shows the outlines of the triangles of which the surface comprises."
\end{pycode}
\py[chapter3a]|zoomed_surface|


\begin{pycode}[chapter3a]
poly_norms = ttf.make_poly_norms(surface.output)
normvec = mlab.pipeline.glyph(poly_norms.output)
normvec.glyph.glyph_source.glyph_source = normvec.glyph.glyph_source.glyph_dict['arrow_source']
normvec.glyph.glyph.scale_mode = 'data_scaling_off'
normvec.glyph.glyph.color_mode = 'color_by_scale'
normvec.glyph.glyph.scale_factor = 5
normvec.glyph.glyph_source.glyph_position = 'tail'

zoomed_glyphs = ch3.save_figure('zoomed_glyphs', fig, fext='.png', azimuth = 85, elevation = 80, distance=50, focalpoint=[63, 120, 110], aa=20)
zoomed_glyphs.figure_width = vtk_width
zoomed_glyphs.caption=r"By using the surface triangles a vector normal to the surface can be calculated at each vertex. This is done automatically by the vtkRuledSurfaceFilter algorithm."
\end{pycode}

\py[chapter3a]|zoomed_glyphs|

Once the normal vectors have been constructed, it is simple to compute the azimuthal vector from the magnetic field unit vector and the normal vector $\vec{n_\phi}= \vec{n_\perp} \times \vec{n_\parallel}$ where $\vec{n_\perp},\ \vec{n_\parallel} $ are also both unit vectors. 
Using  $\vec{n_\perp},\ \vec{n_\parallel}$ and $\vec{n_\phi}$ it  is possible to project any vector quantity calculated in the simulations into this reference frame, and use it for the analysis of MHD waves.
