%*****************************************************************************************
%*********************************** Fith Chapter ****************************************
%*****************************************************************************************

\chapter{Effects of Expansion Factor on Logarithmic Spiral MHD Wave Excitation}

\begin{pycode}[chapter5]
ch5 = texfigure.Manager(pytex, number=5, base_path='./Chapter5/')

from streamlines import Streamlines

BL = np.array([0.015, 0.05, 0.15, 0.45, 1.5])

from sacconfig import SACConfig

cfg = SACConfig()
\end{pycode}

This work, as a follow-up to \cite{Mumford2015}, investigates the effect of logarithmic spiral-type velocity drivers in the solar photosphere and their properties as MHD wave generation mechanisms.
\cite{Mumford2015} studied five different photospheric velocity fields as drivers for MHD waves.
It was concluded that the logarithmic, Archemedian and uniform spiral drivers all generate similar ($\pm 10\%$) excited energy fluxes.
The spiral expansion factors were selected arbitrarily in \cite{Mumford2015}.
This work analyses the effects of the spiral expansion factor on the MHD waves generated by these velocity fields, motivated by the observational studies and constraints of \cite{Bonet2008}.
In \cite{Bonet2008} magnetic bright points (MBPs) were observed spiralling in an inter-granular lane, where cold plasma sinks down into the convection zone.
\cite{Bonet2008} fit the observed locations of the MBP with time to the equation for a logarithmic spiral, shown in Equation (\ref{eq:log_spiral}),
\begin{equation}
\theta = \frac{1}{B_L}\ln(r/a),
\label{eq:log_spiral}
\end{equation}
where $r$ is the radius of the spiral and $a$ is a positive real constant, and obtained a value of $B_L^{-1} = 6.4 \pm 1.6 = B_L = 0.15$ for the dimensionless expansion factor parameter.

In \cite{Bonet2010} a larger sample of photospheric vortices were studied, despite not fitting spirals to the observed motions, a number density of photospheric vortices was calculated as $d \simeq 3.1 \times 10^{-3}$ vortices Mm$^{-2}$ minute$^{-1}$, which therefore provides an upper limit of the number of logarithmic spiral-like vortices in the solar photosphere.

In this work we investigate the role of the spiral expansion factor ($B_L$) in the generation of MHD waves in a non-potential Gaussian magnetic flux tube, embedded in a realistic stratified solar atmosphere.
The observational result of \cite{Bonet2008} is used as a starting point and values $\pm 3\times$ and $\pm 10\times$ that value are then employed to give five points in the parameter space, centred around their result, which is illustrated in Figure \ref{fig:B_L_values}.


\section{Simulation Configuration}\label{sec:simconfig}
The simulations performed for this study utilise a realistic stratified solar atmosphere constructed by taking the VALIIIc \citep{vernazza1981} hydrodynamical properties and adding a non-potential self-similar magnetic field.
The self-similar magnetic field configuration is derived from the ones employed by \citet{fedun2011} and recently analytically described in \cite{gent2013, gent2014}; based on \citet{schluter1958, deinzer1965, low1980, Schussler2005}, and identical to the one in \cite{Mumford2015}.
A magnetic field is constructed via this method, then added to the hydrostatic background and then the pressure balance is satisfied using magneto-hydrostatic equilibrium as described by Equation~(\ref{eq:mhs-condition}), \textit{i.e.}
\begin{equation}
-(\mathbf{B_b}\cdot \nabla)\mathbf{B_b} + \nabla\left(\frac{\mathbf{B_b}^2}{2}\right) + \nabla p = \rho\mathbf{g}.
\label{eq:mhs-condition}
\end{equation}
(Equation~(\ref{eq:mhs-condition}) corrects the missing negative term in \cite{Mumford2015}, the calculations are not affected.)
By using a magnetic foot point strength of $120$ mT and the background atmosphere as specified by the VALIIIc model, the resulting numerical domain has the plasma $\beta > 1$ at every point.

The Sheffield Advanced Code (SAC) \citep{Shelyag2008} used in this work is configured identically to \cite{Mumford2015}. The domain has a spatial extent of $2.0 \times\ 2.0\ \times\ 1.6$ Mm$^3$, in $x$, $y$, $z$ respectively, with the origin in the $z$ direction $0.061$ Mm above the photosphere. The domain is divided up into $128^3$ grid points giving a physical size of $15.6\ \times\ 15.6\ \times\ 12.5$ km$^3$ for each grid cell.

The magnetohydrostatic background is perturbed during the simulations using a 3D Gaussian weighted logarithmic spiral velocity driver, as described by Equation~(\ref{eq: slog}) \citep{Mumford2015}:
\begin{subequations}
    \begin{align}
    V_x &= A \frac{\cos(\theta + \phi)}{\sqrt{x^2 + y^2}}\ e^{-\left(\frac{z^2}{\Delta z^2} + \frac{x^2}{\Delta x^2} + \frac{y^2}{\Delta y^2}\right)} \sin \left(2\pi \frac{t}{P}\right),\\
    V_y &= - A \frac{\sin(\theta + \phi)}{\sqrt{x^2 + y^2}}\ e^{-\left(\frac{z^2}{\Delta z^2} + \frac{x^2}{\Delta x^2} + \frac{y^2}{\Delta y^2}\right)} \sin \left(2\pi \frac{t}{P}\right),\label{eq:Slog}\\
    \end{align}
    \label{eq: slog}
\end{subequations}
where:
\begin{equation*}
\theta = tan^{-1}\left(\frac{y}{x}\right),\ \phi = tan^{-1}\left(\frac{1}{B_L}\right),\notag	
\end{equation*}
$A=\frac{20}{\sqrt{3}}$, $\Delta x = \Delta y = 0.1$ Mm and $\Delta z = 0.05$ and $P=\py|int(cfg.period)|$ s.
Here, $B_L$ is the logarithmic spiral expansion factor discussed in Section~\ref{sec:intro}.

Figure \ref{fig:All_log_spirals} shows the calculated velocity profiles for the peak vertical height of the driver.
Overplotted on these profiles are streamlines that trace a logarithmic spiral with different expansion factors.

\begin{pycode}[chapter5]
size = list(texfigure.figsize(pytex))
size[1] = 1.5
fig, ax = plt.subplots(figsize=size)
ax.plot(BL, np.ones(BL.size), 'x', markersize=10, mew=2)
ax.errorbar([0.15], [1], xerr=np.array([[-1*(0.15-1/(6.4-1.6)), 0.15+1/(6.4+1.6)]]).T, mew=2, elinewidth=2)
ax.semilogx()
ax.get_yaxis().set_visible(False)
ax.set_frame_on(False)
ax.get_xaxis().tick_bottom()
ax.xaxis.set_tick_params(width=2)
ax.xaxis.set_tick_params(width=2, which='minor')
ax.xaxis.set_major_formatter(matplotlib.ticker.ScalarFormatter())
ax.xaxis.set_ticks(BL)
xmin, xmax, ymin, ymax = ax.axis()
ax.add_artist(plt.Line2D((xmin, xmax), (ymin, ymin), color='black', linewidth=1.4))
l = ax.set_xlim([0.01, 2.0])
l = ax.set_xlabel(r'$B_L$', fontsize=18)

fig.tight_layout(h_pad=0.01)
#bL_line = save_fig(cfg, fig=fig, fname='bline.pdf')

bL_line = ch5.save_figure('B_L_values', fig, fext='.pgf')
bL_line.caption = r"The parameter space of $B_L$ used in this work, with the $x$-axis on a logarithmic scale. The green error bars show the fit uncertainty of the value observed by \citet{Bonet2008}."
\end{pycode}

\py[chapter5]|bL_line|

\begin{pycode}[chapter5]
#Use Equation 1 to calculate the vector field in a 2D plane to plot it.
time = np.linspace(0,60,480)
dt = time[1:] - time [:-1]
period = 240.

x = np.linspace(7812.5,1992187.5,128)
y = np.linspace(7812.5,1992187.5,128)

x_max = x.max()
y_max = y.max()

xc = 1.0e6
yc = 1.0e6

xn = x - xc
yn = y - yc

delta_x=0.1e6
delta_y=0.1e6

xx, yy = np.meshgrid(xn,yn)
exp_y = np.exp(-(yn**2.0/delta_y**2.0))
exp_x = np.exp(-(xn**2.0/delta_x**2.0))

exp_x2, exp_y2= np.meshgrid(exp_x,exp_y)
exp_xyz = exp_x2 * exp_y2


#==============================================================================
# Define Driver Equations and Parameters
#==============================================================================
#A is the amplitude, B is the spiral expansion factor
A = 1

#Tdamp defines the damping of the driver with time, Tdep is the ocillator
tdamp = lambda time1: 1.0 #*np.exp(-(time1/(period)))
tdep = lambda time1: np.sin((time1*2.0*np.pi)/period) * tdamp(time1)

#Define a peak index to use for scaling in the inital frame
max_ind = np.argmax(tdep(time) > 0.9998)

def get_log(B):
    #Logarithmic
    phi = np.arctan2(1,B)
    theta = np.arctan2(yy,xx)
    
    uy = np.sin(theta + phi)
    ux =  np.cos(theta + phi)
    
    vx = lambda time1: (ux / np.sqrt(ux**2 + uy**2)) * exp_xyz * tdep(time1) * A
    vy = lambda time1: (uy / np.sqrt(ux**2 + uy**2)) * exp_xyz * tdep(time1) * A
    
    vv = np.sqrt(vx(time[max_ind])**2 + vy(time[max_ind])**2)
    
    return vx, vy, vv

blfigs = texfigure.MultiFigure(3, 2, reference="All_log_spirals")
for bl in BL:
    fig, ax = plt.subplots(figsize=texfigure.figsize(pytex, 0.5),
                           gridspec_kw={'bottom':0.2, 'top':0.95})
    #============================================================================
    # Do the Plotting
    #============================================================================
    vx, vy, vv = get_log(bl)
    # Calculate Streamline
    slines = Streamlines(x,y,vx(time[max_ind]),vy(time[max_ind]),maxLen=7000,
    x0=xc, y0=yc, direction='forwards')
    
    im = ax.imshow(vv, cmap='Blues', extent=[7812.5,x_max,7812.5,y_max])
    im.set_norm(matplotlib.colors.Normalize(vmin=0,vmax=1))
    #ax.hold()
    
    Sline, = ax.plot(slines.streamlines[0][0],slines.streamlines[0][1],color='red',linewidth=2, zorder=40)
    
    #Add colourbar
    divider = make_axes_locatable(ax)
    cax = divider.append_axes("right", size="5%", pad=0.2)
    cbar = plt.colorbar(im,cax)
    cbar.set_label(r"$|V|$ [ms$^{-1}$]")
    scalar = matplotlib.ticker.ScalarFormatter(useMathText=False,useOffset=False)
    scalar.set_powerlimits((-3,3))
    cbar.formatter = scalar
    cbar.ax.yaxis.get_offset_text().set_visible(True)
    cbar.update_ticks()
    #cbar.solids.set_rasterized(True)
    cbar.solids.set_edgecolor("face")
    
    #Add quiver plot overlay
    #qu = ax.quiver(x,y,vx(time[max_ind]),vy(time[max_ind]),scale=25*A,color='k',zorder=20, linewidth=1)
    ax.axis([8.0e5,12.0e5,8.0e5,12.0e5])
    
    ax.xaxis.set_major_formatter(scalar)
    ax.yaxis.set_major_formatter(scalar)
    ax.xaxis.set_major_locator(matplotlib.ticker.MaxNLocator(5))
    ax.yaxis.set_major_locator(matplotlib.ticker.MaxNLocator(5))
    ax.xaxis.get_offset_text().set_visible(False)
    ax.yaxis.get_offset_text().set_visible(False)
    ax.set_xlabel("X [Mm]")
    ax.set_ylabel("Y [Mm]")
    
    #plt.tight_layout()
    
    Fig = ch5.save_figure('driver-{}'.format(bl).replace('.', '-'), fig)
    Fig.subfig_width = r'0.495\columnwidth'
    Fig.caption = r'$B_L = {}$'.format(bl)
    
    blfigs.append(Fig)
   
blfigs.caption = r"Cuts in the [$x$-$y$] plane through the driving velocity field. The magnitude of velocity is plotted in blue with velocity vectors overplotted in black and a streamline seeded at the centre plotted in red. A plot is shown for each value of $B_L$ used in a simulation."

\end{pycode}

\py[chapter5]|blfigs|

\section{Analysis}\label{sec:analysis}

To quantify the MHD wave modes generated by the logarithmic spiral velocity drivers it is necessary to quantify the relative proportion of the excited MHD wave modes.
The modes present in the domain are assumed to be uniquely determined by the three wave modes present in a uniform homogeneous plasma, namely, the fast magnetoacoustic mode, the slow magnetoacoustic mode and the Alfv\'en mode.
%TODO: WRONG WRONG WRONG
These three modes are separable into three vector components of perturbation with respect to the magnetic field.
The slow magnetoacoustic mode is the dominant contributor to the vector component perpendicular to the magnetic field vector.
The fast magnetoacoustic mode is the dominant mode in the parallel vector component with respect to the magnetic field.
The Alfv\'en mode can be identified in the third vector component found via the cross product of the parallel and perpendicular vector.
However, plasma geometry and conditions in the simulation domain make this approximation somewhat imperfect, because there are no clear MHD eigenmodes.
Further, these three modes become degenerate in cylindrical geometry giving rise to sausage, kink, and fluting modes.
Also, due to the complex plasma conditions in the simulation domain the modes may become physically coupled meaning that it is impossible to completely separate the modes.
Despite these complications the description of the modes based on the three vector components in the magnetic field frame is taken as a good way to describe, identify and quantify the MHD wave modes in the system.

To identify theses waves via the vector components relative to the magnetic field the identification of a vector perpendicular to the magnetic field vector is required.
In a 2D system this is a trivial step, however, in a 3D simulation it is ill-defined.
The solution to this problem, used in this work, is to define a magnetic flux surface which encapsulates a constant amount of magnetic flux at all heights in the domain.
This surface then allows the computation of a vector perpendicular to it and, thus, to the magnetic field lines it is constructed from.
These `flux surfaces' are initially constructed from a ring of axisymmetric field lines computed in the static background conditions.
The field line seed points then move with the plasma velocity throughout the simulation, which results in the flux surface being constructed from the same field lines at all times in the simulation.
The combination of the surface normal vectors and the magnetic field vector provide the information required to calculate the azimuthal vector via the cross product, which provides a third vector parallel to the surface but perpendicular to the magnetic field.

These surfaces are constructed, using the VTK library\footnote{Visualistation ToolKit 5.10.0 (\url{www.vtk.org})}, for three different characteristic initial radii (measured at the top of the domain) of $156$ km, $468$ km and $936$ km from the centre of the domain, for each simulation, giving a good sampling through the differing plasma properties of the domain.
This allows the analysis of the excited modes at different points in the domain, giving an overall picture of the waves.

Using the flux surfaces, defined above, we can now decompose any vector quantity in the domain into the parallel, perpendicular and azimuthal components, allowing study of the velocity and magnetic field perturbation vectors.
While the velocity and magnetic field perturbation vectors are good for identifying and studying wave behaviour itself, to quantify the amount of each wave mode generated the wave energy flux is computed using Equation (\ref{eq:wave_energy}) from \cite{bogdan2003}.

\begin{equation}
\vec{F}_{wave} \equiv \widetilde{p}_k \vec{v} + \frac{1}{\mu_0} \left(\vec{B}_b \cdot \vec{\widetilde{B}}\right) \vec{v} - \frac{1}{\mu_0}\left(\vec{v} \cdot \vec{\widetilde{B}} \right) \vec{B}_b,
\label{eq:wave_energy}
\end{equation}
where subscript $b$ represents a background variable, tilde represents a perturbation from the background conditions and $p_k$ represents kinetic pressure.

The wave energy flux (Equation (\ref{eq:wave_energy})) is decomposed onto the flux surface in the same way as the velocity vector, subject to the same limitations as the velocity.

\subsection{Results}\label{subsec:results}

To assist in the visualisation and analysis of the results provided by the flux surfaces the vector components along one field line are extracted for all time steps and plotted as time-distance diagrams in Figures \ref{fig:TD_velocity_r30} and \ref{fig:TD_flux_r30}.

Combining the decomposed velocity vector plotted in Figure \ref{fig:TD_velocity_r30} and the decomposed wave flux vector plotted in Figure \ref{fig:TD_flux_r30} we can reliably describe the nature of the waves generated in the simulations.
Overplotted on all panels in Figures~\ref{fig:TD_velocity_r30} \& \ref{fig:TD_flux_r30} are the phase speeds for the background conditions, the dot-dashed line is the fast speed $v_f$, the dashed line is the sound speed $c_s$, the dotted line is the Alfv\'en speed $v_A$ and the solid line is the slow speed $v_s$.
By comparing these characteristic wave mode speeds to the ridges in the time-distance diagrams it can be seen that in the panels for the torsional component (third panel in each figure), the dominant perturbation travels with the Alfv\'en speed (solid line).
We interpret this perturbation as an Alfv\'en wave.
For the perpendicular component (second panels) it can be seen that the dominant perturbation travels with the fast speed (dashed line), therefore this perturbation could be interpreted as a fast (kink or sausage) mode.
We can infer that this perturbation is more likely to be a sausage mode perturbation due to the nature of the driver, in that it should not perturb the axis of the flux tube and, that we observe no significant displacement on the flux surfaces during the simulation.
The most interesting result is shown for the parallel component (top panel in each figure), where for lower values of $B_L$, the amplitudes are low, but the perturbations that are present travel with the fast speed (dotted line).
However, as $B_L$ increases the perturbations change form.
There seems to appear a second, superimposed perturbation travelling with a speed close to that of the slow (or tube) speeds, which could be a slow sausage mode.
This second perturbation seems to grow proportionally to $B_L$, and can be seen to be dominant in Figures \ref{fig:TD_flux_r30_4} and \ref{fig:TD_flux_r30_5}.

The wave flux graphs in Figure \ref{fig:TD_flux_r30} are components normalised to the magnitude of the wave flux vector, thus showing the relative strengths of the components.
Taking Figure \ref{fig:TD_flux_r30_1} for the $B_L=\py|BL[0]|$ spiral it can be seen that most of the excited wave flux is in the azimuthal component, associated with the Alfv\'en wave.
As the expansion factor ($B_L$) increases, the driver becomes more radial, and the flux starts to shift from the azimuthal component into the parallel component.
This is interpreted as a change of the dominant mode from the torsional Alfv\'en wave into a sausage mode with dominant velocity perturbations parallel to the field lines.
Considering the range of $B_L$, found by \cite{bonet2008} and illustrated in the range spanned by Figure \ref{fig:TD_velocity_r30_3} and \ref{fig:TD_velocity_r30_4}, it can be seen that even within this parameter range the parallel component becomes substantially more dominant, meaning the change in spectrum of excited MHD wave modes is sensitive to the expansion factor of a spiral driver.

\cite{Mumford2015} reported that there is a small but significant percentage of the wave energy flux contained in the perpendicular component.
This appears to be inversely coupled to the spiral expansion factor of the driver, as it decreases proportionally with the azimuthal wave flux component.
The size of the perpendicular component is also inversely proportional to the initial radius of the flux surface, as can be seen by its decrease in the three panels of Figure \ref{fig:flux_comparison}.

%\begin{pycode}[velTD]
%pflux_labels = {'par_label':r'$V_\parallel$ ms$^{-1}$', 
%'perp_label':r'$V_\perp$ ms$^{-1}$',
%'phi_label':r'$V_\phi$ ms$^{-1}$'}
%beta = False
%
%def add_triple_phase(ax, tube_r):
%ps = get_phase_speeds(cfg, tube_r)
%for ax0 in ax:
%add_phase_speeds(ax0, color='g', **ps)
%
%captions = []
%fnames = []
%for bl in BL:
%cfg.exp_fac = bl
%
%fig, ax = plt.subplots(nrows=3, ncols=1, figsize=(6,5))
%
%kwargs = get_single_velocity(cfg, 'r30', beta=beta)
%kwargs.update(pflux_labels)
%
%triple_plot(ax, **kwargs)
%
%#Remove the top two x labels
%ax[0].set_xlabel('')
%ax[1].set_xlabel('')
%add_triple_phase(ax, 'r30')
%#add_all_bpert(ax, 'r30')
%fig.tight_layout(h_pad=0.05)
%fnames.append(save_fig(cfg, fig=fig, fname='veltd_{}.pdf'.format(bl)))
%captions.append(r'$B_L = {}$'.format(bl))
%
%\end{pycode}
%
%\newcommand{\fwidth}{0.48\textwidth}
%\begin{figure*}
%    \centering
%    
%    \begin{subfigure}[b]{\fwidth}
%        \py[velTD]|get_pgf_include(fnames[0])|
%        \caption{\py[velTD]|captions[0]|}
%        \label{fig:TD_velocity_r30_1}
%    \end{subfigure}
%    \begin{subfigure}[b]{\fwidth}
%        \py[velTD]|get_pgf_include(fnames[1])|
%        \caption{\py[velTD]|captions[1]|}
%        \label{fig:TD_velocity_r30_2}
%    \end{subfigure}
%    
%    \begin{subfigure}[b]{\fwidth}
%        \py[velTD]|get_pgf_include(fnames[2])|
%        \caption{\py[velTD]|captions[2]|}
%        \label{fig:TD_velocity_r30_3}
%    \end{subfigure}
%    \begin{subfigure}[b]{\fwidth}
%        \py[velTD]|get_pgf_include(fnames[3])|
%        \caption{\py[velTD]|captions[3]|}
%        \label{fig:TD_velocity_r30_4}
%    \end{subfigure}
%    
%    \begin{subfigure}[b]{\fwidth}
%        \py[velTD]|get_pgf_include(fnames[4])|
%        \caption{\py[velTD]|captions[4]|}
%        \label{fig:TD_velocity_r30_5}
%    \end{subfigure}
%    \caption{
%        Velocity time-distance diagrams for all simulated values of $B_L$ for the surface with an initial top radius of $468$ km.
%        Shown in green are the phase speeds for the background conditions, the dot-dashed line is the fast speed $v_f$, the dashed line is the sound speed $c_s$, the dotted line is the Alfv\'en speed $v_A$ and the solid line is the slow speed $v_s$.
%        Note that plasma $\beta > 1$ for all heights in the domain.
%    }
%    \label{fig:TD_velocity_r30}
%\end{figure*}

%\begin{pycode}[fluxTD]
%pflux_labels = {'par_label':r'$F_\parallel / |\vec{F}|$ ms$^{-1}$', 
%'perp_label':r'$F_\perp / |\vec{F}|$ ms$^{-1}$',
%'phi_label':r'$F_\phi / |\vec{F}|$ ms$^{-1}$'}
%beta = False
%
%def add_triple_phase(ax, tube_r):
%ps = get_phase_speeds(cfg, tube_r)
%for ax0 in ax:
%add_phase_speeds(ax0, color='c', **ps)
%
%captions = []
%fnames = []
%for bl in BL:
%cfg.exp_fac = bl
%
%fig, ax = plt.subplots(nrows=3, ncols=1, figsize=(6,5))
%
%kwargs = get_single_percentage_flux(cfg, 'r30', beta=beta)
%kwargs.update(pflux_labels)
%kwargs.update({'cmap': 'PRGn'})
%
%triple_plot(ax, **kwargs)
%#Remove the top two x labels
%ax[0].set_xlabel('')
%ax[1].set_xlabel('')
%add_triple_phase(ax, 'r30')
%#add_all_bpert(ax, 'r30')
%fig.tight_layout(h_pad=0.05)
%fnames.append(save_fig(cfg, fig=fig, fname='fluxtd_{}.pdf'.format(bl)))
%pytex.fignum += 1
%captions.append(r'$B_L = {}$'.format(bl))
%
%\end{pycode}
%
%\begin{figure*}
%    \centering
%    
%    \begin{subfigure}[b]{\fwidth}
%        \py[fluxTD]|get_pgf_include(fnames[0])|
%        \caption{\py[fluxTD]|captions[0]|}
%        \label{fig:TD_flux_r30_1}
%    \end{subfigure}
%    \begin{subfigure}[b]{\fwidth}
%        \py[fluxTD]|get_pgf_include(fnames[1])|
%        \caption{\py[fluxTD]|captions[1]|}
%        \label{fig:TD_flux_r30_2}
%    \end{subfigure}
%    
%    \begin{subfigure}[b]{\fwidth}
%        \py[fluxTD]|get_pgf_include(fnames[2])|
%        \caption{\py[fluxTD]|captions[2]|}
%        \label{fig:TD_flux_r30_3}
%    \end{subfigure}
%    \begin{subfigure}[b]{\fwidth}
%        \py[fluxTD]|get_pgf_include(fnames[3])|
%        \caption{\py[fluxTD]|captions[3]|}
%        \label{fig:TD_flux_r30_4}
%    \end{subfigure}
%    
%    \begin{subfigure}[b]{\fwidth}
%        \py[fluxTD]|get_pgf_include(fnames[4])|
%        \caption{\py[fluxTD]|captions[4]|}
%        \label{fig:TD_flux_r30_5}
%    \end{subfigure}
%    \caption{
%        Normalised wave energy flux time-distance diagrams for all simulated values of $B_L$ for the surface with an initial top radius of $468$ km.
%        Shown in blue are the phase speeds for the background conditions, the dot-dashed line is the fast speed $v_f$, the dashed line is the sound speed $c_s$, the dotted line is the Alfv\'en speed $v_A$ and the solid line is the slow speed $v_s$.
%        Note that plasma $\beta > 1$ for all heights in the domain.
%    }
%    \label{fig:TD_flux_r30}
%\end{figure*}

This change in excitation of MHD waves is summarised in Figure \ref{fig:flux_comparison}, where the average value of $\displaystyle\frac{F_{\parallel, \perp, \phi}^2}{F_\parallel^2 + F_\perp^2 + F_\phi^2}$ for all time is plotted.
Figure \ref{fig:flux_comparison} shows that, between the values of $B_L=0.15$ and $B_L=0.45$ there is a turning point where the torsional component becomes less dominant, with expansion factors larger than $0.15$ having the parallel component being the dominant component.
This turning point occurs within the range of the fitted spirals in \cite{bonet2008} and, therefore, implies that photospheric spirals may indeed generate a variety of different MHD modes with varying strengths.