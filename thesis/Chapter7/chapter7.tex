% !TeX root = ../smumford_thesis.tex
%*****************************************************************************************
%********************************* Seventh Chapter ***************************************
%*****************************************************************************************

\chapter{Conclusions}\label{ch:conclusions}

This thesis has studied the generation of Magnetohydrodynamic waves in the solar photosphere, and their propagation from the photosphere to the base of the chromosphere.
In \cref{ch:drivers} five different photospheric drivers were used to excite MHD waves.
The vertical driver was found to excite primarily fast mode perturbations in the $V_\parallel$ component of velocity.
The horizontal driver primarily excited slow mode perturbations in the perpendicular $V_\perp$ component.
The three torsional drivers, a circular driver as well as Archemedian and logarithmic spirals all excited between $40$ and $60$ \% of their wave flux in the Alfv\'en mode, with the rest distributed between the fast and slow modes.
The uniformly low proportion of excited Alfv\'en wave for all the torsional drivers has an interesting implications for the generation of the widely sort after Alfv\'en wave.
If even idealised circular motions in the photosphere only excite $\approx 45$\% of their wave flux in the Alfv\'en mode, then the estimates of the total amount of available Alfv\'en flux, which could propagate through the chromosphere at potentially heat it and the corona, may be over estimated.

\Cref{ch:expfac} continued the study of the logarithmic spiral driver.
In this chapter, the expansion factor was varied and the effects on the distribution of excited wave modes studied.
In \cref{ch:drivers} the expansion factor was arbitrarily chosen to be $B_L = 0.05$, in \cref{ch:expfac} a variety of expansion factors were simulated based around the observational results of \cite{bonet2008}.
\cite{bonet2008} observed a MBP spiralling in a inter-granular lane, a logarithmic spiral was fitted to the observed locations of the MBP and a expansion factor of $B_L^{-1} = 6.4 \pm 1.6 = B_L = 0.15$ calculated.
Unlike the logarithmic spiral driver used in \cref{ch:drivers}, not all the expansion factors simulated in \cref{ch:expfac}, resulted in the Alfv\'en wave being the dominant mode.
In fact, the mid point of the parameter space studied, $B_L = 0.15$ was the last point simulated where the Alfv\'en mode was dominant, the two points with higher expansion factors that this the fast mode was the dominant mode.
This accentuates the results of \cref{ch:drivers}, in that even less Alfv\'en flux is generated for driver profiles based on observational data.
If a normal distribution of expansion factors were chosen around the observed expansion factor $B_L = 0.15$ a large proportion of these vortexes would be generating more fast mode flux than Alfv\'en flux.

\Cref{ch:period} varied the driving period of the logarithmic spiral driver, while keeping the expansion factor at $B_L = 0.15$, in line with the simulations presented in \cref{ch:expfac}.
The period choices in \cref{ch:drivers,ch:expfac} were $240$ and $180$ s respectively, both of these were selected arbitrarily.
To cover a good range of the period parameter space, $10$ periods were selected varying from $30$ to $300$ seconds, in steps of $30$ s.
The upper limit of $300$ seconds being chosen for a combination of physical and practical purposes.
The maximum lifetime of the MBPs observed by \cite{sanchezalmeida2004} was $10$ minutes, so setting the upper limit as $300$ s allows for two complete periods with this upper bound of MBP lifetime.
In addition to this, running simulations for a much longer time period leads to interference by some reflection from the top numerical boundary.
The effects of the period on the excited wave mode distribution were varied.
The Alfv\'en fluxes varied up to a maximum of $20$\% for the narrowest flux surface, and substantially less than that for the other two wider surfaces.
Interestingly, however, there were some variation in the form of the observed wave fronts in the velocity time-distance diagrams.
At higher periods there is a small shift in the velocity perturbations from the fast mode to the slow mode ($V_\parallel$ to $V_\perp$).
This observation is not really reflected in the average wave flux results, however, there is a small increase in the perpendicular component for the last $3$ or $4$ periods in the sample.
Overall therefore it can be concluded that while the period has some effect on the wave modes generated, especially close to the axis of the flux tube, the effect is much less pronounced than for the other changes made to the driver in this thesis.

