% !TeX root = ../smumford_thesis.tex
%*****************************************************************************************
%********************************** Seventh Chapter ***************************************
%*****************************************************************************************

\chapter{Conclusions and Future Work}\label{ch:conclusions}

\section{Summary and Conclusions}

This thesis has studied the generation of magnetohydrodynamic waves in the solar photosphere, and their propagation from the photosphere to the base of the chromosphere.
In \cref{ch:drivers}, five different photospheric drivers were used to excite MHD waves.
The vertical driver was found to excite primarily fast mode perturbations in the $V_\parallel$ component of the velocity.
The horizontal driver primarily excited slow mode perturbations in the perpendicular ($V_\perp$) component.
The three torsional drivers, a circular driver as well as Archemedian and logarithmic spirals all excited between $40$ and $60$ \% of their wave flux in the Alfv\'en mode, with the rest distributed between the fast and slow modes.
The uniformly low proportion of excited Alfv\'en wave for all the torsional drivers has an interesting implications for the generation of the widely sort after Alfv\'en wave.
If even idealised circular motions in the photosphere only excite $\approx 45$\% of their wave flux in the Alfv\'en mode, then the estimates of the total amount of available Alfv\'en flux, which could propagate through the chromosphere at potentially heat it and the corona, may be overestimated.

\Cref{ch:expfac} continued the study of the logarithmic spiral driver.
In this chapter, the expansion factor was varied and the effects on the distribution of excited wave modes studied.
In \cref{ch:drivers} the expansion factor was arbitrarily chosen to be $B_L = 0.05$, in \cref{ch:expfac} a variety of expansion factors were simulated based around the observational results of \cite{bonet2008}.
This study observed MBPs spiralling in a inter-granular lane, a logarithmic spiral was fitted to the observed locations of the MBP and a expansion factor of $B_L^{-1} = 6.4 \pm 1.6 = B_L = 0.15$ calculated.
Unlike the logarithmic spiral driver used in \cref{ch:drivers}, not all the expansion factors simulated in \cref{ch:expfac} resulted in the Alfv\'en wave being the dominant mode.
In fact, the midpoint of the parameter space studied, $B_L = 0.15$ was the last point simulated where the Alfv\'en mode was dominant, for the two points with higher expansion factors the fast mode was the dominant mode.
This accentuates the results of \cref{ch:drivers}, in that even less Alfv\'en flux is generated for driver profiles based on observational data.
If a distribution of expansion factors are present in the photosphere around the observed expansion factor $B_L = 0.15$ a large proportion of these vortexes would be generating more fast mode flux than Alfv\'en flux.

\Cref{ch:period} varied the driving period of the logarithmic spiral driver, while keeping the expansion factor at $B_L = 0.15$, in line with the simulations presented in \cref{ch:expfac}.
The period choices in \cref{ch:drivers,ch:expfac} were $240$ and $180$ s respectively, both of these were selected arbitrarily.
To cover a good range of the period parameter space, $10$ periods were selected varying from $30$ to $300$ seconds, in steps of $30$ s.
The upper limit of $300$ seconds being chosen for a combination of physical and practical purposes.
The maximum lifetime of the MBPs observed by \cite{sanchezalmeida2004} was $10$ minutes, so setting the upper limit as $300$ s allows for two complete periods with this upper bound of MBP lifetime.
In addition to this, running simulations for a much longer time period leads to interference by some reflection from the top numerical boundary.
The effects of the period on the excited wave mode distribution were varied.
The Alfv\'en fluxes varied up to a maximum of $20$\% for the narrowest flux surface, and substantially less than that for the other two wider surfaces.
Interestingly, however, there was some variation in the form of the observed wave fronts in the velocity time-distance diagrams.
At higher periods there is a small shift in the velocity perturbations from the fast mode to the slow mode ($V_\parallel$ to $V_\perp$).
This observation is not really reflected in the average wave flux results, however, there was a small increase in the perpendicular component for the last $3$ or $4$ periods in the sample.
Overall, it can be concluded that while the period has some effect on the wave modes generated, especially close to the axis of the flux tube, the effect is much less pronounced than for the other changes made to the driver in this thesis.

When considering the conclusion of this thesis, namely that spiral drivers excite a spectra of wave modes, it is worth keeping in mind the limitations of the analysis.
Primarily, the limitation of the spatial extent of the data studied.
Due to the need to construct flux surfaces to successfully decompose the vector quantities into a reference frame that lends itself to the analysis of MHD wave modes, these wave modes can, by definition, only be analysed on the constructed surfaces.
Throughout this thesis, three representative surfaces have been used to understand the dynamics at different points throughout the domain.
However, when considering the results, especially in \cref{ch:period}, where interesting results were observed close to the axis of the magnetic flux tube, it would be advantageous to be able to study the variation of the analysis continuously through the domain. 
In addition to this, the types of torsional oscillatory drivers used in these simulations have not been observed in the solar atmosphere.
The observations used for comparisons are of flows, presumed to be downward flows in inter-granular lanes, it would be an interesting extension of this work to study the effects of these downward spiral flows on wave excitation.

Beyond this fundamental limitation of the analysis method applied, the theoretical description of the plasma used to interpret the results presents some limitations.
Firstly, a uniform field approximation was described in \cref{ch:background}, this is an over-simplification of the numerical domain, where a plasma with some degree of spherical symmetry, in a stratified atmosphere was studied.
While much theory exists surrounding the nature of MHD waves in spherical geometry, this too would prove inadequate to analytically describe the plasma conditions in the simulation, indeed this is the purpose of numerical experiments.
However, some further analysis using this theory could be performed, especially utilising the spatial coordinates of the flux surfaces in the domain, to measure the displacement of the surface from the equilibria.

\section{Future Work}

The research documented in this thesis has explored a variety of parameters of the velocity fields that drive waves in the solar atmosphere.
The logarithmic spiral driver was studied the most intensely with the period and its expansion factor both analysed.
The scope for future work is vast, with substantially more parameters that could be varied.
One set of parameters which has remained constant over all the simulations run in this thesis is the FWHM of the Gaussian driving volume.
This parameter defines in what volume the majority  of the driving energy is added to the simulation.
In this thesis, this volume has been quite small, meaning most of the energy has been added close to the centre of the magnetic flux tube.
This may not be completely physical, because the plasma motions in the photosphere are not limited to the volume of a MBP, or even the inter-granular lane in which they are embedded.
When varying the FWHM of the driving profile it would be important to reconsider the analysis undertaken in \cref{ch:period} to ensure the total kinetic energy remained constant.
This would involve moving the $G(x,y,z)$ term out of the constant of proportionality, and accounting for it when selecting the amplitude.
The results of this study would be interesting, especially when analysing the difference between the flux surfaces closest to the centre of the magnetic flux tube, to the ones further away.

An effect that has not been explored in this thesis, is the effects of the boundary conditions on the results of the simulations.
Specifically, for the linear wave modes studied the magnetic field perturbations and the velocity pertubations should be tightly coupled.
This effect is not accounted for in the boundary conditions implemented by the SAC code, so the boundary conditions may introduce some coupling between the wave modes.
This effect and the coupling between the velocity and magnetic field could be investigated by driving the magnetic field as opposed to the velocity, as in an ideal plasma for linear waves the effects of this should be identical.

More fundamentally, the magnetic configuration itself could be varied.
With the work performed in \cite{gent2013,gent2014} it is trivial and computationally efficient to calculate multiple stable background atmospheres.
The atmosphere used in this work, is limited to below the transition region, because the expansion properties of the magnetic field, as used, would cause the pressure and density to become un-physical above this region.
Therefore, a different more self-consistent atmosphere should be built and the effects of the changing expansion rate on the wave generation studied.
If an atmosphere was constructed which had the same footpoint properties, \textit{e.g.} magnetic field strength and FWHM, the author hypothesises that the effects on the wave flux profiles would be minimal.
However, it is more than possible that the changes in plasma properties in the higher regions of the simulation domain would cause some deviation from the results presented in this thesis.

Finally, an interesting avenue for future study is to construct more complex background atmospheres from multiple flux tubes, such as in \cite{gent2014} and future extensions to that work.
This would allow for construction of an atmosphere based on observation data of the magnetic field, and more interestingly co-aligned observations of the photospheric velocity field.
While these results would more accurately mimic the reality of the solar surface they would present significant challenges in the analysis of the simulations.
The magnetic flux surface algorithm presented in \cref{sec:fluxsurfaces} is capable of selecting any flux surface even in a highly unstructured domain.
This would enable similar analysis to that of this thesis, however care would have to be taken in the interpretation.
