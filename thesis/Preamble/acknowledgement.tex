% ************************** Thesis Acknowledgements *****************************

\begin{acknowledgements}      

Writing this thesis, and performing the research it contains would not have been possible without the help of a large number of people and organisations.
On this page I will attempt to provide credit where credit is due.

The first and most important credit goes to Heather my wife, who's patience I have no doubt tested over the last four years, and especially in the last few months as I have been writing up.
Without her support I would have given up or gone insane a long time ago.
Credit must also go to the rest of my family, who have always given me a lot of love and support.
Next, I have to thank my supervisor Robertus who has always provided guidance when I have required it.
On top of this a special note of thanks goes to Viktor Fedun, who has provided me with a large amount of the technical knowledge I needed to use the SAC code and has always been willing to go above and beyond to lend me a hand.
Lastly, thanks goes to Fred, who brought a fresh prospective to things and helped me a lot with writing Python tools to make SAC easier to use.

Worthy of a page each are all of the fantastic software packages which have made my life easier over the last four years.
A very special mention has to go out to the SunPy community \citep{thesunpycommunity2015a}.
While only a small section of this document is dedicated to my work on that project, the people in the SunPy community, especially Russ, Steven, Jack, Albert, David, Andy and Dan deserve a lot of credit for giving me such an excellent distraction from my research.
I only hope that I can continue to contribute to the future of observational solar physics for a long time.

The rest of the scientific Python community can not escape without thanks, yt \citep{turk2011}, Mayavi \citep{ramachandran2011}, SciPy \citep{jones2001}, matplotlib \citep{hunter2007}, IPython \citep{perez2007}, scikit-image \citep{vanderwalt2014} and Astropy \citep{theastropycollaboration2013} have all made my life substantially easier.
The yt community have helped me immensely since I met some of them at SciPy 2013, I just wish I had discovered how very useful yt was about 9 months earlier!!
Mayavi has done an excellent job of preventing me from having to learn C++ to use VTK and once you understand it, it's a phenomenally powerful piece of software.
scikit-image, gets a special mention for letting me sit with them and enjoy two very fun sprint sessions at SciPy 2013 and EuroSciPy 2014.
Astropy, deserves a special mention for being a very supportive community and being so welcoming to myself and the SunPy project. Special thanks goes to Thomas Robitaille who got me involved in organising an excellent conference in Leiden and has always been on the end of an internet messaging service to lend a hand.
Finally, credit must go to the PythonTeX \citep{poore2015} project, which has made writing this document substantially easier, as all the Python code for all the figures and data is contained inside the LaTeX source code.

Lastly, I have to thank the whole of the H23 crew, the original H23c group, Nabil, Chris, Sky and Aditi and the rest of the H23 people, including Freddie, Stevie, Alex, Sam and Rahul who have all had the misfortune of sharing an office with me while my code was broken.
Sam deserves a special mention for coming to join me after undergrad and taking me out on enough bike rides to keep me sane.
Thanks to everyone who has helped me have an excellent four years of hard work, but a lot of fun.

\end{acknowledgements}
