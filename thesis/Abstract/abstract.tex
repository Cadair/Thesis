% ************************** Thesis Abstract *****************************
\begin{abstract}
This thesis investigates the properties of various modelled photospheric motions as generation mechanisms for magnetohydrodynamic (MHD) waves in the low solar atmosphere.
The solar atmosphere is heated to million-degree temperatures, yet there is no fully understood heating mechanism which can provide the $\approx 300$ W/m$^2$ required to keep the quiet corona at its observed temperatures.
MHD waves are one mechanism by which this energy could be provided to the upper solar atmosphere, however, these waves need to be excited.
The excitation of these waves, in or below the photosphere is a complex interaction between the plasma and the magnetic field embedded within it.

This thesis studies a model of a small-scale magnetic flux tube based upon a magnetic bright point (MBP).
These features are very common in the photosphere and have been observed to be affected by the plasma motions.
The modelled flux tube has a foot point magnetic field strength of $120$ mT and a FWHM of $90$ km, and is embedded in a realistic, stratified solar atmosphere based upon the VALIIIc model.

To better understand the excitation of MHD waves in this type of magnetic structures, a selection of velocity profiles are implemented to excite waves.
Initially a study of five different driving profiles was performed.
A uniform torsional driver as well as Archimedean and logarithmic spiral drivers which mimic observed torsional motions in the solar photosphere.
Along with vertical and horizontal drivers to mimic different motions caused by convection in the photosphere.
The results are then analysed using a novel method for extracting the parallel, perpendicular and azimuthal components of the perturbations, which caters to both the linear and non-linear cases.
Employing this method yields the identification of the wave modes excited in the numerical simulations and enables a comparison of excited modes via velocity perturbations and wave energy flux.
The wave energy flux distribution is calculated, to enable the quantification of the relative strengths of excited modes.
The torsional drivers primarily excite Alfv\'en modes ($\approx 60$\% of the total flux) with contributions from the slow mode, and, for the logarithmic spiral driver, small amounts of slow mode.
The horizontal and vertical drivers primarily excite slow and fast modes respectively, with small variations dependent upon flux surface radius.

This analysis is then applied to more in depth studies of the logarithmic spiral driver.
Firstly, five different values for the $B_L$ spiral expansion factor are chosen which control how rapidly the spiral expands.
Larger values of $B_L$ make the driving profile more radial.
The results of this analysis show that the Alfv\'en wave is the dominant wave for lower values of the expansion factor, whereas, for the higher values the parallel component is dominant.
This transition occurs within the range of the observational constraints, demonstrating that under realistic conditions spiral drivers may not excite most of their wave flux in the Alfv\'en mode.

Finally, the logarithmic spiral is further studied, but with a variety of different periods.
Ten periods from $30$ to $300$ seconds are chosen, and the simulations again analysed using the flux surface method employed previously.
The results of this study are minimal variation in the percentage wave flux in each mode, with no more than $20$ \% variation in any mode for any flux surface studied.
Within this small variation, some non-linear changes in the wave flux were observed, especially around the more important small periods.
Due to the short life time of the MBPs it is thought the short period waves would have more effect and therefore this non-linear variation in wave flux could have some impact on the modes present in the solar atmosphere.

\end{abstract}


















