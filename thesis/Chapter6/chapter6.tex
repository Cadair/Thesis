% !TeX root = ../smumford_thesis.tex
%*****************************************************************************************
%*********************************** Sixth Chapter ***************************************
%*****************************************************************************************
\begin{pycode}[chapter6]
from __future__ import print_function
ch6 = texfigure.Manager(pytex, number=6, base_path='./Chapter6/')

from sacconfig import SACConfig
cfg = SACConfig()
cfg.data_dir = ch6.data_dir

from streamlines import Streamlines
import td_plotting_helpers as ph

from period_amps import sim_params, periods
all_periods = sim_params[:10]
periods = periods[:10]
\end{pycode}

\chapter{Effects of Period on MHD Wave Generation from a Logarithmic Spiral Driver}\label{ch:period}

\Cref{ch:drivers,ch:expfac} studied the effects of the driving velocity profile and the logarithmic spiral expansion factor ($B_L$) on the MHD wave excitation.
In both of these previous chapters, arbitrary periods were chosen, in this chapter the effect of this choice of period on the wave excitation by the logarithmic driver is studied.
The solar photosphere is populated with an outstanding variety of different frequency wave modes.
Acoustic (p-mode) waves have a wide frequency spectra, with a peak power at 5 minutes, and a large number of MHD waves at different frequencies have been observed in the low solar atmosphere; \cite{Freij2014,Dorotovic2014} observe oscillations in magnetic pores at periods ranging from 3 minutes to 25 minutes; \cite{morton2011} observe sausage modes with periods ranging from $30$ to $447$ seconds; \cite{fujimura2009} observe oscillations with periods between $3$ and $6$ minutes in pores and between $4$ and $9$ minutes in the inter-granular lanes.
It is therefore interesting to study a range of possible frequencies for our driver, to see what effects this has on the excitation of MHD modes.

\section{Simulation Configuration}\label{sec:periodconfig}
This chapter employs the same magnetohydrostatic background as \cref{ch:drivers,ch:expfac} which is described in \cref{sec:mhsbackground}.
The plasma is also driven by the same logarithmic spiral driver as given in \cref{eq:Slog,eq:slog5}, the expansion factor is selected as the central point of the parameter sweep performed in \cref{ch:expfac}, \py[chapter6]|r"$B_L={}$".format(cfg.exp_fac)|.
A plot of the driver profile is shown in \cref{fig:slog-profile}.

\begin{pycode}[chapter6]

#Use Equation 1 to calculate the vector field in a 2D plane to plot it.
time = np.linspace(0,60,480)
dt = time[1:] - time [:-1]
period = 240.

x = np.linspace(7812.5,1992187.5,128)
y = np.linspace(7812.5,1992187.5,128)

x_max = x.max()
y_max = y.max()

xc = 1.0e6
yc = 1.0e6

xn = x - xc
yn = y - yc

delta_x=0.1e6
delta_y=0.1e6

xx, yy = np.meshgrid(xn,yn)
exp_y = np.exp(-(yn**2.0/delta_y**2.0))
exp_x = np.exp(-(xn**2.0/delta_x**2.0))

exp_x2, exp_y2= np.meshgrid(exp_x,exp_y)
exp_xyz = exp_x2 * exp_y2


#==============================================================================
# Define Driver Equations and Parameters
#==============================================================================
#A is the amplitude, B is the spiral expansion factor
A = 10

#Tdamp defines the damping of the driver with time, Tdep is the ocillator
tdamp = lambda time1: 1.0 #*np.exp(-(time1/(period)))
tdep = lambda time1: np.sin((time1*2.0*np.pi)/period) * tdamp(time1)

#Define a peak index to use for scaling in the inital frame
max_ind = np.argmax(tdep(time) > 0.9998)

#Logarithmic
B = 0.05
phi = np.arctan2(1,B)
theta = np.arctan2(yy,xx)

uy = np.sin(theta + phi)
ux =  np.cos(theta + phi)

vx = lambda time1: (ux / np.sqrt(ux**2 + uy**2)) * exp_xyz * tdep(time1) * A
vy = lambda time1: (uy / np.sqrt(ux**2 + uy**2)) * exp_xyz * tdep(time1) * A

vv = np.sqrt(vx(time[max_ind])**2 + vy(time[max_ind])**2)

# Calculate Streamline
slines = Streamlines(x,y,vx(time[max_ind]),vy(time[max_ind]),maxLen=7000,
x0=xc, y0=yc, direction='forwards')

#============================================================================
# Do the Plotting
#============================================================================

fig = plt.figure(figsize=texfigure.figsize(pytex))
ax = plt.subplot()
im = ax.imshow(vv, cmap='cubehelix', extent=[7812.5,x_max,7812.5,y_max])
im.set_norm(matplotlib.colors.Normalize(vmin=0,vmax=10))

Sline, = ax.plot(slines.streamlines[0][0],slines.streamlines[0][1],color='red',linewidth=2, zorder=40)

#Add colourbar
divider = make_axes_locatable(ax)
cax = divider.append_axes("right", size="5%", pad=0.2)
cbar = plt.colorbar(im,cax)
cbar.set_label(r"$|\vec{V}|$ [ms$^{-1}$]")
scalar = matplotlib.ticker.ScalarFormatter(useMathText=False,useOffset=False)
scalar.set_powerlimits((-3,3))
cbar.formatter = scalar
cbar.ax.yaxis.get_offset_text().set_visible(True)
cbar.update_ticks()
#cbar.solids.set_rasterized(True)
cbar.solids.set_edgecolor("face")

#Add quiver plot overlay
#qu = ax.quiver(x,y,vx(time[max_ind]),vy(time[max_ind]),scale=25*A,color='#00DDFF',zorder=20)
ax.axis([8.0e5,12.0e5,8.0e5,12.0e5])

ax.xaxis.set_major_formatter(scalar)
ax.yaxis.set_major_formatter(scalar)
ax.xaxis.set_major_locator(matplotlib.ticker.MaxNLocator(5))
ax.yaxis.set_major_locator(matplotlib.ticker.MaxNLocator(5))
ax.xaxis.get_offset_text().set_visible(False)
ax.yaxis.get_offset_text().set_visible(False)
ax.set_xlabel("X [Mm]")
ax.set_ylabel("Y [Mm]")

fig.tight_layout()

slog_fig = ch6.save_figure('slog-profile', fig=fig, fext='.pdf')
slog_fig.caption = r"Horizontal velocity profile of the logarithmic spiral driver with expansion factor $B_L = {}$. The magnitude of velocity is shown by the colour map and the cyan arrows follow the vector field. The red line is a velocity streamline seeded in the centre of the domain.".format(cfg.exp_fac)
slog_fig.figure_width=r'\columnwidth'
\end{pycode}

\py[chapter6]|slog_fig|

This chapter aims to vary the period ($P$) of the driver, and measure the effects on the wave excitation, however, varying the period of the driver will vary the total amount of energy added to the domain by the driver.
This would therefore heavily bias the analysis of the results, so it important that the amplitude of the driver is varied along with the period to maintain a constant energy input.
Below, the relationship between the period ($P$) and amplitude ($A$) is derived to maintain a constant amount of kinetic energy over the run time of the simulation ($T$) assuming $T = nP$, where $n$ is an integer.

The kinetic energy for any point in space at any instant in time is given by:
\begin{equation}
    E_k = \frac{1}{2}\ m\ v^2\label{eq:Ek}
\end{equation}
where $m$ is the mass and $v$ is the velocity.
Initially $E_T$ can be computed over an arbitrary volume $V$, which leads to:
\begin{equation}
    m = \rho(x,y,z)\ V.\label{eq:mass}
\end{equation}

The simulations are perturbed by a driver with the following general profile:
\begin{equation}
    v(x,y,z,t) = A\ G(x,y,z) \sin \left( \frac{2\pi t}{P} \right),\label{eq:vprofile}
\end{equation}
where $A$ is the amplitude of the velocity and $G(x,y,z)$ is a normalised spatial distribution.

Substituting \cref{eq:vprofile} into \cref{eq:Ek} for velocity gives:
\begin{align}
    E_{T}(x,y,z) &= \int_T \frac{1}{2}\ \rho(x,y,z)\ V\ A^2\ G^2(x,y,z)\ \sin^2\left(\frac{2\pi t}{P} \right) dt \\
    &= \frac{1}{2}\ \rho(x,y,z)\ V\ A^2\ G^2(x,y,z) \int_T \sin^2\left(\frac{2\pi t}{P} \right) dt \\
    & = \frac{1}{2}\ \rho(x,y,z)\ V\ A^2\ G^2(x,y,z) \left[ \frac{1}{2}T - \frac{P}{8\pi} \sin \left(\frac{4\pi T}{P} \right) \right]
\end{align}
recalling $T = nP$ this simplifies to 
\begin{equation}
    E_{T}(x,y,z) = \frac{nP}{4}\ \rho(x,y,z)\ V\ A^2\ G^2(x,y,z), \label{eq:Et_xyz}
\end{equation}

In the chosen background equilibrium the profile $\rho(x,y,z)$ is given by a numerical calculation from a reference background and modified for the presence of the magnetic flux tube.
This means that \cref{eq:Et_xyz} can only be numerically integrated and therefore, can be written as:
\begin{equation}
    E_T = \frac{nPA^2V}{4}\ \left( \sum_{x,y,z} \rho(x,y,z)\ G^2(x,y,z) \right),\label{eq:ET}
\end{equation}

\Cref{eq:ET} provides a relationship between the amplitude and period of the driver, however, it can be simplified by considering that many of the variables remain constant for each simulation performed in this work.
For all simulations run in this work the driver is the same, meaning $G(x,y,z)$ is constant, the background conditions and therefore $\rho(x,y,z)$ are also constant as is the numerical domain and therefore $V$.
It is therefore possible to let,
\begin{equation}
    Q = \frac{V}{4} \sum_{x,y,z} \rho(x,y,z)\ G^2(x,y,z)
\end{equation}
where $Q$ is a constant.
Substituting this into \cref{eq:ET} the final result is obtained:
\begin{align}
    E_T &= nPA^2\ Q, \\
    A^2 &= \frac{1}{E_T n Q} \frac{1}{P} \\
    A^2 &\propto \frac{1}{P}
\end{align}

Using the arbitrary amplitude selected in \cref{ch:drivers} of $10$ ms$^{-1}$, the desired amplitude for each of the periods selected can be calculated.
The result of this calculation is shown in \cref{tab:period-amp}.
\begin{table}
    \centering
    \begin{tabular}{cc}
        Period [seconds] & Amplitude [ms$^{-1}$] 	\\ \hline
        $30.0$           & $20\sqrt{2}$           	\\[2ex]
        $60.0$           & $20$  		            \\[2ex]
        $90.0$           & $20\sqrt{\frac{2}{3}}$  \\[2ex]
        $120.0$          & $10\sqrt{2}$        	\\[2ex]
        $150.0$          & $4\sqrt{10}$            \\[2ex]
        $180.0$          & $\frac{20}{\sqrt{3}}$   \\[2ex]
        $210.0$          & $20\sqrt{\frac{2}{7}}$  \\[2ex]
        $240.0$          & $10$                 	\\[2ex]
        $270.0$          & $\frac{20}{3}\sqrt{2}$  \\[2ex]
        $300.0$          & $4\sqrt{5}$           	\\[2ex]
        % $330.0$          & $20\sqrt{\frac{2}{11}}$ \\[2ex]
    \end{tabular}
    \caption{Tabulation of the period and amplitude pairs used in this work so that total kinetic energy input is constant.}
    \label{tab:period-amp}
\end{table}

\subsection{Results}\label{subsec:results}

To analyse the MHD wave generation we need to parametrise the relative strength of each component for different periods of driver.
To do this we decompose the velocity and the wave energy flux as defined in \cite{bogdan2003} onto the field line reference frame.
The velocity decomposition allows us to analyse the generated modes and identify what types of modes are in the generated spectra.
The wave energy flux analysis is presented in terms of $\vec{F}^2_j$ percentages, this allows a neat visualisation showing the relative strengths of each component, where the three components sum to $100\%$.


To make visualisation and analysis of the surfaces easier the values of the decomposed parameter is shown for one field line for all time steps in the simulation.
In Figure \ref{fig:TD_vel_r30} the values of velocity are shown in the form of time-distance diagrams for these field line strips, in Figure \ref{fig:TD_fwave_r30} the decomposed square wave flux is shown. Overlaid on both sets of plots are the characteristic phase speeds of a uniform plasma; the Alfv\'en speed $v_A$ and sound speed $c_s$, as well as the fast speed $v_f^2 = \sqrt{c_s^2 + v_A^2}$ and the slow (tube) speed $v_t^{-2} = \sqrt{c_s^{-2} + v_A^{-2}}$.
While these speeds will deviate from the true speeds in the non-uniform simulation domain, we can use them, in combination with knowledge of high-$\beta$ plasma properties to identify the wave modes present in the simulations.


It is clear from the $V_\phi$ (lowest) frames in Figure \ref{fig:TD_vel_r30} that the dominant perturbation is travelling at approximately the Alfv\'en speed, we can therefore reliably deduce that the torsional component of the velocity is, as expected, the Alfv\'en wave.
The $V_\perp$ (second) panels also a wave front propagating at the slow speed. For the high-$\beta$ plasma in the sub-chromosphere region of the solar atmosphere, this is the velocity component that is perturbed by the slow wave in a uniform plasma. In the shorter-period frames ($30$s and $90$s) there is a lower-amplitude front propagating close to the fast or sound speeds.
This is attributed to the coupling of the fast and slow wave modes due to the inhomogeneity of the plasma.
Finally, in the $V_\parallel$ (top) panels, there is not one dominant wave front, however evidence of two wave fronts, one propagating at the slow speed and one at the fast speed can be discerned.
The front propagating with the fast speed can be attributed to the fast mode, as in a uniform high-$\beta$ plasma the fast mode would perturb the parallel component of the velocity vector.
As with the $V_\perp$ component the existence of the slow mode is attributed to the non-uniform nature of the simulation domain.


The identification of the wave modes in the velocity perturbations can inform the analysis of the wave flux time-distance diagrams in Figure \ref{fig:TD_fwave_r30}.
In Figure \ref{fig:TD_fwave_r30} the total square wave flux is calculated as the sum of the square of each component, $ F^2 = F_\parallel^2 + F_\perp ^2  + F_\phi^2$, the square of each component is then normalised by this square total to give a percentage value for each component.
This is then plotted along one field line, like the velocity components.


The percentage wave flux shown in Figure \ref{fig:TD_fwave_r30}, can be combined with the analysis of Figure \ref{fig:TD_vel_r30} to determine the relative strengths of the wave modes.
In comparison to the upper panels of Figure \ref{fig:TD_vel_r30}, where it was difficult to distinguish between the fronts travelling at the fast and slow speeds, in the $F^2_\parallel$ (top) panel of Figure \ref{fig:TD_fwave_r30} it is clear that the component with the most flux is the fast mode.
Figure \ref{fig:TD_fwave_r30} also shows that in the $F^2_\perp$ (middle) panel, the flux is more evenly shared between the two superimposed components, with the ratio apparently changing dependant upon period.
This observation should be considered when drawing conclusions from the relative strength of the $F^2_\parallel$ component.
The $F^2_\phi$ (bottom) panel is again dominated by the Alfv\'en component.

In Figure \ref{fig:period-flux} a summary of the average percentage square wave flux is presented for each of the \py[chapter6]|len(all_periods)| simulations performed.
The average value was taken for one field line for all time throughout the simulation.
The three panels of Figure \ref{fig:period-flux} are for three flux surfaces seeded at different initial radii at the top of the domain, showing results for different parts of the simulation domain.
In all three panels it can be seen that the averages for the perpendicular component (green dashes), remain constant with respect to period.
While the torsional (red crosses) and parallel (blue dots) components fluxes are clearly period dependant.
Recalling the analysis of Figures \ref{fig:TD_vel_r30} and \ref{fig:TD_fwave_r30} from above, we can attribute the perpendicular flux to the slow wave, the parallel flux to the fast wave and the torsional flux to the Alfv\'en wave.
We can therefore conclude that the relative strengths of the fast mode and the Alfv\'en mode are period dependant, with the Alfv\'en mode overall dominating more at larger periods.
While the growth in relative strength of the Alfv\'en mode is reasonably linear for the $156$ km radius flux surface, the larger flux surfaces show some variation in average wave flux. 


% SIX Velocity T-D Graphs:
\begin{pycode}[chapter6]
from period_amps import periods, str_amps
velocity_labels = {'par_label':r'$ \vec{V}_\parallel$ ms$^{-1}$', 
                   'perp_label':r'$ \vec{V}_\perp$ ms$^{-1}$',
                   'phi_label':r'$ \vec{V}_\phi$ ms$^{-1}$'}
beta = False
cfg.exp_fac = 0.15
ph.xxlim = 600

def add_all_bpert(ax, tube_r, N=4, levels=None):
    kwargs = ph.get_triple(cfg, beta=beta, single='bpert')
    x = kwargs['x_{}'.format(tube_r)]
    y = kwargs['y_{}'.format(tube_r)]
    par = kwargs['par_line_{}'.format(tube_r)].T[::-1, :]
    par[np.abs(par)<=1e-12] = 0
    perp = kwargs['perp_line_{}'.format(tube_r)].T[::-1, :]
    perp[np.abs(perp)<=1e-12] = 0
    phi = kwargs['phi_line_{}'.format(tube_r)].T[::-1, :]
    phi[np.abs(phi)<=1e-12] = 0
    ax[0].contour(x, y, par, N, colors='k', linewidths=np.linspace(0.5,1.5,N))
    ax[1].contour(x, y, perp, N, colors='k', linewidths=np.linspace(0.5,1.5,N))
    ax[2].contour(x, y, phi, N, colors='k', linewidths=np.linspace(0.5,1.5,N))	                   

def add_triple_phase(ax, tube_r):
    ps = ph.get_phase_speeds(cfg, tube_r)
    for ax0 in ax:
        ph.add_phase_speeds(ax0, color='g', **ps)

captions = {p: r"Period: ${}$ s Amplitude:".format(p) + a + r" ms$^{{-1}}$" for p, a in zip(periods, str_amps)[:10]}
#print(captions, file=sys.stderr)
figsize = texfigure.figsize(pytex, height_ratio=0.85)
multifig = texfigure.MultiFigure(3, 1, reference='TD_vel_r30')
for i, paf in enumerate(all_periods[::4]):
    [setattr(cfg, f, getattr(paf, f)) for f in paf._fields]

    fig, ax = plt.subplots(nrows=3, ncols=1, figsize=figsize)
    
    kwargs = ph.get_single_velocity(cfg, 'r30', beta=beta)
    kwargs.update(velocity_labels)
    
    ph.triple_plot(ax, **kwargs)
    add_triple_phase(ax, 'r30')
    #Remove the top two x labels
    ax[0].set_xlabel('')
    ax[1].set_xlabel('')
    fig.tight_layout(h_pad=0.05)
    
    Lfig = ch6.save_figure('TD_vel_r30_p{}'.format(str(paf.period).replace('.','-')), fig=fig, fext='.pdf')
    Lfig.caption = "Velocity components for " + captions[paf.period]
    multifig.append(Lfig)

multifig.caption = r"""Velocity time-distance diagrams for six different period and amplitude combinations are plotted, in each pane three components of velocity are plotted for a flux surface of $r=468$ km. Overlaid on the velocity are magnetic field perturbation contours which are thicker for larger values and dashed for negative values. Shown in green are the phase speeds for the background conditions, the dot-dashed line is the fast speed $v_f$, the dashed line is the sound speed $c_s$, the dotted line is the Alfv\'en speed $v_a$ and the solid line is the slow speed $v_s$."""

\end{pycode}

\py[chapter6]|multifig|


\begin{pycode}[chapter6]
flux_labels = {'par_label':r'$ \vec{F}_\parallel^2$ \%', 
               'perp_label':r'$ \vec{F}_\perp^2$ \%',
               'phi_label':r'$ \vec{F}_\phi^2$ \%'}

multifig = texfigure.MultiFigure(3, 1, reference='TD_fwave_r30')
for i, paf in enumerate(all_periods[::4]):
    [setattr(cfg, f, getattr(paf, f)) for f in paf._fields]
    
    fig, ax = plt.subplots(nrows=3, ncols=1, figsize=figsize)
    
    kwargs = ph.get_single_percentage_flux(cfg, 'r30', beta=beta)
    kwargs.update(flux_labels)
    kwargs.update({'cmap':'PiYG'})
    ph.triple_plot(ax, **kwargs)
    add_triple_phase(ax, 'r30')
    #Remove the top two x labels
    ax[0].set_xlabel('')
    ax[1].set_xlabel('')
    fig.tight_layout(h_pad=0.05)
    
    Lfig = ch6.save_figure('TD_wave_r30_p{}'.format(str(paf.period).replace('.','-')), fig=fig, fext='.pdf')
    Lfig.caption = "Percentage $F^2$ components for " + captions[paf.period]
    multifig.append(Lfig)

multifig.caption = r"""Percentage square wave flux along one field line is plotted over the length of the simulation, for different period and amplitude combinations. Shown in green are the phase speeds for the background conditions, the dot-dashed line is the fast speed $v_f$, the dashed line is the sound speed $c_s$, the dotted line is the Alfv\'en speed $v_a$ and the solid line is the slow speed $v_s$."""

\end{pycode}

\py[chapter6]|multifig|


\begin{pycode}[chapter6]

from period_amps import periods, sim_params
sim_params = sim_params[:10]
periods = periods[:10]

size = texfigure.figsize(pytex, height_ratio=1.2)
fig, axs = plt.subplots(nrows=3, figsize=size, sharex=True)
titles = ["Flux Surface Radius $=156$ km", "Flux Surface Radius $=468$ km", "Flux Surface Radius $=936$ km"]
tubes = []
for i, ax in enumerate(axs):
    AvgsP = ph.get_all_avgs(cfg, cfg.tube_radii[i], sim_params)
    tubes.append(AvgsP)
    ax.plot(periods, AvgsP[0], 'o', label=r"$F_\parallel^2$", mew=0, ms=5)
    ax.plot(periods, AvgsP[1], '_', label=r"$F_\perp^2$", mew=2, ms=5)
    ax.plot(periods, AvgsP[2], 'x', label=r"$F_\phi^2$", mew=2, ms=5)
    ax.set_ylabel("% Square \n Wave Flux")
    ax.set_title(titles[i])
    ax.xaxis.set_major_formatter(matplotlib.ticker.ScalarFormatter())
    ax.xaxis.set_ticks(periods)
    ax.set_ylim([10, 75])
    ax.set_xlim([25, 305])

axs[-1].set_xlabel("Period [s]")

#axs[0].legend(bbox_to_anchor=(1.06, 1.05))
plt.tight_layout(h_pad=0.1)

period_flux = ch6.save_figure('period-flux', fig=fig, fext='.pgf')
period_flux.caption = r"""Average percentage square wave flux plotted against period. For each vector component on the flux surface the value of the wave flux squared along one field line is taken and then the fraction of the square total calculated, and then averaged over all time. This provides a high-level overview of the relative strengths of each mode. The azimuthal component is shown as red crosses, the parallel component as blue circles and the perpendicular component is shown as green dashes. The top panel displays the average wave flux for the flux surface closest to the centre of the domain at $r=156$ km the second panel at $r=468$ km and the bottom panel at $r=936$ km."""

\end{pycode}

\py[chapter6]|period_flux|

\section{Conclusion}\label{sec:conclusion}

In this chapter the same logarithmic spiral driver that was studied in \cref{ch:drivers,ch:expfac}, but varied the oscillatory period of the driver.

