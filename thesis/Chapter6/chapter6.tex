% !TeX root = ../smumford_thesis.tex
%*****************************************************************************************
%*********************************** Sixth Chapter ***************************************
%*****************************************************************************************
\begin{pycode}[chapter6]
from __future__ import print_function
ch6 = texfigure.Manager(pytex, number=6, base_path='./Chapter6/')

from sacconfig import SACConfig
cfg = SACConfig()

from streamlines import Streamlines
\end{pycode}

\chapter{Effects of Period on MHD Wave Generation from a Logarithmic Spiral Driver}\label{ch:period}

\Cref{ch:drivers,ch:expfac} studied the effects of the driving velocity profile and the logarithmic spiral expansion factor ($B_L$) on the MHD wave excitation.
In both of these previous chapters, arbitrary periods were chosen, in this chapter the effect of this choice of period on the wave excitation by the logarithmic driver is studied.
The solar photosphere is populated with an outstanding variety of different frequency wave modes.
Acoustic (p-mode) waves have a wide frequency spectra, with a peak power at 5 minutes, and a large number of MHD waves at different frequencies have been observed in the low solar atmosphere; \cite{Freij2014,Dorotovic2014} observe oscillations in magnetic pores at periods ranging from 3 minutes to 25 minutes; \cite{morton2011} observe sausage modes with periods ranging from $30$ to $447$ seconds; \cite{fujimura2009} observe oscillations with periods between $3$ and $6$ minutes in pores and between $4$ and $9$ minutes in the inter-granular lanes.
It is therefore interesting to study a range of possible frequencies for our driver, to see what effects this has on the excitation of MHD modes.

\section{Simulation Configuration}\label{sec:periodconfig}
This chapter employs the same magnetohydrostatic background as \cref{ch:drivers,ch:expfac} which is described in \cref{sec:mhsbackground}.
The plasma is also driven by the same logarithmic spiral driver as given in \cref{eq:Slog,eq:slog5}, the expansion factor is selected as the central point of the parameter sweep performed in \cref{ch:expfac}, \py[chapter6]|r"$B_L={}$".format(cfg.exp_fac)|.
A plot of the driver profile is shown in \cref{fig:slog-profile}.

\begin{pycode}[chapter6]

#Use Equation 1 to calculate the vector field in a 2D plane to plot it.
time = np.linspace(0,60,480)
dt = time[1:] - time [:-1]
period = 240.

x = np.linspace(7812.5,1992187.5,128)
y = np.linspace(7812.5,1992187.5,128)

x_max = x.max()
y_max = y.max()

xc = 1.0e6
yc = 1.0e6

xn = x - xc
yn = y - yc

delta_x=0.1e6
delta_y=0.1e6

xx, yy = np.meshgrid(xn,yn)
exp_y = np.exp(-(yn**2.0/delta_y**2.0))
exp_x = np.exp(-(xn**2.0/delta_x**2.0))

exp_x2, exp_y2= np.meshgrid(exp_x,exp_y)
exp_xyz = exp_x2 * exp_y2


#==============================================================================
# Define Driver Equations and Parameters
#==============================================================================
#A is the amplitude, B is the spiral expansion factor
A = 10

#Tdamp defines the damping of the driver with time, Tdep is the ocillator
tdamp = lambda time1: 1.0 #*np.exp(-(time1/(period)))
tdep = lambda time1: np.sin((time1*2.0*np.pi)/period) * tdamp(time1)

#Define a peak index to use for scaling in the inital frame
max_ind = np.argmax(tdep(time) > 0.9998)

#Logarithmic
B = 0.05
phi = np.arctan2(1,B)
theta = np.arctan2(yy,xx)

uy = np.sin(theta + phi)
ux =  np.cos(theta + phi)

vx = lambda time1: (ux / np.sqrt(ux**2 + uy**2)) * exp_xyz * tdep(time1) * A
vy = lambda time1: (uy / np.sqrt(ux**2 + uy**2)) * exp_xyz * tdep(time1) * A

vv = np.sqrt(vx(time[max_ind])**2 + vy(time[max_ind])**2)

# Calculate Streamline
slines = Streamlines(x,y,vx(time[max_ind]),vy(time[max_ind]),maxLen=7000,
x0=xc, y0=yc, direction='forwards')

#============================================================================
# Do the Plotting
#============================================================================

fig = plt.figure(figsize=texfigure.figsize(pytex))
ax = plt.subplot()
im = ax.imshow(vv, cmap='cubehelix', extent=[7812.5,x_max,7812.5,y_max])
im.set_norm(matplotlib.colors.Normalize(vmin=0,vmax=10))

Sline, = ax.plot(slines.streamlines[0][0],slines.streamlines[0][1],color='red',linewidth=2, zorder=40)

#Add colourbar
divider = make_axes_locatable(ax)
cax = divider.append_axes("right", size="5%", pad=0.2)
cbar = plt.colorbar(im,cax)
cbar.set_label(r"$|\vec{V}|$ [ms$^{-1}$]")
scalar = matplotlib.ticker.ScalarFormatter(useMathText=False,useOffset=False)
scalar.set_powerlimits((-3,3))
cbar.formatter = scalar
cbar.ax.yaxis.get_offset_text().set_visible(True)
cbar.update_ticks()
#cbar.solids.set_rasterized(True)
cbar.solids.set_edgecolor("face")

#Add quiver plot overlay
#qu = ax.quiver(x,y,vx(time[max_ind]),vy(time[max_ind]),scale=25*A,color='#00DDFF',zorder=20)
ax.axis([8.0e5,12.0e5,8.0e5,12.0e5])

ax.xaxis.set_major_formatter(scalar)
ax.yaxis.set_major_formatter(scalar)
ax.xaxis.set_major_locator(matplotlib.ticker.MaxNLocator(5))
ax.yaxis.set_major_locator(matplotlib.ticker.MaxNLocator(5))
ax.xaxis.get_offset_text().set_visible(False)
ax.yaxis.get_offset_text().set_visible(False)
ax.set_xlabel("X [Mm]")
ax.set_ylabel("Y [Mm]")

fig.tight_layout()

slog_fig = ch6.save_figure('slog-profile', fig=fig, fext='.pdf')
slog_fig.caption = r"Horizontal velocity profile of the logarithmic spiral driver with expansion factor $B_L = {}$. The magnitude of velocity is shown by the colour map and the cyan arrows follow the vector field. The red line is a velocity streamline seeded in the centre of the domain.".format(cfg.exp_fac)
slog_fig.figure_width=r'\columnwidth'
\end{pycode}

\py[chapter6]|slog_fig|

