% !TeX root = ../smumford_thesis.tex
%*****************************************************************************************
%*********************************** Eigth Chapter ***************************************
%*****************************************************************************************
%\begin{pycode}[chapter8]
%from __future__ import print_function
%ch8 = texfigure.Manager(pytex, number=8, base_path='./Chapter8/')
%\end{pycode}

\chapter{SunPy: A Tool for Open Solar Physics}\label{ch:sunpy}

The SunPy project \citep{thesunpycommunity2015a} aims to facilitate and promote the use and development of a community-led, free and open-source solar data-analysis software based on the scientific Python environment.
Utilizing the scientific Python environment which has an extensive tool-set, allows for the creation of high-quality data-analysis software package for Solar Physics.
The adoption of Python as a scientific programming language is, in part, due to that fact that Python is a free, general-purpose, powerful, and easy-to-learn high-level programming language.
This fact has fostered a large community that is slowly growing in many fields both inside and outside of science, where Python is also widely used in areas such as `big data' analytics, web development, and educational environments.

A perfect example is the \texttt{pandas} \citep{mckinney2010, mckinney2012} Python package, which was started to simply analysis of financial data and has since grown into a generalised time-series data-analysis package.
Further, Python has seen extensive adoption within the astronomy community \citep{greenfield2011} which is a field that shares many goals and challenges with the solar physics community.

The development of a package such as SunPy is made possible by the rich ecosystem of scientific packages available in Python.
Core packages such as \texttt{NumPy}, 
\texttt{SciPy} \citep{jones2001}, and \texttt{matplotlib} \citep{hunter2007} provide the basic functionality expected of a scientific programming language, such as array manipulation, core numerical algorithms, and visualisation, respectively.
Building upon these foundations, packages such as \texttt{astropy} \citep[astronomy;][]{theastropycollaboration2013}, \texttt{pandas} \citep[time-series;][]{mckinney2012}, and \texttt{scikit-image} \citep[image processing;][]{vanderwalt2014} provide more domain-specific functionality.

Within Solar Physics, the SolarSoft (SSW) library \citep{freeland1998} is widely used, which is built upon the IDL (Interactive Data Language), a proprietary data-analysis environment. In contrast, SunPy is a open-source package that provides the \textit{core} tools for solar data analysis and aims to be a free and modern alternative.

\section{Community Development}
SunPy is a community developed package, the source code is hosted on GitHub \url{https://github.com/sunpy} and two mailing lists and an IRC channel serve as the projects communication channels.
This collaborative nature of the project is its greatest strength.
This chapter will my primary contributions to the project, but it is to be understood that they were all developed in a collaborative manner with the other members of the SunPy community.

I have been involved heavily in the SunPy project for the past three years.
During this time I have mentored three students through the Google Summer of Code and ESA Summer of Code in Space programs, founded the SunPy board as a governance structure for the project and served as lead developer.
During my time as lead developer I have strived to integrate the SunPy project with the wider scientific community and especially the astrophysical community.
I have contributed to many different projects to add or fix features for use by SunPy, especially in Astropy where I have helped guide design and creation of multiple features used in SunPy.

\section{Representation of Imaging Data}

One of the main submodules of SunPy is the `Map' submodule which presents a unified interface to different sources of imaging data.
Most image data in solar physics is stored in the FITS (Flexible Image Transport System) file format \citep{pence2010}.
However, even where data to be loaded by SunPy is in the FITS file format the critical information, the metadata, describing the observations is not stored in a consistent manner.
Despite the development of standard ways to describe coordinate information in FITS files (FITS-WCS) \citep{greisen2002,thompson2006} data generated before these standards were adopted and other meta data not related to the coordinates of the images do not follow these standards.
The problem of non-standard meta data is the primary motivator for the architecture of the SunPy `Map' submodule.

The `Map' module employs the object orientated (OO) nature of the Python programming language to allow specific data sources, such as different instruments, to map their meta data to a standard SunPy interface to that data.
The \verb|GenericMap| class defines this standard interface, it provides human readable properties to access common meta data, such as the FITS-WCS coordinate information and other common meta data such as instrument and date.
The \verb|GenericMap| class defines these mappings using a strict interpretation of the standards and the most common conventions in solar physics.
Specific instruments that deviate from these standards, or desire other specialisations, define a subclass of the \verb|GenericMap| class to override any properties to provide the correct information to the user in the standard manner.
This design results in a class hierarchy where each instrument has a associated class, and the user would have to manually load the data into the correct class to obtain the desired behaviour.
To improve the user experience a unified interface to the `Map' module was designed for the first official release of SunPy (v0.2).

The function of this unified interface was to extract the data from a file provided by the user and load it into the correct class.
The implementation of this original interface, named \verb|make_map| was complex and hard to extend.
It is for this reason that for SunPy v0.3 the interface to the Map submodule was redesigned, a process which the I led.
The new design of the `Map' module made use of a system where each subclass of \verb|GenericMap| registers itself with a `Factory' class upon its initilisation.
This registration procedure supplies a function to the factory class to use to ascertain from the meta data which class should be used to interpret the data.
The implementation of the \verb|RegisteredFactoryBase| was provided by a contributor to SunPy, and the integration of it into the `Map' module was performed by the author.
The \verb|Map| class provides a high-level interface to the user, it is deisgned to be able to parse a wide variety of inputs, including filenames, filenames with wildcards, and raw data and meta data pairs.
This flexibility makes it simple for the user to initialise a Map object specific to the data being loaded, without the user having to understand or appreciate the underlying architecture.
This design also makes it simple for developers of SunPy or advanced users to create a Map class for a custom instrument or data source, and continue using their existing code.

\section{Solar Physics Coordinate Systems}

\cite{thompson2006} describes a set of commonly used coordinate systems for solar physics data analysis.
These systems are frequently used by any person performing analysis of solar imaging data.
As of SunPy v0.6 SunPy handles representation of these coordinate systems by pairs of floating point numbers, and the conversion between the systems by calls to functions where the extra information needed to perform the conversion is explicitly provided.
This system, while functional for the current internal needs of the SunPy library, is not intuitive or powerful for the end users of the library.

The Astropy library \citep{theastropycollaboration2013} in its v0.4 release provided a modular and general framework for the representation of celestial coordinate systems and the transformations between them.
Prior to the development and subsequent release of this new submodule, the author collaborated with the Astropy developers to ensure that the design of this new system would be compatible with the requirements of solar physicists and the coordinate systems described in \cite{thompson2006}.
The author also contributed to the implementation of the Astropy submodule, by co-writing the coordinate representation system.

Since the development of the Astropy coordinates framework, I have led the work to implement the solar physics coordinates systems within the SunPy library.
This work is expected to form a part of the upcoming v0.7 release of the SunPy library.
The implementation of these systems presented some challenges specific to solar physics.
The main convention difference between astronomy and solar astronomy is the `wrapping' of the longitude coordinate in spherical coordinate systems.
In most astrophysical systems the longitude runs between $0^\circ$ and $360^\circ$, whereas in solar physical systems, including the most commonly used helio-projective system, the longitude coordinate takes values between $-180^\circ$ and $+180^\circ$.
This difference in convention required subclassing the Astropy `Representation' classes to change the default values of the coordinate wrapping.
This itself lead to extra requirements to sanitise inputs and ensure the Astropy library did not override the change in defaults.

The second major solar physics specific challenge that was overcome in the implementation was the transition from the projective coordinate frame of imaging data to the three-dimensional solar centric coordinate systems commonly used to describe features independent of the location of the observer.
The mathematics of this coordinate conversion are described in \cite{thompson2006}, and regularly utilised by a multitude of codes and analysis pipelines.
The process however, is not without its assumptions and it was important in the design of the user interface that no flexibility was lost and all options exposed to the user.
This required that the fundamental assumption of this conversion, that the Sun emits radiation on a solid sphere of known radius was customisable by the user.
The assumption of a fixed radius sphere allows the calculation of the distance between an observer and the point on the sphere corresponding to a position in an image.
This calculation of distance converts a two dimensional projective coordinate system into a three dimensional observer centric coordinate system, which in turn allows the conversion to different coordinate frames.
This conversion step is only applied either when the user explicitly requests it without changing coordinate system, or when it is required to perform a coordinate transformation requested by the user.
Conversion from the two dimensional to the three dimensional system uses the \verb|RSun| attribute of the \verb|HelioProjectiveFrame| class which allows the user to customise the physical radius of the Sun being observed.

This implementation of the solar physics coordinate systems on top of the Astropy coordinates module provides a significant advantage to the SunPy library.
Due to the coordinate independent nature of the design, tools that are built for astrophysical applications can support solar physics coordinate systems with no extra work required by the authors of these tools.
This, therefore, provides both an excellent, user friendly interface to a set of very common solar physics coordinate representations and transformations while getting access to the high quality tools which make use of this framework.

\section{The Future of SunPy}

Adoption of the SunPy library by the wider solar physics community is predicated on the provision of high quality tools which enable new and exciting discoveries.
The new features described in this chapter form part of the wider effort to enable solar physics data analysis within the Python ecosystem.
The author has contributed to the SunPy library in many smaller ways, and has acted in the role of lead developer, while building the community around the project.

