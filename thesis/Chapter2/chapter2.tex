%*****************************************************************************************
%*********************************** Second Chapter ***************************************
%*****************************************************************************************
\label{ch:Background}
\chapter{Background}  %Title of the Second Chapter

\begin{pycode}[chapter2]
ch2 = texfigure.Manager(pytex, number=2, base_path='./Chapter2/')
\end{pycode}

%{\Large Knowledge at start:}
%\begin{itemize}
%	\item The coronal heating problem
%	\item MHD waves as a solution to the corona heating problem
%	\item photospheric dynamics
%	\item Magnetic wave guides.
%\end{itemize}
%
%{\Large Knowledge at end:}
%\begin{itemize}
%	\item The ideal MHD equations
%	\item Wave solutions for a uniform media
%	\item velocity perturbation calculations
%	\item Wave flux calcs
%	\item SAC, and numerical solutions to the ideal MHD equations
%\end{itemize}

%\section*{Outline}
%
%\begin{enumerate}
%
%	\item MHD
%	\begin{enumerate}
%		\item Ideal MHD
%		\item MHD Waves
%		
%	\end{enumerate}
%	\item Computational Methods
%	\begin{enumerate}
%		\item Numerical Solutions to Partial Differential Equations
%		\item Numerical MHD
%		\item The Sheffield Advanced Code (SAC)
%	\end{enumerate}
%	\item Magneto-static Background Conditions
%	
%\end{enumerate}
	

%********************************** %First Section  **************************************
\section{Magnetohydrodynamics}\label{sec:MHD}

Magnetohydrodynamics (MHD) is the description of a plasma as a single conducting fluid.
%TODO: Approximations
The MHD description makes a series of approximations as to the nature of plasmas, LIST THEM..
These approximations hold for the vast majority of solar plasma, they do however break down in the solar wind, where the electron-ion collision time is sufficiently long that the single fluid description breaks down.

A further approximation that can be added to the MHD equations is that of a perfectly conducting single fluid, this is called ideal MHD, and this also holds for the majority of solar plasmas, however it breaks down in regions of plasma where the magnetic field lines change configuration, or reconnect with one another.
The ideal MHD equations are given below:
\newcommand{\condev}{\left(\frac{\partial}{\partial t} + \underline{\mathrm{v}}\cdot\nabla\right)}
\begin{align}                                                         
    \dfrac{\partial \rho }{\partial t} + \nabla \cdot (\rho \vec{v}) =       
    0,
    \tag{Mass Conservation}\\
    %                               
    \rho  \condev\vec{v} =
    -\nabla p + \dfrac{1}{\mu}(\nabla \times \vec{B}) \times \vec{B} + \rho \vec{g},
    \tag{Equation of Motion}\\
    %
    \condev \left(\dfrac{p}{\rho^\gamma} \right)  = 0,
    \tag{Energy Equation}\\
    %
    \dfrac{\partial \mathrm{B}}{\partial t} = \nabla \times (\vec{v} \times \vec{B}),
    \tag{Induction Equation}               
\end{align}
subject to
\begin{align}
    \nabla \cdot \vec{B} = 0,
    \tag{Solenoid Equation}\\
    %
    p = \mathrm{k_B} \dfrac{\rho}{\mathrm{m}} \mathrm{T},
    \tag{Ideal Gas Law}\\
    %
    \underline{\mathrm{E}} = - \vec{v} \times \vec{B},
    \tag{Ohm's Law}\\
    %
    \underline{\mathrm{j}} = \nabla \times \vec{B}/ \mu.
    \tag{Electric Current}                          
\end{align}
Where $\rho$ is the density, $\underline{\mathrm{v}}$ is the velocity, $p$ is the pressure, $\gamma$ is the ratio of specific heats ($5/3$), $\underline{\mathrm{B}}$ is the magnetic field, $\mathrm{k_B}$ is Boltzmann's constant, $\mathrm{m}$ is the mass, $\mathrm{T}$ is the temperature, $\underline{\mathrm{E}}$ is the electric field, $\underline{\mathrm{j}}$ is the current density and $\mu$ is the vacuum permeability. 


\begin{align*}                                                         
    \vec{B} &= \vec{B}_0 + \vec{B}_1(\vec{r},t)\\               
    \vec{v} &= 0 + \vec{v}_1(\vec{r},t)\\               
    p &= p_0 + {p_1}(\vec{r},t)\\               
    \rho &= \rho_0 + {\rho_1}(\vec{r},t)\\              
\end{align*}
Here, subscripts are used to separate out the background ($\mathrm{B}_0$) and perturbation ($\mathrm{B}_1$) quantities.
There is assumed to be no background flow and that for all perturbations they are much smaller than the background value (e.g., $\mathrm{B}_0 \gg \mathrm{B}_1$).      
This leads to the linearised MHD equations,
\begin{align}                                                         
    \dfrac{\partial \rho_1 }{\partial t} + (\vec{v}_1 \cdot \nabla)\rho_0 + + \rho_0 (\nabla \cdot \vec{v}_1) =       
    0,
    \tag{Mass Conservation}\\
    %
    \rho_0 \dfrac{\partial \vec{v}_1}{\partial t} =
    -\nabla p_1 + \dfrac{1}{\mu}(\nabla \times \vec{B}_1) \times \vec{B}_0 + \rho_1 \vec{g},
    \tag{Equation of Motion}\\
    %
    \dfrac{\partial p_1}{\partial t} + (\vec{v}_1 \cdot \nabla)p_0 - c_s^2 \left( \dfrac{\partial \rho_1}{\partial t} + (\vec{v}_1 \cdot \nabla)\rho_0 \right) = 0,
    \tag{Energy Equation}\\
    %
    \dfrac{\partial \mathrm{B}_1}{\partial t} = \nabla \times (\vec{v}_1 \times \vec{B}_0),
    \tag{Induction Equation}\\
    %
    \nabla \cdot \vec{B}_1 = 0,
    \tag{Solenoid Equation}               
\end{align}
where we can define the first characteristic speed in MHD; the sound speed, $c_s^2 = \frac{\gamma p_0}{\rho_0}$.
There is another important characteristic speed and that is the Alfv\`{e}n speed, $c_A^2 = \frac{\mathrm{\underline{B}}_0^2}{\sqrt{\rho_0}}$.


\subsection{MHD Waves}\label{sec:MHDwaves}

\subsection{Velocity Perturbations}\label{sec:Vpert}

\subsection{Calculating Wave Flux}\label{sec:waveflux}


To calculate the relative strengths of the excited waves we compute the `wave energy flux' vector everywhere in the domain using Equation \ref{eq:wave_energy}.
\begin{equation}
\vec{F}_{wave} \equiv \widetilde{p}_k \vec{v} + \frac{1}{\mu_0} \left(\vec{B}_b \cdot \vec{\widetilde{B}}\right) \vec{v} - \frac{1}{\mu_0}\left(\vec{v} \cdot \vec{\widetilde{B}} \right) \vec{B}_b,
\label{eq:wave_energy}
\end{equation}
where a subscript $b$ represents a background variable, a tilda represents a perturbation from the background conditions and $p_k$ represents kinetic pressure.

This equation has been widely used to calculate the energy contained in linear MHD perturbations.
It is discussed in detail in \cite{bogdan2003} where it is compared to the `true' MHD flux for linear perturbations and found to be generally clearer. 
It is used in \cite{vigeesh2009, vigeesh2012, khomenko2012}. 
For a full derivation and discussion relating to time-averaging see \cite{leroy1985}.
Calculating wave energy flux using Equation \ref{eq:wave_energy} provides a vector which is useful in plotting time distance diagrams and analysing wave modes.

\section{Computational Methods}\label{sec:numericalmethods}

\section{Sheffield Advanced Code}\label{sec:SAC}