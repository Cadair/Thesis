% !TeX root = ../smumford_thesis.tex
%*****************************************************************************************
%*********************************** Second Chapter ***************************************
%*****************************************************************************************
\chapter{Background}\label{ch:background}  %Title of the Second Chapter

\begin{pycode}[chapter2]
ch2 = texfigure.Manager(pytex, number=2, base_path='./Chapter2/')
\end{pycode}
	

\section{Magnetohydrodynamics}\label{sec:MHD}

Ideal magnetohydrodynamics (MHD) is the description of a plasma as a single perfectly conducting fluid.
This description of the plasma has certain constraints on its validity.
For the purposes of study in this thesis, the ideal MHD equations are a very applicable description of the plasma in the solar atmosphere.
The first assumption the MHD description makes regarding the nature of the solar plasma is that it behaves like a fluid.
This means that there are very frequent collisions between the particles that comprise the plasma.
Connected to this is the assumption that the temperature of the electrons and the ions are equal, implying that there are frequent interactions between the two species, and that the plasma can be treated as a single fluid.
Secondly, the MHD description is only valid in a certain window of temporal and length scales.
The characteristic length of the plasma has to be sufficiently large so that the particle motion around the magnetic field can be ignored.
The temporal scales also have to be substantially longer than the frequency of the kinetic motions.
However, the temporal scale has to be short enough that the slow dissipative effects, such as resistive decay of the magnetic field, can be neglected.
Two other approximations are made, which enable the description of the plasma as a single fluid: the quasi-neutrality assumption, which is the assumption that there are very similar numbers of positive and negative charges present in the plasma; and the assumptions that the relative velocities of the positive and negative charges are small.
Finally, but very importantly, it is assumed that the plasma is non-relativistic, i.e. the motions of the plasma are substantially smaller than the speed of light.
The application of all these assumptions leads to a formulation of the equations governing the motion of the plasma based on Maxwell's equations and the equations of gas dynamics, these are the ideal MHD equations, which are given below:
\newcommand{\condev}{\left(\frac{\partial}{\partial t} + \vec{v}\cdot\nabla\right)}

\begin{align}                                                  
    \dfrac{\partial \rho }{\partial t} + \nabla \cdot (\rho \vec{v}) = 0,
    &&\text{(Mass Conservation)}\label{eq:imhd_mass}\\
    %                               
    \rho  \condev\vec{v} + \nabla p - \dfrac{1}{\mu}(\nabla \times \vec{B}) \times \vec{B} - \rho \vec{g} = 0,
    &&\text{(Equation of Motion)}\label{eq:imhd_motion}\\
    %
    \frac{\partial p}{\partial t} + \vec{v}\cdot\nabla p + \gamma p \nabla \cdot \vec{v}  = 0,
    &&\text{(Energy Equation)}\label{eq:imhd_equation}\\
    %
    \dfrac{\partial \vec{B}}{\partial t} - \nabla \times (\vec{v} \times \vec{B}) = 0,
    &&\text{(Induction Equation)}\label{eq:imhd_induction}
\end{align}
subject to
\begin{align}
    \nabla \cdot \vec{B} = 0,
    &&\text{(Solenoidal Condition)}\\
    %
    p = \mathrm{k_B} \dfrac{\rho}{\mathrm{m}} \mathrm{T},
    &&\text{(Ideal Gas Law)}                    
\end{align}
where $\rho$ is the density, $\vec{v}$ is the velocity, $p$ is the pressure, $\gamma$ is the ratio of specific heats (usually taken as $5/3$), $\vec{B}$ is the magnetic field, $\mathrm{k_B}$ is Boltzmann's constant, $\mathrm{m}$ is the mass, $\mathrm{T}$ is the temperature, and $\mu$ is the magnetic permeability of free space~\citep{goedbloed2004}.


\subsection{MHD Waves}\label{sec:MHDwaves}
Just like a non-ionised fluid, which supports a sound wave, due to the restoring force of the pressure, plasma supports wave phenomena~\citep{alfven1942}.
Waves in plasmas also interact with the magnetic field, and the coupling of the magnetic field to the motion of the plasma.
This leads to the presence of a wide variety of wave modes in plasma, dependent upon the geometry and physical properties of the plasma being perturbed~\citep{jess2015}.
For the analysis performed in this chapter we shall consider a plasma with a static, and uniform background, the magnetic field shall be of a constant strength and aligned solely with the $z$ axis.
As will be demonstrated, this configuration leads to the existence of three wave modes, called the fast-magnetoacoustic wave, the slow-magnetoacoustic wave and the Alfv\'en wave.
These are the only wave modes an infinite uniform plasma supports, in more complex geometries a wider variety of modes exist.
This choice of geometry is a highly simplified model of the solar atmosphere, the observed atmosphere has both strong non-uniformity in density and magnetic field as well as many other properties which violate these assumptions.
As will be discussed later, this simplified model allows the construction of a model which can be easily applied to the numerical simulations and physical inferences drawn.

In this section we are going to summarise the derivation of the MHD wave equation for a uniform plasma.
The starting point for this analysis is the ideal MHD equations that are described in \cref{sec:MHD}.
Our static background conditions are described by \cref{eq:linearize} below,
\begin{equation}\label{eq:linearize}
    \begin{aligned}                                                    
        \vec{B} &= \vec{B}_b + \tilde{\vec{B}}(\vec{r},t),\\
        \vec{v} &= 0 + \tilde{{v}}(\vec{r},t),\\
        p &= p_b + \tilde{p}(\vec{r},t),\\
        \rho &= \rho_b + {\tilde{\rho}}(\vec{r},t),\\
    \end{aligned}
\end{equation}
where the subscript $b$ denotes a background quantity, and the tilde denotes a perturbation to this quantity.
The perturbation is much smaller than the background quantity.
In assuming a static background plasma, we are implicitly assuming the solar atmosphere largely static, and there are no bulk flows present.
This is clearly false on most time scales observed on the Sun, however, the assumption going forward is that on the short ($\approx 10$ minute) time scales discussed for the rest of this thesis, that we can assume the background solar atmosphere is static.
The presence of flows and other background movement in the plasma will change the behaviour of wave modes in the solar atmosphere and is the subject of study. %TODO: reference
Substituting \cref{eq:linearize} into the ideal MHD equations and neglecting 2nd order or higher terms and any gradients of the homogeneous background quantities, leads to the linearised MHD equations given below,
\begin{align}                                                         
    \dfrac{\partial \tilde{\rho} }{\partial t} + \rho_b (\nabla \cdot \tilde{\vec{v}}) =       
    0,
    &&\text{(Mass Conservation)}\label{eq:lmhd_mass}\\
    %
    \rho_0 \dfrac{\partial \tilde{\vec{v}}}{\partial t} =
    -\nabla \tilde{p} + \dfrac{1}{\mu}(\nabla \times \tilde{\vec{B}}) \times \vec{B}_b,
    &&\text{(Equation of Motion)}\label{eq:lmhd_motion}\\
    %
    \dfrac{\partial \tilde{p}}{\partial t} + \gamma p_b \left( \nabla \cdot \tilde{\vec{v}} \right) = 0,
    &&\text{(Energy Equation)}\label{eq:lmhd_energy}\\
    %
    \dfrac{\partial \tilde{\vec{B}}}{\partial t} = \nabla \times (\tilde{\vec{v}} \times \vec{B}_b),
    &&\text{(Induction Equation)}\label{eq:lmhd_induction}\\
    %
    \nabla \cdot \tilde{\vec{B}} = 0.
    &&\text{(Solenoidal Condition)}\label{eq:lmhd_solenoid}              
\end{align}
The sound speed and the Alfv\'en speed for the static background state are given by 
\begin{align}
    c &\equiv \sqrt{\frac{\gamma p_b}{\rho_b}},\label{eq:soundspeed}\\
    \vec{b} &\equiv \frac{\vec{B}_b}{\sqrt{\rho_b}},\label{eq:alfvenspeed}
\end{align}
respectively.

To simplify the analysis and, with the aim of describing the velocity perturbations in \cref{sec:Vpert}, it is helpful to describe the system of linearised MHD equations just in terms of the velocity $\vec{v}$.
This is achieved by substituting \cref{eq:lmhd_mass,eq:lmhd_energy,eq:lmhd_induction} into \cref{eq:lmhd_motion}, which gives
\begin{equation}
    \frac{\partial^2 \tilde{\vec{v}}}{\partial t^2} - \left( (\vec{b} \cdot \nabla)^2 \vec{I} + (b^2 + c^2) \nabla\nabla - \vec{b} \cdot \nabla (\nabla\vec{b} + \vec{b}\nabla) \right) \cdot \tilde{\vec{v}} = 0,\label{eq:mhdwaveeq}
\end{equation}
where $\vec{I}$ is the identity matrix, $\vec{b}$ is the Alfv\'en speed given in \cref{eq:alfvenspeed} and $c$ is the sound speed given in \cref{eq:soundspeed}~\citep{goedbloed2004}.
This equation can be solved using plane wave solutions, i.e. solutions of the form $\rho(\vec{r}, t) = \hat{\rho} e^{i(\vec{k}\cdot\vec{r} - \omega t)}$, which equate to performing the following substitutions into \cref{eq:mhdwaveeq}: $\frac{\partial}{\partial t} \rightarrow - i \omega$ and $\nabla \rightarrow i \vec{k}$.
Which, given in matrix form yields the relation:
\begin{equation}
\begin{pmatrix}
    &-k_\perp(b^2+c^2)-k_\parallel^2 b^2			& 0			& -k_\perp k_\parallel c^2\\
    & 0									&-k_\parallel b^2	& 0\\
    &-k_\perp k_\parallel c^2					& 0			& -k_\parallel c^2
\end{pmatrix}
\begin{pmatrix}
v_\parallel\\
v_\phi\\
v_\perp
\end{pmatrix}
= - \omega^2
\begin{pmatrix}
v_\parallel\\
v_\phi\\
v_\perp
\end{pmatrix}.\label{eq:eigenvalue}
\end{equation}
In this form the notation for the components of the velocity is with respect to the magnetic field. $v_\parallel$ is the velocity component parallel to the magnetic field, $v_\perp$ is perpendicular to the magnetic field in one plane and $v_\phi$ represents velocity perpendicular to the magnetic field in the other plane. It should also be noted that the notation for perturbed velocity is dropped for the rest of this chapter, as there is no background velocity, so all velocity terms are perturbations.

To obtain non-trivial solutions the determinant of \cref{eq:eigenvalue} is calculated,
\begin{equation}
    (\omega - k_\parallel^2 b^2)\left( \omega^4 - k^2(b^2+c^2)\omega^2 + k_\parallel k^2b^2c^2 \right) = 0,
\end{equation}
where $k = k_\parallel + k_\perp$.

Three physically interesting solutions can be found to \cref{eq:eigenvalue}, by equating the two terms, in brackets, to zero.
The first term $(\omega - k_\parallel^2 b^2)$ leads to the Alfv\'en wave solution:
\begin{align}
\omega^2_A &= k^2_\parallel b^2,\\
\omega_A &= \pm k_\parallel b,
\label{eq:omegaA}
\end{align}
where the positive solution is forward-propagating and the negative solution backward-propagating.
The second term leads to two solutions, describing propagation in both directions. 
Taking the square root of the second term, and then solving the quadratic,
\begin{equation}
    \omega^2_{s,f} = \frac{1}{2}k^2(b^2+c^2)\left(1 \pm \sqrt{1 - \sigma (k^2_\parallel/k^2)}\right)
    \label{eq:omegasf}
\end{equation}
where
\begin{equation}
    \sigma=\frac{4b^2c^2}{(b^2 + c^2)^2}
    \label{eq:sigmasf}
\end{equation}
which are the fast and slow magneto-acoustic modes, for the positive and negative solutions, respectively.

These three modes, and the $\omega = 0$ entropy mode are all the oscillatory solutions supported by a plasma with a uniform background.
In \cref{sec:Vpert}, the perturbation of the velocity vector caused by oscillations is identified, which will be used throughout this thesis for identification of these modes.


\subsection{Velocity Perturbations}\label{sec:Vpert}

The following chapters use the components of the velocity perturbation to identify and characterise the wave modes in the numerical domain.
In this section it shall be shown that the three wave modes derived in the last section each perturb a different component of the velocity with respect to the magnetic field vector.

First, consider the Alfv\'en wave, with its eigenfrequencies given by \cref{eq:omegaA}.
Substituting \cref{eq:omegaA} into \cref{eq:eigenvalue} we can clearly see that the only component of the velocity perturbed by the Alfv\'en wave is the $v_\phi$ component, or that perpendicular to the plane of $\vec{k}$.

Next, considering the slow and fast modes, and starting from \cref{eq:omegasf}, the velocity perturbations can be calculated.
Substituting \cref{eq:omegasf} into the eigenvalue equation leads to the following relationship between $v_\perp$ and $v_\parallel$:
\begin{equation}
    v_\parallel = \alpha_{s,f} \frac{k_\parallel}{k_\perp}v_\perp,
    \label{eq:vyvz}
\end{equation}
where
\begin{equation}
    \alpha_{s,f} = 1 - \frac{k^2b^2}{\omega^2_{s,f}}.
    \label{eq:alphasf}
\end{equation}

At this stage it is helpful to make a simplifying assumption about the domain in which we wish to derive this relationship.
In \cref{sec:mhsbackground}, the physical domain to be modelled in this thesis will be described such that it is entirely below the transition region and has plasma $\beta > 1$ everywhere.
In this region of the solar atmosphere the plasma pressure or kinetic pressure is much larger than the magnetic pressure.
In other words, the kinetic effects dominate the dynamics of the plasma.
This is called the high-$\beta$ regime, as $\displaystyle\beta = \frac{p_k}{p_m}$, or $\displaystyle\beta = \frac{2c^2}{\gamma b^2}$ in terms of $c$ and $\vec{b}$ as defined in \cref{eq:soundspeed,eq:alfvenspeed} above, so in a high-$\beta$ regime it is clear that $c^2 >> b^2$.
This assumption can be used to simplify \cref{eq:omegasf} because $(c^2 + b^2) \approx c^2$.
Applying this assumption to \cref{eq:sigmasf} and then to \cref{eq:omegasf}:

\begin{align}
    \sigma &\approx \frac{4b^2c^2}{(b^2+c^2)^2}\\
           &\approx \frac{4b^2c^2}{c^4}\\
           &\approx \frac{4b^2}{c^2}
\end{align}

\begin{align}
    \omega^2_{s,f} =& \frac{1}{2}k^2(b^2+c^2)\left(1 \pm \sqrt{1 - \sigma \frac{k^2_\parallel}{k^2}}\,\right),\\
                   =& \frac{1}{2}k^2c^2\left(1 \pm \sqrt{1 - \frac{4b^2}{c^2} \frac{k^2_\parallel}{k^2}}\,\right).\\
\end{align}
Let,
\begin{align}
    \delta =& \frac{4b^2}{c^2} \frac{k^2_\parallel}{k^2},\notag\\
\end{align}
such that,
\begin{align}
    \omega^2_{s,f} =& \frac{1}{2}k^2c^2\left(1 \pm (1 - \delta)^{\frac{1}{2}}\,\right).\label{eq:omegasf-highb1}
\end{align}
Performing a Taylor expansion of $(1-\delta)^{\frac{1}{2}}$ and considering $\delta << 1$, gives,
\begin{align}
    f(\delta) &= f(0) + \delta f^\prime(0) + O(\delta^2),\notag\\
    %
    (1-\delta)^{\frac{1}{2}} &= 1 - \frac{1}{2}\delta + O(\delta^2),\\
                             &= 1 - \frac{1}{2}\frac{4b^2}{c^2} \frac{k^2_\parallel}{k^2} + O(\delta^2),\\
    (1-\delta)^{\frac{1}{2}} &= 1 - \frac{2b^2}{c^2} \frac{k^2_\parallel}{k^2} + O(\delta^2).
\end{align}
Substituting this back into \cref{eq:omegasf-highb1} one obtains,
\begin{align}
    \omega^2_{s,f} =& \frac{1}{2}k^2c^2\left(1 \pm \left(1 - \frac{2b^2}{c^2} \frac{k^2_\parallel}{k^2}\right)\right).
\end{align}

It is at this point the slow mode and the fast mode need to be considered in isolation.
The fast mode is the solution where the positive root is taken, and the slow mode the negative root.
Considering the slow mode first:
\begin{align}
    \omega^2_{s} =& \frac{1}{2}k^2c^2\left(1 - \left(1 - \frac{2b^2}{c^2} \frac{k^2_\parallel}{k^2}\right)\right),\\
    =& \frac{1}{2}k^2c^2\left(\frac{2b^2k^2_\parallel}{k^2c^2}\right),\\
    \omega^2_{s} =& b^2k^2_\parallel.
\end{align}
Then the fast mode:
\begin{align}
    \omega^2_{f} =& \frac{1}{2}k^2c^2\left(1 + \left(1 - \frac{2b^2}{c^2} \frac{k^2_\parallel}{k^2}\right)\right)\\
                 =& \frac{1}{2}k^2c^2\left(2 - \frac{2b^2}{c^2} \frac{k^2_\parallel}{k^2}\right)\\
                 =& k^2c^2 - \frac{k^2c^2b^2}{c^2}\frac{k^2_\parallel}{k^2}\\
                 =& k^2c^2 - k^2b^2\frac{k^2_\parallel}{k^2}.
\end{align}
Applying the high $\beta$ approximation $c^2 >> b^2$,
\begin{align}
   \omega^2_{f} =& k^2c^2.
\end{align}
Having simplified $\omega_{s,f}$, $\alpha_{s,f}$ can be calculated:
\begin{align}
    \alpha_s =& 1 - \frac{k^2b^2}{\omega^2_s}\\
             =& 1 - \frac{k^2b^2}{b^2k_\parallel^2}\\
             =& 1 - \frac{k^2}{k_\parallel^2}\\
             =& 1 - \frac{k^2_\parallel + k^2_\perp}{k^2_\parallel}\\
             =& 1 - \frac{k_\parallel^2}{k_\parallel^2} + \frac{k^2_\perp}{k^2_\parallel}\\
             =& \frac{k^2_\perp}{k^2_\parallel},
\end{align}

\begin{align}
    \alpha_f =& 1 - \frac{k^2b^2}{\omega^2_f}\\
    =& 1 - \frac{k^2b^2}{k^2c^2}\\
    =& 1 - \frac{b^2}{c^2}.
\end{align}
Again, considering the high-$\beta$ approximation, $\displaystyle\frac{b^2}{c^2} << 1$
\begin{align}
    \alpha_f =& 1.
\end{align}

Substituting into \cref{eq:vyvz} for the slow mode the following relation is obtained:
\begin{align}
    v_\parallel =& \alpha_s \frac{k_\parallel}{k_\perp}v_\perp\\
                =& \frac{k^2_\perp}{k^2_\parallel}\frac{k_\parallel}{k_\perp}v_\perp\\
                =& \frac{k_\perp}{k_\parallel}v_\perp,
\end{align}
and, for the fast mode:
\begin{align}
    v_\parallel =& \alpha_f \frac{k_\parallel}{k_\perp}v_\perp\\
                =& \frac{k_\parallel}{k_\perp}v_\perp.
\end{align}

Finally, to demonstrate the relative strength of the parallel and perpendicular modes the following ratio is calculated:
\begin{equation}
    \frac{|v_z|}{|v_x|+|v_z|}.
\end{equation}
For the fast mode this becomes:
\begin{align}
\frac{|v_\parallel|}{|v_\perp|+|v_\parallel|} &= \frac{\frac{k_\parallel}{k_\perp}|v_\perp|}{|v_\perp|+\frac{k_\parallel}{k_\perp}|v_\perp|},\\
&= \frac{\frac{k_\parallel}{k_\perp}}{1+\frac{k_\parallel}{k_\perp}},\\
&= \frac{1}{\frac{k_\perp}{k_\parallel}+1}.
\end{align}
By assuming that the waves propagate parallel to the magnetic field, i.e. upwards through the atmosphere, implying $k_\parallel >> k_\perp$, and $\displaystyle \frac{k_\perp}{k_\parallel} << 1$ it can be seen that $|v_\parallel| >> |v_\perp|$, and therefore the fast mode can be seen to perturb the parallel component of the velocity.

However, for the slow mode this becomes:
\begin{align}
    \frac{|v_\parallel|}{|v_\perp|+|v_\parallel|} &= \frac{\frac{k_\perp}{k_\parallel}|v_\perp|}{|v_\perp|+\frac{k_\perp}{k_\parallel}|v_\perp|},\\
    &= \frac{\frac{k_\perp}{k_\parallel}}{1+\frac{k_\perp}{k_\parallel}},\\
    &= \frac{1}{\frac{k_\parallel}{k_\perp} + 1},
\end{align}
Again, assuming that $k_\parallel >> k_\perp$, \textit{i.e.} $\displaystyle \frac{k_\parallel}{k_\perp} >> 1$, so that for the slow mode, $|v_\parallel| << |v_\perp|$.
The slow mode perturbs, predominately, the component of the velocity perpendicular to the magnetic field.  


\subsection{Calculating Wave Flux}\label{sec:waveflux}

To calculate the relative strengths of the excited waves we compute the `wave energy flux' vector everywhere in the domain.
To do this we use the equation for the wave energy flux given in equation (2.14) of~\cite{leroy1985}.
This equation can be written using notation similar to the rest of this thesis, by applying $\vec{H} = \vec{B}/\mu_0$ and the vector identity $\vec{a} \times (\vec{b} \times \vec{c}) = \vec{a} \cdot \vec{c}\vec{b} - \vec{a} \cdot \vec{b}\vec{c}$.
The result is given in \cref{eq:wave_energy} below.
\begin{equation}
\vec{F}_{\textbf{wave}} \equiv \widetilde{p}_k \vec{v} + \frac{1}{\mu_0} \left(\vec{B}_b \cdot \vec{\widetilde{B}}\right) \vec{v} - \frac{1}{\mu_0}\left(\vec{v} \cdot \vec{\widetilde{B}} \right) \vec{B}_b,
\label{eq:wave_energy}
\end{equation}
where a subscript $b$ represents a background variable, a tilde represents a perturbation from the background conditions and $p_k$ represents kinetic pressure.

The validity of this equation for the wave energy flux is widely discussed, %TODO: Ref
while it can be derived by linearising the total MHD energy flux to the second order, as described in~\cite{leroy1985}, it is only one of many solutions to the fundamental energy conservation equation.
This fact is discussed in detail in Section 4 of~\cite{bogdan2003}, where the utility of this method of calculating wave energy flux for numerical simulations of the solar atmosphere is considered.
\cite{bogdan2003} compares the above definition of the wave energy flux to the total nonlinear MHD energy flux, it is shown that the total energy flux is dominated by a `stationary flow' of energy, or local effects, rather than the energy contained in the wave like phenomena that this thesis wishes to study.
It is because of this that this method of quantifying the energy contained within the MHD waves generated in the simulation domain is chosen.
It should not however, be taken as a perfect metric, which is why it is chosen only to use \cref{eq:wave_energy} to calculate the relative strengths of the wave modes.
This analysis provides the desired result: an understanding of how the different plasma conditions and wave drives change the spectra of MHD wave modes generated, while eliminating any concern about the accuracy of \cref{eq:wave_energy} when accounting for absolute energy values.



\section{Computational Methods}\label{sec:numericalmethods}
Differential equations are part of the fundamental language of physics, from their origins describing motion in Newton's laws to the MHD equations which describe the dynamics of plasma in the Sun. 
Often coupled systems of differential equations cannot be solved analytically either at all or without making large assumptions about the nature of the system, such as the geometry.
The most widely known system of equations that cannot be solved analytically is the three body problem, where, described by Newton's laws, three masses interact with each other via gravity.
The numerical solutions to this problem allowed the revolution in modern space flight, and the launch of the two Voyager probes on their `grand tour' of the solar system.
Numerical solutions to the MHD equations are widely used to gain an insight into systems which are too complex to allow analytical solutions of the equations.
This section covers the basic principals of numerical approximations to differentials and some of the limitations of this method.

A differential is the gradient of a function over an infinitesimally small range.
The numerical approximation takes this range and makes it finite, calculating the differential from an approximation of some kind, the most simple being a finite difference approximation.
The finite difference approach is the one used in this thesis and commonly for computational fluid dynamics problems, it approximates the differential by calculating the gradient over a finite length in space.
In multiple dimensions this changes to calculating the derivative over a finite grid, the resolution of which determines the accuracy of the solution.




\subsection{Finite Difference Method}

The definition of a derivative is $f'(x)=\lim\limits _{\Delta_{x}\to0}\frac{f(x+\Delta_{x})-f(x)}{\Delta_{x}}$; the finite difference method approximates this equation by making $\Delta_{x}$ finite.
The general form of a finite difference equation is $f(x+b)-f(x+a)$. The two simplest variations on this equation are the forward and backward differences, $\Delta_{+}f=f(x+\Delta_x)-f(x)$ and $\Delta_{-}f=f(x-\Delta_x)-f(x)$, respectively.
These two types of equations can be used to approximate the derivative, $f'(x)\approx\frac{f(x+\Delta_{x})-f(x)}{\Delta_{x}}$ for a forward difference and $f'(x)\approx\frac{f(x)-f(x-\Delta_{x})}{\Delta_{x}}$ for a backward difference.

These equations can be derived from a Taylor expansion of $f(x\pm\Delta_{x})$,
\begin{align}
f(x-\Delta_{x}) & = f(x)-\Delta_{x}f'(x)+\frac{\Delta_{x}^{2}f''(x)}{2!}-\frac{\Delta_{x}^{3}f'''(x)}{3!}+\frac{\Delta_{x}^{4}f^{iv}(x)}{4!}+...\label{eq:TaylorForward}\\
f(x+\Delta_{x}) & = f(x)+\Delta_{x}f'(x)+\frac{\Delta_{x}^{2}f''(x)}{2!}+\frac{\Delta_{x}^{3}f'''(x)}{3!}+\frac{\Delta_{x}^{4}f^{iv}(x)}{4!}+...\label{eq:TaylorBackward}
\end{align}


The forward difference equation is the first-order truncation of the $f(x+\Delta_{x})$ Taylor series, and likewise for the backward difference.
From this it can be seen that the truncation error of a forward and backward difference approximation is $O(\Delta_{x})$, and that the accuracy could be improved by increasing the number of terms included.

Another way to fundamentally reduce the error, while maintaining the first-order nature of these solutions is to combine the forward and backward difference into a central difference approximation of the form  $f(x)=\frac{1}{2} \left(f(x + \Delta x) + f(x - \Delta x)\right)$.
To first order the result is 
\begin{equation}
f'(x)=\frac{f(x+\Delta_{x})-f(x-\Delta_{x})}{2\Delta_{x}}.\label{eq:First Order CD}
\end{equation}
This formulation has error $O(\Delta_{x}^{2})$ due to the combination of the two Taylor series expansions.
The accuracy of the solution is important, for it determines how well the computed solution represents the true solution.
The most obvious way to increase the accuracy of the solution is to increase the number of terms included from the Taylor expansion of the forward and backward differences.

The Sheffield Advanced Code (SAC), described in \cref{sec:SAC} uses the 4th order central difference scheme to calculate the spatial derivatives.
Below the derivation of a fourth-order central difference scheme in one dimension is given.
Starting from \cref{eq:TaylorForward,eq:TaylorBackward} and subtracting the second from the first, to the fourth order gives,
\begin{equation}
f(x+\Delta_{x})-f(x-\Delta_{x})=2\Delta_{x}f'(x)+\frac{2\Delta_{x}^{3}f'''(x)}{3!}+O(\Delta_{x}^{4}).\label{eq:centraldifferencedx}
\end{equation}
The second step is then to calculate the same subtraction for $2\Delta_{x}$ which can be written as
\begin{equation}
f(x+2\Delta_{x})-f(x-2\Delta_{x})=4\Delta_{x}f'(x)+\frac{16\Delta_{x}^{3}f'''(x)}{3!}+O(\Delta_{x}^{4}).\label{eq:CentralDifference2dx}
\end{equation}
Finally, subtracting \cref{eq:CentralDifference2dx} from $8\times$(\cref{eq:centraldifferencedx}) and rearranging for $f'(x)$ gives
\begin{equation}
f'(x)=\frac{8f(x+\Delta_{x})-8f(x-\Delta_{x})-f(x+2\Delta_{x})+f(x-2\Delta_{x})}{12\Delta_{x}}+O(\Delta_{x}^{4}),\label{eq:4thOrderCentralDifferenceUniform}
\end{equation}
which is the fourth order central difference scheme in one dimension for a uniform spacing of $\pm\Delta_{x}$.
This scheme can be expanded into $n$ dimensions by using the basic property of differentiation $\frac{\partial^{2}u}{\partial x\partial y}=\frac{\partial}{\partial x}\left(\frac{\partial u}{\partial y}\right)=\frac{\partial}{\partial y}\left(\frac{\partial u}{\partial x}\right)$.
This scheme provides good accuracy while being computationally efficient.
However, as discussed in the next section, it suffers with some numerical stability constraints.


\subsubsection{Numerical Stability}

Due to the nature of the approximations used in numerical calculation of derivatives, some schemes are limited in their ability to provide an accurate solution in certain conditions.
There is a wide variety of methods used to numerically approximate derivatives, this section has dealt with one of the most conceptually and mathematically simple methods.
This simplicity means it can be implemented in a very computationally efficient manner, however, it also means that the solution can become unstable.

Instability in the calculation of derivatives occurs when a small error in the calculation of the derivative at one point on the mesh is amplified as the calculation proceeds on to subsequent points.
This error can be removed in one of two primary ways; it can be avoided altogether by employing a different numerical approximation to the derivative, normally one better suited to the properties of the equations or physical parameters of the solution; or it can be numerically suppressed by artificially compensating for it.
The field of developing numerical approximations to derivatives is very large, and there are many different options available.
The approach the SAC code takes is to use additional terms in the MHD equations to counteract any instability, this is discussed in \cref{sec:SAC}.


\section{Sheffield Advanced Code}\label{sec:SAC}

This thesis uses the Sheffield Advanced Code (SAC)~\citep{shelyag2008} for all MHD simulations.
The SAC code solves the ideal MHD equations, for a plasma with static background conditions.
This approach enables the code to solve for small perturbations, while the background conditions vary by orders of magnitude through the solar atmosphere.
The successful application of this approach is clearly dependant upon a realistic, magnetohydrostatic background condition.
The construction of this background is described in \cref{sec:mhsbackground}.

The SAC code was developed to simulate MHD waves as solutions to the ideal MHD equations given in \cref{eq:imhd_mass,eq:imhd_induction} with a static background.
Therefore, the equations solved by SAC are a rearranged version of the ideal MHD equations and are given below~\citep[taken from][]{shelyag2008},
\begin{equation}
\frac{\partial\tilde{\rho}}{\partial t}+\nabla\cdot[\mathbf{v}(\rho_{b}+\tilde{\rho})]=0+D_{\rho}(\tilde{\rho}),
\end{equation}
\begin{equation}
\frac{\partial[(\rho_{b}+\tilde{\rho})\mathbf{v}]}{\partial t}+\nabla\cdot[\mathbf{v}(\rho_{b}+\tilde{\rho})\mathbf{v}-\mathbf{\tilde{B}\tilde{B}}]-\nabla[\mathbf{\tilde{B}}\mathbf{B}_{b}+\mathbf{B}_{b}\mathbf{\tilde{B}}]+\nabla\tilde{p_{t}}=\tilde{\rho}\mathbf{g}+D_{\rho v}[(\tilde{\rho}+\rho_{b})\mathbf{v}],
\end{equation}
\begin{equation}
\frac{\partial\tilde{e}}{\partial t}+\nabla\cdot[\mathbf{v}(\tilde{e}+e_{b})-\mathbf{\tilde{B}\tilde{B}}\cdot\mathbf{v}+\mathbf{v}\tilde{p_{t}}]-\nabla[(\mathbf{\tilde{B}B}_{b}+\mathbf{B}_{b}\mathbf{\tilde{B}})\cdot\mathbf{v}]+p_{tb}\nabla\mathbf{v}-\mathbf{B}_{b}\mathbf{B}_{b}\nabla\mathbf{v}=\tilde{\rho}\mathbf{g}+D_{e}(\tilde{e}),
\end{equation}
\begin{equation}
\frac{\partial\mathbf{\tilde{B}}}{\partial t}+\nabla\cdot[\mathbf{v}(\mathbf{\tilde{B}}+\mathbf{B}_{b})-(\mathbf{\tilde{B}}+\mathbf{B}_{b})\mathbf{v}=0+D_{B}(\mathbf{\tilde{B}}).
\end{equation}
Where:
\begin{equation}
\tilde{p_{t}}=\tilde{p_{k}}+\frac{\tilde{\mathbf{B}}^{2}}{2}+\mathbf{B}_{b}\mathbf{\tilde{B}},
\end{equation}
is the total perturbation pressure, and: 
\begin{equation}
\tilde{p}_{k}=(\gamma-1)\left(\tilde{e}-\frac{(\tilde{\rho}+\rho_{b})\mathbf{v}}{2}-\mathbf{B}_b\mathbf{\tilde{B}}+-\frac{\tilde{\mathbf{B}}^{2}}{2}\right),
\end{equation}
so,
\begin{equation}
\tilde{p}_{t}=(\gamma-1)\left[\tilde{e}-\frac{(\tilde{\rho}+\rho_{b})\mathbf{v}^{2}}{2}\right]-(\gamma-2)\left(\mathbf{B}_{b}\mathbf{\tilde{B}}+\frac{\tilde{\mathbf{B}}^{2}}{2}\right),
\end{equation}
and,
\begin{equation}
p_{tb}=(\gamma-1)e_{b}-(\gamma-2)\frac{\mathbf{B}_{b}^{2}}{2}
\end{equation}
is total background pressure.

The first major change between these equations and the ones given in \cref{eq:imhd_mass,eq:imhd_induction} is the addition of the background and perturbation variables for each term.
This allows the study of the perturbations on top of a static background.
Another, more subtle, change is the move to the energy density per-unit volume $e$ and more importantly the addition of various $D$ terms.

The $D$ terms in the SAC equations are hyper-diffusion or hyper-viscous terms that enforce numerical stability upon the equations, they are adapted from~\cite{nordlund1995} and shall not be replicated here in detail, however, they are described in \citet{shelyag2008}.
The function of the diffusive and viscous terms are to smooth out any deviations from the solution on a local scale to a larger scale, thereby reducing the compounding of the error keeping the solution stable.
There are secondary effects from using this type of numerical stability enforcement, primarily that if the diffusion terms are too high it can lead to damping of the actual solution and deviation of the approximation from the correct solution, also in regions of rapid change such as shocks it adds an extra non-physical diffusion to the shock.
However by choosing a minimum value of the coefficients to maintain stability these effects can be reduced to an acceptable level, especially for the linear wave studies presented in this thesis.

