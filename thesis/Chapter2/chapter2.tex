%*****************************************************************************************
%*********************************** Second Chapter ***************************************
%*****************************************************************************************
\label{ch:Background}
\chapter{Background}  %Title of the Second Chapter

\begin{pycode}[chapter2]
ch2 = texfigure.Manager(pytex, number=2, base_path='./Chapter2/')
\end{pycode}

%{\Large Knowledge at start:}
%\begin{itemize}
%	\item The coronal heating problem
%	\item MHD waves as a solution to the corona heating problem
%	\item photospheric dynamics
%	\item Magnetic wave guides.
%\end{itemize}
%
%{\Large Knowledge at end:}
%\begin{itemize}
%	\item The ideal MHD equations
%	\item Wave solutions for a uniform media
%	\item velocity perturbation calculations
%	\item Wave flux calcs
%	\item SAC, and numerical solutions to the ideal MHD equations
%\end{itemize}

%\section*{Outline}
%
%\begin{enumerate}
%
%	\item MHD
%	\begin{enumerate}
%		\item Ideal MHD
%		\item MHD Waves
%		
%	\end{enumerate}
%	\item Computational Methods
%	\begin{enumerate}
%		\item Numerical Solutions to Partial Differential Equations
%		\item Numerical MHD
%		\item The Sheffield Advanced Code (SAC)
%	\end{enumerate}
%	\item Magneto-static Background Conditions
%	
%\end{enumerate}
	

%********************************** %First Section  **************************************
\section{Magnetohydrodynamics}\label{sec:MHD}

\subsection{MHD Waves}\label{sec:MHDwaves}

\subsection{Velocity Perturbations}\label{sec:Vpert}

\subsection{Calculating Wave Flux}\label{sec:waveflux}


To calculate the relative strengths of the excited waves we compute the `wave energy flux' vector everywhere in the domain using Equation \ref{eq:wave_energy}.
\begin{equation}
\vec{F}_{wave} \equiv \widetilde{p}_k \vec{v} + \frac{1}{\mu_0} \left(\vec{B}_b \cdot \vec{\widetilde{B}}\right) \vec{v} - \frac{1}{\mu_0}\left(\vec{v} \cdot \vec{\widetilde{B}} \right) \vec{B}_b,
\label{eq:wave_energy}
\end{equation}
where a subscript $b$ represents a background variable, a tilda represents a perturbation from the background conditions and $p_k$ represents kinetic pressure.

This equation has been widely used to calculate the energy contained in linear MHD perturbations.
It is discussed in detail in \cite{bogdan2003} where it is compared to the `true' MHD flux for linear perturbations and found to be generally clearer. 
It is used in \cite{vigeesh2009, vigeesh2012, khomenko2012}. 
For a full derivation and discussion relating to time-averaging see \cite{leroy1985}.
Calculating wave energy flux using Equation \ref{eq:wave_energy} provides a vector which is useful in plotting time distance diagrams and analysing wave modes.

\section{Computational Methods}\label{sec:numericalmethods}

\section{Sheffield Advanced Code}\label{sec:SAC}